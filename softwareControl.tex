\chapter{Software de adquisición}
\label{cap2}

Texto explicación introductiva.

\section{Entorno Hardware}
	En un sistema empotrado el software y hardware están muy vinculados. Para entender el funcionamiento del software es muy importante estar
	familiarizado con el diseño y funcionalidad del hardware. Impulsados por esta razón a continuación procederemos a hacer una descripción de
	los aspectos más importantes del hardware que compone nuestro sistema de adquisición. Volvemos a enfatizar que el autor de este trabajo no
	ha formado parte  en la realización de los módulos hardware que serán descritos a continuación.
	\subsection{BeagleBone Black}
		La BeagleBone Black es un computador empotrado, open-source y single board. La placa viene con Linux, distribución Angstrom y versión
		de núcleo 3.8.  Lo más característico de la placa son los 2x46 pines de extensión disponibles. Estos pines son los utilizados para
		integrar este componente con el resto del sistema de adquisición. La mayoría de estos pines son multipropósito, pueden ser
		configurados para tener funcionalidades diversas y además esta configuración puede ser realizada de forma dinámica.
		\par
		La configuración de los pines se realiza mediante el “Device Tree Overlay”, estructura de datos que se utiliza para describir el
		hardware. Para la realización de este trabajo no tendremos que lidiar con este mecanismo, utilizaremos una librería que nos facilite
		el trabajo. La librería utilizada es “Adafruit BeagleBone IO Python”. Esta librería nos ayudara a configurar dinámicamente los pines
		de extensión, además la librería ofrece métodos para operar sobre estos una vez configurados. En la figura 1 podemos ver las
		funcionalidades que los pines de la BeagleBone Black ofrecen y las que vamos a utilizar para este trabajo.
		\begin{table}[h]
	  		\begin{tabular}{|l|l|}
	    			\hline
	    			\rowcolor[HTML]{C0C0C0} 
	    			{\color[HTML]{000000} \textbf{Disponibles}} & {\color[HTML]{000000} \textbf{Usados}} \\ \hline
	    				7 Analog Pins                               & 4 Analog Pins                          \\ \hline
	    				65 Digital Pins at 3.3V                     & 1 Digital Pin                          \\ \hline
	    				2x I2C                                      & 0x I2C                                 \\ \hline
	    				2x SPI                                      & 0x SPI                                 \\ \hline
	    				4 Timers                                    & 0 Timers                               \\ \hline
	    				4x UART                                     & 2x UART                                \\ \hline
	    				8x PWM                                      & 0x PWM                                 \\ \hline
	    				A/D Converter                               & Not used                               \\ \hline
	  		\end{tabular}
			\centering
			\caption{BeagleBone Black Headers.}
			\label{fig:BBBPins}
		\end{table}
	\subsection{FPGA}
		Intro FPGA. La FPGA es el componente que más interactuar con la BeagleBone Black.
		Intro, Reset, UARTS 

\section{Configuración del sistema}
	System Services. Como funcionan. Como las usamos.
	\subsection{WatchDog}
	\subsection{Time Sync}
	\subsection{Software de adquisición}

\section{Sensor Manager}
	\subsection{Pressure}
	\subsection{HVPS}

\section{Reader}
	Explicar funcionamiento.

\section{Counts Pettioner}
	Intro funcionamineto.
	\subsection{Deriva local}
	\subsection{Petición de cuentas}
	\subsection{Petición de datos sensores}
	\subsection{Guardar datos}

\section{DbUpdater}

\section{testing}
	Porque son importantes los unit tests.
	\subsection{moking}
	\subsection{Coverage}

\section{Mantenimiento}
	Dejar el sistema de adquisición de datos funciónando algun tiempo. Hacer analisis de los datos. Describir experiencia.  TODO dejar para el capítuo de Conclusiones.
