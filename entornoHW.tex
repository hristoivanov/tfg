\chapter{Entrono Hardware}
\label{entornoHW}

En un sistema empotrado el software y hardware están muy vinculados. Para entender el funcionamiento del software de adquisición es muy importante estar
familiarizado con el diseño y funcionalidad del hardware. Impulsados por esta razón en este capítulo procederemos a hacer una descripción de
los aspectos más importantes del hardware que compone nuestro sistema de adquisición. Volvemos a enfatizar que el autor de este trabajo no
ha formado parte  en la realización de los módulos hardware que serán descritos a continuación.
\section{BeagleBone Black}
	La BeagleBone Black\cite{Beagle} es un computador empotrado, open-source y single board. La placa viene con Linux, distribución Angstrom y versión
	de núcleo 3.8. Lo más característico de la placa son los 2x46 pines de extensión disponibles. Estos pines son los utilizados para
	integrar este componente con el resto del sistema de adquisición. La mayoría de estos pines son multipropósito, pueden ser
	configurados para tener funcionalidades diversas y además esta configuración puede ser realizada de forma dinámica\cite{BeagleWiki}.
	\par
	La configuración de los pines se realiza mediante el \emph{Device Tree Overlay}, estructura de datos que se utiliza para describir el
	hardware. Para la realización de este trabajo no tendremos que lidiar con este mecanismo, utilizaremos una librería que nos facilite
	el trabajo. La librería utilizada es \emph{Adafruit BeagleBone IO Python}\cite{AdaFruitGit}. Esta librería nos ayudara a configurar dinámicamente los pines
	de extensión, además la librería ofrece métodos para operar sobre estos una vez configurados. En la tabla \ref{tab:BBBPins} podemos
	ver las funcionalidades que los pines de la BeagleBone Black ofrecen y las que vamos a utilizar para este trabajo.
	\begin{table}[h]
		\begin{tabular}{|l|l|}
			\hline
			\rowcolor[HTML]{C0C0C0} 
			{\color[HTML]{000000} \textbf{Disponibles}} & {\color[HTML]{000000} \textbf{Usados}} \\ \hline
				7 Analog Pins                               & 4 Analog Pins                          \\ \hline
				65 Digital Pins at 3.3V                     & 1 Digital Pin                          \\ \hline
				2x I2C                                      & 0x I2C                                 \\ \hline
				2x SPI                                      & 0x SPI                                 \\ \hline
				4 Timers                                    & 0 Timers                               \\ \hline
				4x UART                                     & 2x UART                                \\ \hline
				8x PWM                                      & 0x PWM                                 \\ \hline
				A/D Converter                               & Not used                               \\ \hline
		\end{tabular}
		\centering
		\caption{BeagleBone Black Headers.}
		\label{tab:BBBPins}
	\end{table}
\subsection{FPGA}
	Intro FPGA. La FPGA es el componente que más interactuar con la BeagleBone Black. \ref{tab:FPGAUart1}
	Intro, Reset, UARTS 
	\begin{table}[h]
		\tiny
		\begin{tabularx}{\textwidth}{|l|c|c|X|c|c|c|c|c|}
			\hline
			\rowcolor[HTML]{C0C0C0} 
			\multicolumn{1}{|r|}{\textbf{Bit}}    	& 7 & 6          & 5 				& 4 	       & 3 	     & 2 	  & 1          & 0 	     \\ \hline
			\cellcolor[HTML]{C0C0C0}\textbf{Byte 1} & 0 & 1          & 0  				& \multicolumn{5}{c|}{Canal (0-17)}				     \\ \hline
			\cellcolor[HTML]{C0C0C0}\textbf{Byte 2} & 1 & Cuenta (6) & Cuenta (5)      		& Cuenta (4)   & Cuenta (3)  & Cuenta (2) & Cuenta (1) & Cuenta (0)  \\ \hline
			\cellcolor[HTML]{C0C0C0}\textbf{Byte 3} & 1 & X          & Nivel ('1'->alto, '0'->bajo) & Cuenta (11)  & Cuenta (10) & Cuenta (9) & Cuenta (8) & Cuenta (7)  \\ \hline
		\end{tabularx}
		\caption{FPGA. Palabra de ancho de pulso}

		\begin{tabularx}{\textwidth}{|l|c|X|X|X|X|X|X|X|}
			\hline
			\rowcolor[HTML]{C0C0C0} 
			\multicolumn{1}{|r|}{\textbf{Bit}} 	& 7 & 6 		       & 5 		       & 4		       & 3 		       & 2		       & 1          	       & 0			\\ \hline
			\cellcolor[HTML]{C0C0C0}\textbf{Byte 1} & 0 & 0                        & OverFlow FIFO Tubo 5  & OverFlow FIFO Tubo 4  & OverFlow FIFO Tubo 3  & OverFlow FIFO Tubo 2  & OverFlow FIFO Tubo 1  & OverFlow FIFO Tubo 0	\\ \hline
			\cellcolor[HTML]{C0C0C0}\textbf{Byte 2} & 1 & OverFlow FIFO Tubo 12    & OverFlow FIFO Tubo 11 & OverFlow FIFO Tubo 10 & OverFlow FIFO Tubo 9  & OverFlow FIFO Tubo 8  & OverFlow FIFO Tubo 7  & OverFlow FIFO Tubo 6	\\ \hline
			\cellcolor[HTML]{C0C0C0}\textbf{Byte 3} & 1 & Almost Full FIFO General & OverFlow FIFO General & OverFlow FIFO Tubo 17 & OverFlow FIFO Tubo 16 & OverFlow FIFO Tubo 15 & OverFlow FIFO Tubo 14 & OverFlow FIFO Tubo 13	\\ \hline
		\end{tabularx}
		\caption{FPGA. Palabra de estado}

		\begin{tabularx}{\textwidth}{|l|X|c|c|c|c|c|c|c|}
			\hline
			\rowcolor[HTML]{C0C0C0} 
			\multicolumn{1}{|r|}{\textbf{Bit}}    	& 7 & 6           & 5 		& 4 	      & 3 	    & 2 	 & 1           & 0 	     	\\ \hline
			\cellcolor[HTML]{C0C0C0}\textbf{Byte 1} & 0 & 1           & 1  		& X	      & X	    & X	  	 & X	       & X	     	\\ \hline
			\cellcolor[HTML]{C0C0C0}\textbf{Byte 2} & 1 & Cuenta0(6)  & Cuenta0(5) 	& Cuenta0(4)  & Cuenta0(3)  & Cuenta0(2) & Cuenta0(1)  & Cuenta0(0)  	\\ \hline
			\cellcolor[HTML]{C0C0C0}\textbf{Byte 3} & 1 & Cuenta0(13) & Cuenta0(12)	& Cuenta0(11) & Cuenta0(10) & Cuenta0(9) & Cuenta0(8)  & Cuenta0(7)  	\\ \hline
			\cellcolor[HTML]{C0C0C0}\textbf{Byte 4} & 1 & X		  & X	 	& X	      & X	    & X		 & Cuenta0(15) & Cuenta0(14)	\\ \hline
		\end{tabularx}
		\caption{FPGA. Palabra de datos}
		\label{tab:FPGAUart1}
	\end{table}


	Intro FPGA. La FPGA es el componente que más interactuar con la BeagleBone Black.
	Intro, Reset, UARTS 
