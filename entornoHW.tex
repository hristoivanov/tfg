\chapter{Entrono Hardware}
\label{entornoHW}

En un sistema empotrado el software y hardware están muy vinculados. Para entender el funcionamiento del software de adquisición es muy importante estar
familiarizado con el diseño y funcionalidad del hardware. Impulsados por esta razón en este capítulo procederemos a hacer una descripción de
los aspectos más importantes del hardware que compone nuestro sistema de adquisición. Volvemos a enfatizar que el autor de este trabajo no
ha formado parte  en la realización de los módulos hardware que serán descritos a continuación.
\section{BeagleBone Black}
	BeagleBone Black\cite{Beagle}\cite{BeagleWiki} es un computador empotrado, open-source y single board. La placa viene con Linux, distribución Angstrom y versión
	de núcleo 3.8. En la figura \ref{fig:beaglebone} podemos ver un diagrama de bloques que refleja los componentes hardware que componen la
	BeagleBone. A continuación vamos a hacer una breve descripción de los módulos que utilizaremos para este trabajo.
	\begin{itemize}
	  \item 	ARM CORTEX A8\cite{BeagleCore} y SDRAM. El procesador es suficientemente potente para soportar la distribución Linux y satisfacer
	    		las necesidades de nuestro software. Los 512MB DDR3 de memoria RAM son suficientes para los propósitos de este trabajo.
		\item 	Ethernet Connector. El conector RJ-45 nos permitirá conectarnos a Internet. Una vez establecida la conexión podremos transmitir
		  	los datos al NMDB. La conexión también nos permitirá conectarnos a la placa vía SSH, de esta manera un operador podría realizar
			operaciones de mantenimiento de forma remota. 
		\item	eMMC y microSD. Utilizaremos la memoria integrada (eMMC) de 4GB para albergar el sistema operativo y el software de adquisición.
		  	La microSD la utilizaremos para guardar los datos y logs producidos por el software de adquisición.
		\item 	Analog Pins. Los 7 pines analógicos nos permiten usar sensores cuyo output es una señal analógica. Ejemplo de estos sensores son
		  	las fuentes de alimentación analógicas o los sensores de temperatura, en algunas estaciones la temperatura ambiente se monitoriza
			también. Estos sensores producen señales cuya tensión eléctrica es proporcional al valor de la magnitud medida. 
		\item 	GPIOs. Las GPIO nos permiten trabajar con señales digitales, tanto de entrada como de salida. Un ejemplo de uso es la señal de
		  	Reset de la FPGA, señal digital activa a bajo nivel.
		\item	UARTs. La placa ofrece 4 puertos series de los que utilizaremos 2 para comunicarnos con la FPGA.
	\end{itemize}
	\begin{figure}[h]
		\centering
		\includegraphics[keepaspectratio, width=1\textwidth]{./img/beaglebone.png}
		\caption{BeagleBone Black. Diagrama de bloques hardware}
		\label{fig:beaglebone}
	\end{figure}
	Fijándonos en la figura \ref{fig:beaglebone} podemos ver que muchos de los módulos son accesibles mediante los pines de extensión\cite{BeagleWikiExp}.
	La placa ofrece 2x46 pines de los cuales muchos son multipropósito, pueden ser configurados para tener funcionalidades diversas y además esta
	configuración puede ser realizada de forma dinámica. La configuración de los pines se realiza mediante el \emph{Device Tree Overlay}, estructura de
	datos que se utiliza para describir el hardware. Para la realización de este trabajo no tendremos que lidiar con este mecanismo, utilizaremos una
	librería que nos facilite el trabajo. La librería utilizada es \emph{Adafruit BeagleBone IO Python}\cite{AdaFruitGit}. Esta librería nos ayudara a
      	configurar dinámicamente los pines de extensión, además la librería ofrece métodos para operar sobre estos una vez configurados.


\section{FPGA}
	En esta sección vamos a hacer una descripción de la FPGA. En la figura \ref{fig:fpga} podemos ver un diagrama de bloques que refleja los módulos
	funcionales y las interfaces que posee la FPGA.
	\begin{figure}[h]
		\centering
		\includegraphics[keepaspectratio, width=1\textwidth]{./img/fpga.png}
		\caption{FPGA. Diagrama de bloques.}
		\label{fig:fpga}
	\end{figure}	
	\subsection{Procesado de pulsos}
		Como ya hemos comentado en el capítulo \ref{cap1} la FPGA se encarga de procesar las señales provenientes de los tubos, una vez procesadas
		las señales, la información es transmitida a la BeagleBone Black. Esta transmisión se realiza mediante una UART a la que nos referiremos
		como UART de pulsos. En las tablas \ref{tab:FPGAUartPulso}, \ref{tab:FPGAUartOver}, \ref{tab:FPGAUartCont} podemos ver el formato de los
		datos transmitidos. Vemos que hay tres diferentes mensajes que la FPGA nos puede transmitir.
		\begin{itemize}
			\item	En la tabla \ref{tab:FPGAUartPulso} podemos ver el formato del primer mensaje que vamos a discutir. Este mensaje se
			  	constituye de tres bytes que nos dan información sobre el ancho de un pulso. El mensaje nos permite identificar el canal
				en el que se ha producido el pulso, el nivel (alto/bajo) y la longitud de este. Los pulsos de nivel alto representan la
				detección de una partícula por los tubos contadores, el ancho del pulso es proporcional a la energía de la partícula
				detectada. La longitud de los pulsos de nivel bajo también es mediada para poder identificar pulsos con pequeña
				separación temporal.
			\item 	En la siguiente tabla \ref{tab:FPGAUartOver} podemos ver el mensaje de estado que la FPGA genera. Dado que la UART usa
			  	una FIFO de tamaño fijo es posible que la FIFO se llene y se pierdan mensajes. En este caso la FPGA genera un mensaje de
				estado que es transmitido para indicarle al software que se han producido perdidas de datos.
			\item	Además de medir el ancho de los pulsos la PFGA lleva la cuenta de pulsos recibidos para cada canal. Esta información
			  	puede ser solicitada por la BeagleBone Black utilizando el comando apropiado, ver tabla \ref{tab:FPGAUartComm}. En la
				tabla \ref{tab:FPGAUartCont} podemos ver el formato en el que se transmite la información de las cuentas. Los bytes
				2, 3, 4 son transmitidos 18 veces, una vez para cada canal. Después de transmitir la información de las cuentas la FPGA
				reinicia los contadores para cada canal.
		\end{itemize}
		\begin{table}[h]
			\tiny
			\begin{tabularx}{\textwidth}{|l|c|c|X|c|c|c|c|c|}
				\hline
				\rowcolor[HTML]{C0C0C0} 
				\multicolumn{1}{|r|}{\textbf{Bit}}    	& 7 & 6          & 5 				& 4 	       & 3 	     & 2 	  & 1          & 0 	     \\ \hline
				\cellcolor[HTML]{C0C0C0}\textbf{Byte 1} & 0 & 1          & 0  				& \multicolumn{5}{c|}{Canal (0-17)}				     \\ \hline
				\cellcolor[HTML]{C0C0C0}\textbf{Byte 2} & 1 & Cuenta (6) & Cuenta (5)      		& Cuenta (4)   & Cuenta (3)  & Cuenta (2) & Cuenta (1) & Cuenta (0)  \\ \hline
				\cellcolor[HTML]{C0C0C0}\textbf{Byte 3} & 1 & X          & Nivel ('1'->alto, '0'->bajo) & Cuenta (11)  & Cuenta (10) & Cuenta (9) & Cuenta (8) & Cuenta (7)  \\ \hline
			\end{tabularx}
			\caption{UART pulsos. Palabra de ancho de pulso}
			\label{tab:FPGAUartPulso}
			\begin{tabularx}{\textwidth}{|l|c|X|X|X|X|X|X|X|}
				\hline
				\rowcolor[HTML]{C0C0C0} 
				\multicolumn{1}{|r|}{\textbf{Bit}} 	& 7 & 6 		       & 5 		       & 4		       & 3 		       & 2		       & 1          	       & 0			\\ \hline
				\cellcolor[HTML]{C0C0C0}\textbf{Byte 1} & 0 & 0                        & OverFlow FIFO Tubo 5  & OverFlow FIFO Tubo 4  & OverFlow FIFO Tubo 3  & OverFlow FIFO Tubo 2  & OverFlow FIFO Tubo 1  & OverFlow FIFO Tubo 0	\\ \hline
				\cellcolor[HTML]{C0C0C0}\textbf{Byte 2} & 1 & OverFlow FIFO Tubo 12    & OverFlow FIFO Tubo 11 & OverFlow FIFO Tubo 10 & OverFlow FIFO Tubo 9  & OverFlow FIFO Tubo 8  & OverFlow FIFO Tubo 7  & OverFlow FIFO Tubo 6	\\ \hline
				\cellcolor[HTML]{C0C0C0}\textbf{Byte 3} & 1 & Almost Full FIFO General & OverFlow FIFO General & OverFlow FIFO Tubo 17 & OverFlow FIFO Tubo 16 & OverFlow FIFO Tubo 15 & OverFlow FIFO Tubo 14 & OverFlow FIFO Tubo 13	\\ \hline
			\end{tabularx}
			\caption{UART pulsos. Palabra de estado}
			\label{tab:FPGAUartOver}
			\begin{tabularx}{\textwidth}{|l|X|c|c|c|c|c|c|c|}
				\hline
				\rowcolor[HTML]{C0C0C0} 
				\multicolumn{1}{|r|}{\textbf{Bit}}    	& 7 & 6           & 5 		& 4 	      & 3 	    & 2 	 & 1           & 0 	     	\\ \hline
				\cellcolor[HTML]{C0C0C0}\textbf{Byte 1} & 0 & 1           & 1  		& X	      & X	    & X	  	 & X	       & X	     	\\ \hline
				\cellcolor[HTML]{C0C0C0}\textbf{Byte 2} & 1 & Cuenta0(6)  & Cuenta0(5) 	& Cuenta0(4)  & Cuenta0(3)  & Cuenta0(2) & Cuenta0(1)  & Cuenta0(0)  	\\ \hline
				\cellcolor[HTML]{C0C0C0}\textbf{Byte 3} & 1 & Cuenta0(13) & Cuenta0(12)	& Cuenta0(11) & Cuenta0(10) & Cuenta0(9) & Cuenta0(8)  & Cuenta0(7)  	\\ \hline
				\cellcolor[HTML]{C0C0C0}\textbf{Byte 4} & 1 & X		  & X	 	& X	      & X	    & X		 & Cuenta0(15) & Cuenta0(14)	\\ \hline
			\end{tabularx}
			\caption{UART pulsos. Palabra de datos}
			\label{tab:FPGAUartCont}
			\begin{tabularx}{\hsize}{|c|X|}
		  		\hline
				\rowcolor[HTML]{C0C0C0} 
		  		Commando & Descripción                            \\\hline
		  		0x00     & Configura multiplexor para aparato 1   \\\hline
		  		0x01     & Configura multiplexor para aparato 2   \\\hline
		  		0x02     & Configura multiplexor para aparato 3   \\\hline
		  		0x03     & Configura multiplexor para aparato 4   \\\hline
		  		0x10     & Reset general del sistema              \\\hline
		  		0x11     & Solicita la transmisión de las cuentas \\\hline
			\end{tabularx}
			\caption{UART pulsos. Commandos}
			\label{tab:FPGAUartComm}
		\end{table}
	\subsection{Multiplexor}
		Entre la FPGA y la BeagleBone Black hay una segunda UART a la que nos referiremos como UART de extensión. Esta UART tiene como propósito
		extender el número de UART que podemos conectar a nuestro sistema. La UART de extensión está conectada a un multiplexor implementado en la
		FPGA. El multiplexor tiene como entradas 4 UARTs. Podemos seleccionar una entrada diferente mandando comandos a la FPGA por la UART de pulsos,
		de acuerdo a la especificación de la tabla \ref{tab:FPGAUartComm}. Esta extensión nos permite conectar más sensores que requieran una UART.
		Un ejemplo es el barómetro \emph{BM35}\cite{BM35} utilizado en CALMA.
	\subsection{Reset}
		La FPGA nos permite realizar un Reset del sistema. Todas las variables son puestas a sus valores iniciales, excepto el multiplexor que mantiene
		su estado. Existen dos maneras de realizar este Reset. La primera es mediante un comando trasmitido por la UART de  pulsos, de acuerdo a la
		especificación presentada en la tabla \ref{tab:FPGAUartComm}. La segunda manera de realizar este Reset es mediante una señal digital activa a
		bajo nivel. La señal mencionada está conectada a la BeagleBone Black por lo que es posible realizar el Reset desde nuestro software de adquisición. 
