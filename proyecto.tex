\documentclass[twoside,spanish,a4paper,openright,11pt]{book}
\usepackage[spanish]{babel}
\usepackage[utf8]{inputenc}
\usepackage{graphics}
\usepackage{graphicx}
\usepackage{url}
\usepackage[T1]{fontenc}
\usepackage{fancyhdr}
\usepackage{hyperref}
\usepackage{color}
\usepackage{colortbl}
\usepackage{listings}
\usepackage{colourlist}
\usepackage{booktabs}
\usepackage{amssymb}
\usepackage[acronym, nonumberlist]{glossaries}
\usepackage[table,xcdraw]{xcolor}
\usepackage{rotating}
\usepackage{tabularx}
%\usepackage[caption=false]{subfig} 

\makeglossaries


%Configuración del paquete hyperref
% colores definidos
\definecolor{mygray}{rgb}{.86,.86,.86}
\definecolor{myblue}{rgb}{.20,.20,.6}
\definecolor{gray97}{gray}{.97}
\definecolor{gray75}{gray}{.75}
\definecolor{gray45}{gray}{.45}
\definecolor{whitesmoke}{rgb}{0.96,0.96,0.96}

% configuración del paquete hyperref
\hypersetup{   pdfauthor = {Hristo Ivanov Ivanov}
               pdftitle = {Software empotrado de control y gestión de un monitor de neutrones},
               pdfkeywords = {NMDB, CALMA, TFC},
               pdfsubject = {Proyecto Final de Carrera},
               colorlinks={true},
               pdfstartview={FitV},
               linkcolor={myblue},
               citecolor={myblue},
               urlcolor={myblue}
}





%Nos permite definir los márgenes como queramos del documento completo
\usepackage{anysize}
\marginsize{3.5cm}{3cm}{2cm}{2cm}

%Con esta definición pordemos cambiar los márgenes sobre la marcha utilizando
% \begin{changemargin}{7cm}{0cm}
%	.. .. ..
% \end{changemargin}
\def\changemargin#1#2{\list{}{\rightmargin#2\leftmargin#1}\item[]}
\let\endchangemargin=\endlist





% Control de lneas viudas y huérfanas
\clubpenalty=100000
\widowpenalty=10000
\displaywidowpenalty=1000
\looseness=1

%-----------------------------------------------------------------------------%
%
% Formato de las cabeceras
%
\pagestyle{fancyplain}

\lhead[
        \fancyplain{}
	    {\textrm{\textbf{\thepage}\hspace{4mm}\small\leftmark}}]
        {\fancyplain{}
        {\textrm{\footnotesize\ GSO}}}
\rhead[
        \fancyplain{}
        {\textrm{\footnotesize\ Hristo Ivanov Ivanov}}]
        {\fancyplain{}
        {\textrm{\small\rightmark\hspace{4mm}\normalsize\textbf{\thepage}}}}
\cfoot[]{} % En el pie de página no ponemos nada
\renewcommand{\headrulewidth}{0.1mm}

\graphicspath{{img/}}



%Definiciónes para la representación de consola, ficheros y código
\usepackage{listings}
\lstset{ frame=Ltb,
     framerule=.2pt,
     aboveskip=0.5cm,
     framextopmargin=3pt,
     framexbottommargin=3pt,
     framexleftmargin=0.4cm,
     framesep=0pt,
     rulesep=.4pt,
 %    backgroundcolor=\color{gray97},
 %    rulesepcolor=\color{black},
     %
     stringstyle=\ttfamily,
     showstringspaces = false,
     basicstyle=\small\ttfamily,
     commentstyle=\color{gray45},
     keywordstyle=\bfseries,
     %
     numbers=left,
     numbersep=15pt,
     numberstyle=\tiny,
     numberfirstline = false,
     breaklines=true
   }
 
% minimizar fragmentado de listados
\lstnewenvironment{listing}[1][]
   {\lstset{#1}\pagebreak[0]}{\pagebreak[0]}
 
\lstdefinestyle{consola}
   {basicstyle=\scriptsize\bf\ttfamily,
    backgroundcolor=\color{gray75},
   }

\lstdefinestyle{Fichero}
   {basicstyle=\scriptsize\bf\ttfamily,
    backgroundcolor=\color{whitesmoke},
   }
 
\lstdefinestyle{C}
   {language=C
   }
\lstdefinestyle{Java}
   {language=Java,
moredelim={**[il][\rlap{\textcolor{azure2}{\rule[-4pt]{\linewidth}{\baselineskip}}}]{--}}
%Esta última línea se pone para que cuando se añade delante de una línea de código se resalta 
%poniéndola en el color indicado
   }


\begin{document}

\newacronym{cme}{CME}{Eyección de masa coronal}

\newacronym{nmdb}{NMDB}{Neutron Monitor Database}

\newacronym{gle}{GLE}{Ground level enhancements}

\newacronym{calma}{CALMA}{Monitor de neutrones de Castilla-La Mancha}

\newacronym{fpga}{FPGA}{Field Programmable Gate Array}

\newacronym{mvc}{MVC}{Modelo–vista–controlador}

\newacronym{gpio}{GPIO}{General Purpose Input/Output}

\newacronym{uart}{UART}{Universal Asynchronous Receiver-Transmitter}

\newacronym{api}{API}{Application programming interfaces}

\newacronym{rest}{REST}{Representational state transfer}

\newacronym{url}{URL}{Localizador de recursos uniforme}

\newacronym{rpc}{RPC}{Remote Procedure Call}

\glsaddall


\enlargethispage{1cm}
\thispagestyle{empty}

\begin{center}
\includegraphics[width=2cm,keepaspectratio]{../img/logoUAH.jpg}
\par\end{center}

\begin{center}
\textbf{\huge Universidad de Alcalá}
\par\end{center}{\huge \par}

\begin{center}
{\Large Escuela Politécnica Superior}
\par\end{center}{\Large \par}

\medskip{}


\begin{center}
\textbf{\textsc{\huge Grado en Ingeniería de Computadores }}
\par\end{center}{\huge \par}

\medskip{}


\begin{center}
\textbf{\Large Trabajo de Fin de Grado}
\par\end{center}{\Large \par}

\begin{center}
\textbf{\textsc{\LARGE 
  Software empotrado de control y gestión de un monitor de neutrones
}}
\par\end{center}{\LARGE \par}

\medskip{}


\begin{center}
\textbf{\large Autor:}{\large{} Hristo Ivanov Ivanov}
\par\end{center}{\large \par}
\begin{center}
\textbf{\large Tutor:}{\large{} Óscar García Población }
\par\end{center}{\large \par}

\vspace{1.0cm}
\textbf{\Large TRIBUNAL:}{\Large \par}

\vspace{2.0cm}
\textbf{\Large Presidente:}{\Large{} \line(1,0){287}  }{\Large \par} %%TODO

\vspace{2.0cm}
\textbf{\Large Vocal 1:}{\Large{} \line(1,0){307}  }{\Large \par} %%TODO

\vspace{2.0cm}
\textbf{\Large Vocal 2:}{\Large{} \line(1,0){307}  }{\Large \par} %%TODO

\vspace{0.8cm}
\textbf{ Fecha: \line(1,0){75} \hspace{1cm}Calificación: \line(1,0){150}  }{ \par}




\newpage{\pagestyle{empty}\cleardoublepage}


\pagenumbering{arabic}
\setcounter{page}{3}

\setcounter{tocdepth}{3}

\thispagestyle{empty}
\vspace*{4cm}
\begin{flushright}           %%TODO
\textit{Dedicatorias varias \\,   
varias varias varias varias varias varias varias varias varias varias.}
\end{flushright}




\cleardoublepage
\chapter*{Agradecimientos}  %%TODO


\vspace{10mm}
\textit{Muchos agradecimientos a StackOverflow.}



\tableofcontents
\newpage
\listoffigures % Índice de figuras
\newpage
\renewcommand{\listtablename}{Índice de tablas}
\listoftables % Índice de tablas
\newpage

\chapter{Introducción}
\label{cap1}

El propósito de este documento es describir el \emph{Trabajo de Fin de Grado} realizado por el autor. Este es parte del desarrollo de un nuevo sistema de
adquisición de datos para un monitor de neutrones, instrumento que permite medir los niveles de radiación cósmica. Los objetivos del trabajo son la
realización del software encargado de gestionar la adquisición de datos y el desarrollo de una herramienta Web dirigida a la gestión del sistema.	
\par
Empezamos este capítulo haciendo una breve introducción a los conceptos teóricos y el entorno en el que se desarrolla este trabajo. Continuamos
estableciendo los objetivos y el diseño preliminar. Finalmente aclaramos algunas peculiaridades referentes al proceso de adquisición de datos. El
segundo capítulo describe el entorno hardware en el que se desarrolla el software de adquisición. Los siguientes capítulos describen en detalle el
trabajo realizado. El primero de estos es dedicado al software de adquisición. Debido a la separación impuesta por el modelo \emph{Cliente-Servidor}
se utilizarán dos capítulos para describir la herramienta Web, el primero dedicado al \emph{Servidor} y el segundo al \emph{Cliente}. El último
capítulo de este documento recoge las conclusiones obtenidas del trabajo y la visión futura de este. El documento también incluye un apéndice, en este
son descritos los pasos a seguir para implantar el software de adquisición y la herramienta Web. En este apéndice también son reflejados los aspectos
más importantes del proceso de implementación. Finalmente el documento incluye una lista de referencias bibliográficas a las fuentes de información
utilizadas en el trabajo. 

\section{Introducción a los rayos cósmicos}
	A principios del siglo \emph{XX} se descubrió la existencia de los rayos cósmicos. Estos son partículas subatómicas provenientes del espacio
	exterior. Estas partículas, en su mayoría protones y núcleos de Helio, son muy energéticas debido a su gran velocidad. El origen de estas no
	está muy claro, pero sabemos que proceden del espacio exterior. Raras veces la actividad solar puede producir partículas tan energéticas. 
	\par
	Muchas de estas partículas inciden en la atmósfera terrestre. En las capas altas de la atmósfera se producen las primeras interacciones, estas
	partículas colisionan con las partículas que forman la atmósfera. Esta colisión es muy violenta y causa la división de las partículas
	originales en partículas secundarias. Estas a su vez pueden colisionar con otras partículas de la atmósfera para así formar aún más partículas
	secundarias. Vemos como una sola partícula proveniente del espacio produce el fenómeno denominado \emph{cascadas atmosféricas}. Como es de
	esperar con cada choque consecutivo se pierde parte de la energía. Normalmente las partículas secundarias que alcanzan la superficie terrestre
	tan solo tienen una pequeña fracción de la energía inicial. Si una partícula no posee la energía suficiente la cascada que es originada no se
	propaga hasta la superficie terrestre.
	\par
	Como hemos dicho la mayor parte de la radiación cósmica proviene de fuera de nuestro sistema solar, pero está fuertemente relacionada con los
	ciclos solares. Los ciclos solares de 11 años aproximadamente afectan la actividad solar pasando por un mínimo y un máximo, donde los cambios
	son apreciables en la luminosidad y el campo magnético. Es este segundo, el campo magnético solar, el que afecta a la llegada de radiación
	cósmica a la Tierra. Al ser mayormente partículas con carga eléctrica en la presencia de un fuerte campo magnético estas son desviadas. A
	continuación detallamos los sucesos más comunes que pueden ser observados indirectamente a consecuencia de estudiar la cantidad de radiación
	cósmica.
	\begin{itemize}
		\item	Ciclo solar. Como hemos explicado existe una fuerte relación entre la cantidad de radiación cósmica y la actividad solar. La
			radiación cósmica es un buen indicador de la actividad solar donde la relación es inversa. Menos radiación generalmente
			significa una actividad solar elevada.
		\item	\emph{Forbush decrease}\cite{Forbush1938}. Estos sucesos consisten en un descenso rápido de los niveles de radiación cósmica medida
			en la Tierra. Estos descensos son consecuencia de CME's(Eyección de masa coronal). La materia expulsada en un CME al ser en su
			mayoría plasma, extiende e intensifica el campo magnético solar. Como ya hemos explicado el aumento del campo magnético solar
			conlleva al descenso de radiación cósmica.
		\item	\emph{Ground level enhancements}(GLE). Eventualmente la actividad solar es tan elevada que el Sol es capaz de emitir partículas muy
			energéticas. Estas partículas son a veces tan energéticas que pueden generar cascadas atmosféricas que alcanzan la superficie
			terrestre. Estos sucesos son muy raros, entre 10 y 15 por década. A pesar de ser muy raros estos pueden tener un gran impacto
			en nuestras vidas cotidianas, pueden afectar el funcionamiento de la electrónica sensible que está en orbita e incluso la que
			está en tierra.   
	\end{itemize}

\section{Monitor de neutrones}
	Un monitor de neutrones es una estación terrestre que monitoriza la llegada de partículas extraterrestres de forma indirecta a partir de las
	cascadas atmosféricas. Este está compuesto por cuatro capas especialmente diseñadas para capturar las partículas secundarias producidas en las
	cascadas atmosféricas. A continuación procedemos a explicar estas cuatro capas.
	\begin{itemize}
		\item	\textbf{Reflector}. La primera capa, la exterior, consiste en un escudo reflector que tan solo deja pasar las partículas con
			energía alta.  De esta manera todas las partículas generadas por el entorno inmediato que tienen baja energía rebotan y no
			influyen en la medición.
		\item	\textbf{Productor}. Esta capa compuesta generalmente de material denso tiene como objetivo conseguir algo parecido a las
			cascadas atmosféricas. La idea es tener un material denso para que aumente la probabilidad de que las partículas secundarias
			impacten con las partículas del material y como resultado se produzcan aún más partículas. A las partículas generadas en esta
			capa se les da el nombre de neutrones de evaporación. Son estas partículas las que finalmente serán medidas por el
			instrumento, también son las que le dan nombre. Los neutrones producidos tienen menos energía, por lo que son más fáciles de
			medir.
		\item	\textbf{Moderador}. A pesar de que las partículas que tenemos a este nivel tienen tan solo una fracción de la energía
			original, estas aún siguen siendo demasiado energéticas para ser capturadas. Esta capa tiene como objetivo ralentizar,
			disminuir la energía, de las partículas para así poder capturarlas.
		\item	\textbf{Contador}. Un contador proporcional o tubo contador generalmente está relleno de gas con propiedades específicas.
			Cuando el gas interacciona con los neutrones de evaporación este es ionizado, en el proceso son liberados electrones. Debido
			al campo eléctrico que atraviesa constantemente el gas estos electrones son acelerados hacia el ánodo, hilo conductor que
			atraviesa el tubo contador en su centro. Conforme los electrones ganan energía, estos pueden producir ionizaciones
			secundarias. El número total de electrones que llegan al ánodo se mantiene, sin embargo, proporcional a la energía de la
			partícula inicial. Al llegar al ánodo estos electrones producen una corriente eléctrica. 

	\end{itemize}
	\par
	Los sistemas de adquisición están diseñados para recoger estas pequeñas corrientes eléctricas y medirlas. Tradicionalmente la medida que se
	realiza son eventos por minuto, las señales son capturadas, amplificadas y registradas en un contador que se reinicia cada minuto. A lo largo
	de este trabajo muchas veces nos referiremos a esta medición de eventos por minuto con el nombre de \emph{cuentas}. 

\section{NMDB}
	NMDB\cite{NMDB2011} o \emph{Neutron Monitor Database} es una red mundial de monitores de neutrones. Antes de proceder a hablar sobre la red en
	concreto expondremos las ventajas y objetivos de una red como esta.
	\begin{itemize}
		\item 	Espectro de energías. Al igual que el Sol, la Tierra tiene campo magnético. Este campo magnético repele con mayor fuerza en
		  	las regiones ecuatoriales que en los polos. Esto implica que solo las partículas más energéticas son perceptibles en las
			zonas ecuatoriales, mientras que en los polos las partículas no necesitan ser tan energéticas para alcanzar la superficie.
			Combinando datos de estaciones que se encuentran a diferentes latitudes podemos construir espectrogramas basados en la energía
			de las partículas.
		\item 	Anisotropía. Tener estaciones en diferentes lugares del globo terráqueo implica estar orientado en diferentes direcciones del
		  	espacio. Esto implica poder realizar estudios sobre la procedencia de eventos.
		\item 	Redundancia. Tener muchas estaciones implica detectar el mismo evento en más de una estación. Esto permite comparar los datos
		  	entre estaciones y descartar fluctuaciones grandes, rápidas y asiladas en una sola estación.
		\item 	Cooperación. Estar en una red implica mejorar la comunicación entre las diferentes estaciones. De esta manera los resultados
		  	son mejores y el avance más rápido. 
	\end{itemize}
	\par
	Como ya hemos comentado NMDB es una red mundial, impulsada por la Comisión Europea. Actualmente la red supera las veinte estaciones y
	proporciona datos en tiempo real con resolución de un minuto. Los formatos de los datos están estandarizados entre las diferentes estaciones,
	esto ayuda al análisis científico de estos. Los datos en tiempo real son utilizados para la elaboración de un sistema de alarma
	GLE\cite{GleAlarm}. Como hemos explicado un GLE fuerte puede tener un gran impacto negativo en nuestras vidas. Es beneficioso poder detectar
	estos eventos lo antes posible y este es uno de los objetivos principales del NMDB. 

\section{CALMA}
	CALMA\cite{Medina2013} o \emph{Castilla la Mancha Neutron Monitor} es el primer y único monitor de neutrones en España. Este forma parte del
	NMDB, el equipo técnico responsable de la estación está profundamente implicado en desarrollar sistemas y herramientas que mejoran la red. Un
	ejemplo es el sistema de adquisición implantado en la estación, también implantado en otras estaciones de la red. La estación empezó a operar
	de forma plena en diciembre de 2012 y desde entonces lleva haciéndolo ininterrumpidamente con pequeñas excepciones. Desde su puesta en marcha
	la estación ha registrado 18 Forbush decreases. Desafortunadamente aún no ha habido ningún GLE que detectar, aunque este tendría que ser muy
	energético para ser detectado en una estación con tan poca latitud. A continuación procedemos a hablar más a fondo del estado actual de CALMA,
	hablaremos del sistema de adquisición, base de datos, herramientas y técnicas que son utilizadas.
	\subsection{Sistema de adquisición}
		Como hemos mencionado el sistema de adquisición que está implantado en CALMA es producto del propio equipo\cite{Garcia2014}. Este está
		basado en un sistema empotrado Linux. Si las señales generadas por los tubos contadores superan un determinado umbral, estas son
		covertidas en pulsos digitales por los amplificadores. A continuación, un circuito de adaptación se encarga de elevar los niveles de
		tensión de los pulsos digitales a niveles \emph{TTL} y de adaptar la impedancia de la línea de comunicaciones si el amplificador
		así lo requiere. Seguidamente los pulsos son procesados por una FPGA que es la encargada de medir los eventos por minuto de los 18
		canales. Aunque la estación tan solo tiene 15 tubos contadores el sistema está diseñado para soportar 18, este número es un estándar
		histórico. El software que se ejecuta en el sistema Linux tiene como tarea comunicarse con la FPGA, su labor es recoger las
		\emph{cuentas} de cada minuto y guardar estas en una base de datos. El software hace uso del \emph{Network Time Protocol} para
		asegurar la correcta sincronización temporal entre estaciones, esta sincronización permite contrastar los datos de diferentes
		estaciones. En la figura \ref{fig:acqsis} podemos ver un diagrama de bloques que representa el sistema de adquisición.
		\begin{figure}[h]
			\centering
			\includegraphics[keepaspectratio, width=1\textwidth]{./img/AcqSis.png}
			\caption{Sistema de adquisición.}
			\label{fig:acqsis}
		\end{figure}
		\par
		Actualmente el equipo de CALMA está desarrollando un nuevo sistema de adquisición de datos. El esquema presentado en la figura
		\ref{fig:acqsis} es aplicable al nuevo sistema. Las diferencias entre los dos sistemas de adquisición radican en los componentes. La
		primera de estas está en los amplificadores. Los utilizados en el sistema actual generan pulsos fijos, mientras que los amplificadores
		utilizados en este nuevo sistema de adquisición generan pulsos cuyo ancho es proporcional a la energia de la señal original. El fin es
		medir la energía de las señales generadas por los tubos contadores, esta magnitud es de gran interés técnico.
		\par
		La segunda diferencia está en el circuito de adaptación. Este cambia a fin de soportar los nuevos amplificadores que acabamos de
		explicar. La compatibilidad con los amplificadores utilizados en el sistema de adquisición actual también se mantiene. Esto da mucha
		flexibilidad a la estación, permitiéndole incorporar diferentes amplificadores a la vez.
		\par
		La siguiente diferencia está en la FPGA, pero sobre todo en el \emph{IP core} integrado en cada uno de los dos sistemas. Recordamos
		que en el sistema de adquisición actual este tan solo registra los eventos por minuto para los 18 canales. El nuevo \emph{IP core}
		debe realizar la misma tarea, pero aparte debe poder medir el ancho de los pulsos generados por los amplificadores. Los dos sistemas
		de adquisición utilizan un puerto serie para la comunicación entre software y FPGA. En el nuevo sistema este puerto opera a una
		velocidad mayor a fin de poder transmitir toda la información extra.
		\par
		El nuevo sistema también está pensado para ofrecer una mayor compatibilidad. El fin es poder implantar este en diferentes estaciones
		de forma fácil y rápida. El hardware y software deben ser diseñados con este requisito en mente.
	\subsection{Bases de datos}
		La base de datos que genera CALMA está diseñada de acuerdo con el estándar impuesto por NMDB. Los eventos por minuto de los 18 canales
		se guardan en la tabla \texttt{binTable}. En una segunda tabla, \texttt{CALM\_ori}, se guarda el valor global de la estación, calculado
		a partir de los datos de todos los tubos. Junto a este valor global también se guardan las correcciones que se realizan sobre este. El
		valor global y las correcciones se explican en la última sección de este capítulo. Los datos de esta segunda tabla son revisados
		por un operador humano y finalmente guardados en una tercera tabla cuyo nombre es \texttt{CALM\_rev}. Los datos de estas dos últimas
		tablas se envian al NMDB. Este envío es realizado por un proceso que se ejecuta cada minuto. El envío consiste en calcular las
		diferencias entre los dos conjuntos de datos y tan solo enviar estas.
	\subsection{Herramientas y técnicas}
		El equipo de CALMA hace uso de herramientas que les ayudan a analizar la información desde el punto de vista científico. Actualmente
		no existe ninguna herramienta que proporcione información sobre el estado técnico de la estación, todas las labores relacionadas con
		el mantenimiento son realizadas de forma manual.

\section{Objetivos}
	\textbf{El objetivo de este proyecto es desarrollar el software para el nuevo sistema de adquisición de datos y también desarrollar una
	herramienta que permite operar y monitorizar el estado de la estación}. A continuación procedemos a hacer una descripción más detallada de los
	objetivos de este trabajo.  
	\subsection{Software de adquisición}
		Tal y como hemos explicado en secciones anteriores el equipo de CALMA está desarrollando un nuevo sistema de adquisición. Actualmente
		la mayoría de módulos hardware incluyendo la FPGA están listos. El propósito de este trabajo es realizar el software de adquisición. A
		continuación exponemos algunos de los requisitos más relevantes.
		\begin{itemize}
			\item 	El software debe ser capaz de realizar una correcta comunicación con la FPGA. Esto implica enviar los comandos
				apropiados y ser capaz de interpretar los mensajes de datos trasmitidos por esta. En el capítulo \ref{entornoHW} se
				puede obtener más información sobre la interfaz de comunicación con la FPGA.
			\item 	El software debe ser capaz de calcular el valor global de la estación y las correcciones que se realizan sobre este.
			  	Estos valores son explicados en la última sección de este capítulo.
			\item	El software debe ser capaz de almacenar en local tanto los eventos por minuto de los 18 canales como el dato global y
			  	las correcciones de este.
			\item 	El software también debe ser capaz de mantener una réplica remota de los datos.
			\item 	Al igual que el hardware, el software debe ser capaz de adaptarse fácilmente a diferentes estaciones. Para este
				propósito este debe ser diseñado de tal forma que sea fácil de extender.
			\item 	El software debe ser capaz de poner en marcha la estación completa de forma automática ante la presencia de corriente
			  	eléctrica. 
			\item 	El software debe ser capaz de detectar estados anómalos y actuar conforme a estos. En una estación de este tipo la
				mayoría de veces un funcionamiento anómalo se traduce en no generar datos o generar datos irregulares. Ante la
				presencia de un estado anómalo debe generarse una alerta. El software también debe reaccionar ante dichos estados
				anómalos, la mayoría de problemas se solucionan realizando un reinicio del sistema.
		\end{itemize}
		Aparte de la realización del software en este trabajo también contemplamos el proceso de implantación y mantenimiento de este.
		Volvemos a recordar que los demás módulos que componen el sistema de adquisición ya están desarrollados por el equipo de CALMA. Las
		interfaces de estos son descritas, dado que es necesario para entender este trabajo.
	\subsection{Herramienta Web}
		El segundo objetivo de este trabajo es el desarrollo de una herramienta que facilite la gestión de una estación. La idea de esta
		es del equipo de CALMA. Procedemos a detallar las funcionalidades de la herramienta tal y como las concibe este.
		%TODO Citar el artículo de la herramienta.
		\begin{itemize}
			\item	\textbf{Spike Tool}. Módulo que permitirá la detección de Spikes. Un Spike es un dato anómalo, un dato
				anormalmente grande o pequeño. Usando los datos proporcionados por el sistema de adquisición, este módulo generará
				gráficos. Éstos serán interactivos y su propósito será hacer más fácil la detección de Spikes. Los Spikes detectados
				podrán ser marcados como nulos en el conjunto de datos revisados. 
			\item 	\textbf{Configuración de los parámetros la estación}. Este módulo permitirá la reconfiguración de los parámetros de la
				estación. Esta reconfiguración será llevada a cabo sin interrumpir el proceso de adquisición. Nótese que el software
				de adquisición deberá evolucionar para hacer posible esta funcionalidad.
			\item	\textbf{Alertas}. La herramienta visualizará las alertas producidas por el software de adquisición.
			\item 	\textbf{Histogramas con la intensidad de los eventos}. El nuevo sistema de adquisición proporciona información sobre
				la energía de las partículas detectadas. Este módulo generará histogramas con estos datos. Estos histogramas
				permitirán hacer mejores diagnósticos sobre el funcionamiento de los tubos contadores. 
		\end{itemize}
		Podemos ver que la herramienta ofrece un gran abanico de funcionalidades. Por desgracia en este trabajo tan solo nos centraremos en el
		primer módulo, realizar los demás módulos está fuera del alcance de un trabajo como este. También cabe destacar que tan solo nos
		centraremos en la implementación, no implantaremos ni mantendremos la herramienta. La herramienta será una herramienta Web intuitiva y
		altamente interactiva.

\section{Diseño preliminar}
	En esta sección procedemos a especificar un diseño preliminar para el software de adquisición y para la herramienta Web. La primera subsección
	es dedicada al software de adquisición, mientras que la segunda es reservada para la herramienta Web.
	\subsection{Software de adquisición}
		El sistema de adquisición que se está desarrollando es un sistema empotrado donde hardware y software están muy vinculados. Esto en
		gran medida condiciona el diseño del software que queremos realizar. El software debe ejecutarse sobre una \emph{BeagleBone
		Black}\cite{Beagle} que está integrada con el resto de componentes hardware. La distribución Linux elegida para este trabajo es
		\emph{Angstrom}, la distribución por defecto. El lenguaje de programación elegido es \emph{Python}\cite{Python}, lenguaje interpretado de alto nivel
		con tipado dinámico y sintaxis centrada en producir código legible. Actualmente \emph{Python} es muy popular y existen muchas bibliotecas de
		las que haremos uso. El uso de bibliotecas reduce la carga de trabajo y normalmente resulta en software más robusto. Para la gestión
		de la base de datos hemos elegido \emph{Sqlite3}\cite{Sqlite}, una elección popular en sistemas empotrados. Este es un sistema ligero, de
		alto rendimineto que implementa la base de datos en un único archivo. Esto junto al API bien definido simplifica la administración.
		\par
		En la figura \ref{fig:soft_control_preliminar} podemos ver el diseño preliminar del software de adquisición. Ante la presencia de
		corriente eléctrica este debe ser capaz de inicializarse automáticamente. El primer paso que debe hacer es realizar la configuración
		necesaria para el correcto funcionamiento del sistema de adquisición. Seguidamente debe continuar con el funcionamiento nominal. Este
		consiste de tres pasos que se repiten cíclicamente. El primero es solicitar la información a la FPGA, seguidamente el software debe
		interpretar dicha información y finalmente la información debe ser guardada. Los posibles estados anómalos deben ser detectados y
		contrarrestados. 
		\begin{figure}[h]
			\centering
			\includegraphics[keepaspectratio, width=1\textwidth]{./img/soft_control_preliminar.png}
			\caption{Software de adquisición. Diseño preliminar.}
			\label{fig:soft_control_preliminar}
		\end{figure}
	\subsection{Herramienta Web}
		En la figura \ref{fig:herramienta_web_preliminar} podemos ver el diseño preliminar de la herramienta Web. Como podemos observar el
		diseño está fuertemente basado en el modelo \emph{Cliente-Servidor}\cite{MVCWiki}. En este trabajo utilizamos los términos
		\emph{back-end} y \emph{front-end} para referirnos al \emph{Cliente} y al \emph{Servidor} respectivamente. A continuación explicamos los componentes
		básicos.
		\begin{description}
			\item[Base de Datos.]    
				En este componente residen los datos de nuestra estación. Para la gestión de estos utilizamos un servidor
				\emph{MySQL}\cite{MySql}.
			\item[\emph{Back-End}.]
				Este componente es el encargado de recibir y procesar los mensajes de petición provenientes del \emph{Front-End}.
				Estas peticiones pueden ser de consulta o de acción. En ambos casos este componente procede a comunicarse con la base
				de datos a fin de satisfacer la petición. Finalmente el resultado es trasmitido al \emph{Front-End} en un mensaje de
				respuesta. En el caso de una petición de consulta son devueltos los datos especificados. En el caso de una petición de
				acción es devuelto un mensaje de estado. Para la implementación de este componente hemos utilizado
				\emph{ZendFramework}\cite{ZF} y \emph{Apygility}\cite{Apigility}.
			\item[\emph{Front-end}.] 
				Este componente implementa la interfaz de nuestra aplicación. Es el encargado de presentar la información y manejar
				las peticiones del usuario. El módulo está basado en el patrón \emph{Modelo-Vista-Controlador}. Para implementar la
				\emph{Vista} encargada de presentar los datos hemos utilizado \emph{HighStock}\cite{HighStock}, para el
				\emph{Controlador} responsable de gestionar los eventos empleamos \emph{ExtJs}\cite{ExtJs} y para el modelo encargado
				de manejar los datos utilizamos peticiones \emph{Ajax}\cite{AjaxWiki}.
		\end{description}
		\begin{figure}[h]
			\centering
			\includegraphics[keepaspectratio, width=1\textwidth]{./img/herramienta_web_preliminar.png}
			\caption{Herramienta Web. Diseño preliminar}
			\label{fig:herramienta_web_preliminar}
		\end{figure}
\section{Proceso de adquisición}
	El propósito de este punto es explicar algunos aspectos del proceso de adquisición que son relevantes para este trabajo.
	\par
	Hasta este momento siempre hemos hablado de tubos contadores, en plural, detrás de esto hay una razón. Normalmente las estaciones se componen
	de varios tubos contadores, donde 18 tubos contadores es un estándar. En el proceso de adquisición están envueltos muchos factores
	probabilísticos, esto conlleva a que las medidas en un tubo tengan una gran dispersión. La solución de este problema es tener muchos tubos,
	cuantos más mejor. Combinando datos de diferentes tubos conseguimos reducir esta dispersión. 
	\par
	Tener los datos de muchos tubos contadores permite reducir la dispersión de los datos, sin embargo no es muy práctico trabajar con esos datos
	en crudo. El software del sistema de adquisición actual calcula un valor global a partir de los datos de todos los tubos contadores. El
	software de adquisición que desarrollamos en este trabajo también debe calcular este valor global.  Para este propósito utilizamos el Median
	Algorithm\cite{MedianAlgr}.
	\par
	Anteriormente en este capítulo explicamos las cascadas atmosféricas que son originadas por los rayos cósmicos. Estas cascadas atmosféricas
	dependen de la presión atmosférica. Cuando esta es elevada, es necesaria más energía para que la cascada se propague hasta el nivel terrestre,
	por consecuente los monitores de neutrones registran menos eventos. El valor de la presión atmosférica es monitorizado por los monitores de
	neutrones. Este valor es utilizado para realizar una corrección por presión sobre el valor global de la estación, a lo largo de este trabajo
	utilizamos el término de \emph{valor corregido por presión} para referirnos a este valor. El software de adquisición elaborado para este
	trabajo debe ser capaz de leer el valor de presión desde un barómetro y debe poder calcular esta corrección. 
	\par
	También es realizada una \emph{corrección por eficiencia}. Esta es muy simple y consiste en aplicar un factor multiplicativo al \emph{valor
	corregido por presión}. Este valor es utilizado para para solventar problemas técnicos. Cambios en el entorno inmediato del instrumento o
	cambios en la electrónica utilizada pueden afectan a la cantidad de eventos medidos. Estos cambios son identificados, evaluados y finalmente
	contrarrestados con esta corrección por eficiencia. Este valor dota la estación de consistencia histórica, de esta manera pueden ser
	comparados datos de diferentes intervalos temporales. El software también debe ser capaz de realizar esta corrección. 
	\par
	Al principio de este capítulo explicamos que los tubos contadores están rellenos de gas con propiedades especiales. Sobre este gas es aplicado
	un campo eléctrico. Para la generación de dicho campo son utilizadas fuentes de alimentación de alta tensión. Cambios en la tensión generada
	pueden afectar a la cantidad de eventos registrados. Esto ha conllevado a que el funcionamiento de las fuentes sea monitorizado. El software
	para el sistema de adquisición debe ser capaz de realizar esta operación. Es de esperar que la tensión sea constante, en caso de variaciones
	los datos generados son considerados no consistentes.
	\par
	Un \emph{Spike} es un dato anómalo, un dato anormalmente grande o pequeño. Son datos producidos por mal funcionamientos de la instrumentación,
	cambios bruscos en el entorno inmediato u otros factores desconocidos. Estos están presentes en todas las estaciones. Con la elaboración del
	nuevo sistema de adquisición se pretende reducir el número de \emph{Spikes} generados. Con la elaboración de la herramienta Web pretendemos
	ofrecer un método fácil de identificar y descartar dichos datos.

\chapter{Entorno Hardware}
\label{entornoHW}

En un sistema empotrado el software y hardware están muy vinculados. Para entender el funcionamiento del software de adquisición es muy importante
estar familiarizado con el diseño y funcionalidad del hardware. Impulsados por esta razón en este capítulo procederemos a hacer una descripción de los
aspectos más importantes del hardware que compone el sistema de adquisición. Volvemos a enfatizar que el autor de este trabajo no ha formado parte en
la realización de los módulos hardware que serán descritos a continuación.
\section{BeagleBone Black}
	BeagleBone Black\cite{Beagle}\cite{BeagleWiki} es un computador empotrado, open-source y single board. La placa viene con Linux, distribución
	Angstrom y versión de núcleo 3.8. En la figura \ref{fig:beaglebone} podemos ver un diagrama de bloques que refleja los componentes hardware
	que componen la BeagleBone. A continuación vamos a hacer una breve descripción de los módulos que utilizamos para este trabajo.
	\begin{itemize}
		\item 	ARM CORTEX A8\cite{BeagleCore} y SDRAM. El procesador es suficientemente potente para soportar la distribución Linux y
			satisfacer las necesidades de nuestro software. Los 512MB DDR3 de memoria RAM son suficientes para los propósitos de este
			trabajo.
		\item 	Ethernet Connector. El conector RJ-45 permite establecer una conexión a Internet. Una vez establecida esta conexión los datos
			pueden ser transmitidos al NMDB. También es posible establecer una conexión a la placa vía SSH, de esta manera un operador
			puede realizar operaciones de mantenimiento de forma remota.
		\item	eMMC y microSD. Utilizamos la memoria integrada (eMMC) de 4GB para albergar el sistema operativo y el software de
			adquisición. La microSD es utilizada para guardar los datos y logs producidos por el software de adquisición.
		\item 	Analog Pins. Los 7 pines analógicos permiten usar sensores cuyo output es una señal analógica. Ejemplo de estos sensores son
			las fuentes de alimentación analógicas o los sensores de temperatura, en algunas estaciones la temperatura ambiente se
			monitoriza también. Estos sensores producen señales cuya tensión eléctrica es proporcional al valor de la magnitud medida. 
		\item 	GPIOs. Los \emph{General Purpose Input/Output} permiten trabajar con señales digitales, tanto de entrada como de salida. Un
			ejemplo de uso es la señal de Reset de la FPGA, señal digital activa a bajo nivel.
		\item	UARTs. La placa ofrece 4 puertos serie de los que utilizamos 2 para comunicarnos con la FPGA.
	\end{itemize}
	\begin{figure}[h]
		\centering
		\includegraphics[keepaspectratio, width=1\textwidth]{./img/beaglebone.png}
		\caption{BeagleBone Black. Diagrama de bloques hardware}
		\label{fig:beaglebone}
	\end{figure}
	\par
	Fijándonos en la figura \ref{fig:beaglebone} podemos ver que muchos de los módulos son accesibles mediante los conectores de
	expansión\cite{BeagleWikiExp}. Estos conectores tienen 2x46 pines, de los cuales muchos son multipropósito. Estos pueden ser configurados para tener
	funcionalidades diversas y además esta configuración puede ser realizada de forma dinámica. La configuración de los pines se realiza mediante
	el \emph{Device Tree Overlay}, estructura de datos que se utiliza para describir el hardware. En la realización de este trabajo no lidiamos
	con este mecanismo, sino que utilizamos una librería que facilita el trabajo. La librería utilizada es \emph{Adafruit BeagleBone IO
	Python}\cite{AdaFruitGit}. Esta ayuda a configurar dinámicamente los pines de expansión, además ofrece métodos para
	operar sobre estos una vez configurados.

\section{FPGA}
	La FPGA es el núcleo fundamental del sistema, ya que, se encarga de procesar las señales y transmitir la información al software de
	adquisición. Una FPGA es un circuito integrado, que está formado por unidades lógicas cuya interconexión y funcionalidad pueden ser
	configurados. Esta configuración está definida por el \emph{IP core} integrado, por lo tanto, es este el que establece la funcionalidad de una
	FPGA. En esta sección vamos a hacer una descripción funcional de la FPGA, por consiguiente será explicado el \emph{IP core} de esta.  En la
	figura \ref{fig:fpga} podemos ver un diagrama de bloques que refleja los módulos funcionales y interfaces que el este posee.
	\begin{figure}[h]
		\centering
		\includegraphics[keepaspectratio, width=1\textwidth]{./img/fpga.png}
		\caption{FPGA. Diagrama de bloques.}
		\label{fig:fpga}
	\end{figure}	
	\subsection{Procesado de pulsos}
		Como hemos comentado la FPGA es la encargada de procesar las señales y trasmitir la información a la BeagleBone Black. Esta transmisión
		se realiza mediante una línea serie asíncrona de 1Mbps. La línea está controlada por una UART(Universal Asynchronous
		Receiver-Transmitter) a la que nos referiremos como \emph{UART de pulsos}. En las tablas \ref{tab:FPGAUartPulso},
		\ref{tab:FPGAUartOver} y \ref{tab:FPGAUartCont} podemos ver el formato de los datos transmitidos. Vemos que hay tres mensajes
		diferentes que la FPGA puede transmitir.
		\begin{itemize}
			\item	En la tabla \ref{tab:FPGAUartPulso} podemos ver el formato del primer mensaje que vamos a discutir. Este mensaje se
				compone de tres bytes que dan información sobre el ancho de un pulso. El mensaje permite identificar el canal en el
				que se ha producido el pulso, el nivel (alto/bajo) y la longitud de este. Los pulsos de nivel alto representan la
				detección de una partícula por los tubos contadores y el ancho del pulso es proporcional a la energía de la partícula
				detectada. La longitud de los pulsos de nivel bajo también se mide para poder identificar pulsos con pequeña
				separación temporal, que podrían indicar la presencia de un fenómeno físico denominado multiplicidad.
			\item 	En la siguiente tabla \ref{tab:FPGAUartOver} podemos apreciar el mensaje de estado que la FPGA genera. Dado que la UART usa
				una FIFO de tamaño fijo es posible que la esta se llene y se pierdan mensajes. En este caso la FPGA genera un mensaje
				de estado que es transmitido para indicarle al software que se han producido perdidas de datos.
			\item	Además de medir el ancho de los pulsos la FPGA lleva la cuenta de pulsos recibidos para cada canal. Esta información
				puede ser solicitada por la BeagleBone Black utilizando el comando apropiado(ver tabla \ref{tab:FPGAUartComm}). En la
				tabla \ref{tab:FPGAUartCont} podemos ver el formato en el que se transmite la información de las cuentas. Los bytes 2,
				3 y 4 son transmitidos 18 veces, una vez para cada canal. Después de transmitir la información de las cuentas la FPGA
				reinicia los contadores para cada canal.
		\end{itemize}
		En la figura \ref{fig:fpgaWave} podemos observar un ejemplo de la secuencia de datos transmitidos por la \emph{UART de pulsos}, junto a
		estos podemos apreciar los estímulos que los causan. 

		\begin{table}[h]
			\tiny
			\begin{tabularx}{\textwidth}{|l|c|c|X|c|c|c|c|c|}
				\hline
				\rowcolor[HTML]{C0C0C0} 
				\multicolumn{1}{|r|}{\textbf{Bit}}    	& 7 & 6          & 5 				& 4 	       & 3 	     & 2 	  & 1          & 0 	     \\ \hline
				\cellcolor[HTML]{C0C0C0}\textbf{Byte 1} & 0 & 1          & 0  				& \multicolumn{5}{c|}{Canal (0-17)}				     \\ \hline
				\cellcolor[HTML]{C0C0C0}\textbf{Byte 2} & 1 & Dato (6)	 & Dato (5)      		& Dato (4)     & Dato (3)    & Dato (2)   & Dato (1)   & Dato (0)    \\ \hline
				\cellcolor[HTML]{C0C0C0}\textbf{Byte 3} & 1 & X          & Nivel ('1'->alto, '0'->bajo) & Dato (11)    & Dato (10)   & Dato (9)   & Dato (8)   & Dato (7)    \\ \hline
			\end{tabularx}
			\caption{\emph{UART de pulsos}. Palabra de ancho de pulso}
			\label{tab:FPGAUartPulso}
			\begin{tabularx}{\textwidth}{|l|c|X|X|X|X|X|X|X|}
				\hline
				\rowcolor[HTML]{C0C0C0} 
				\multicolumn{1}{|r|}{\textbf{Bit}} 	& 7 & 6 		       & 5 		       & 4		       & 3 		       & 2		       & 1          	       & 0			\\ \hline
				\cellcolor[HTML]{C0C0C0}\textbf{Byte 1} & 0 & 0                        & OverFlow FIFO Tubo 5  & OverFlow FIFO Tubo 4  & OverFlow FIFO Tubo 3  & OverFlow FIFO Tubo 2  & OverFlow FIFO Tubo 1  & OverFlow FIFO Tubo 0	\\ \hline
				\cellcolor[HTML]{C0C0C0}\textbf{Byte 2} & 1 & OverFlow FIFO Tubo 12    & OverFlow FIFO Tubo 11 & OverFlow FIFO Tubo 10 & OverFlow FIFO Tubo 9  & OverFlow FIFO Tubo 8  & OverFlow FIFO Tubo 7  & OverFlow FIFO Tubo 6	\\ \hline
				\cellcolor[HTML]{C0C0C0}\textbf{Byte 3} & 1 & Almost Full FIFO General & OverFlow FIFO General & OverFlow FIFO Tubo 17 & OverFlow FIFO Tubo 16 & OverFlow FIFO Tubo 15 & OverFlow FIFO Tubo 14 & OverFlow FIFO Tubo 13	\\ \hline
			\end{tabularx}
			\caption{\emph{UART de pulsos}. Palabra de estado}
			\label{tab:FPGAUartOver}
			\begin{tabularx}{\textwidth}{|l|X|c|c|c|c|c|c|c|}
				\hline
				\rowcolor[HTML]{C0C0C0} 
				\multicolumn{1}{|r|}{\textbf{Bit}}    	 & 7 & 6           & 5 		& 4 	      & 3 	    & 2 	 & 1           & 0 	     	\\ \hline
				\cellcolor[HTML]{C0C0C0}\textbf{Byte 1}  & 0 & 1           & 1  	& X	      & X	    & X	  	 & X	       & X	     	\\ \hline
				\cellcolor[HTML]{C0C0C0}\textbf{Byte 2}  & 1 & Dato0 (6)   & Dato0 (5) 	& Dato0 (4)   & Dato0 (3)   & Dato0 (2)  & Dato0 (1)   & Dato0 (0)  	\\ \hline
				\cellcolor[HTML]{C0C0C0}\textbf{Byte 3}  & 1 & Dato0 (13)  & Dato0 (12)	& Dato0 (11)  & Dato0 (10)  & Dato0 (9)  & Dato0 (8)   & Dato0 (7)  	\\ \hline
				\cellcolor[HTML]{C0C0C0}\textbf{Byte 4}  & 1 & X	   & X	 	& X	      & X	    & X		 & Dato0 (15)  & Dato0 (14)	\\ \hline
				\cellcolor[HTML]{C0C0C0}\textbf{Byte 5}  & 1 & Dato1 (6)   & Dato1 (5) 	& Dato1 (4)   & Dato1 (3)   & Dato1 (2)  & Dato1 (1)   & Dato1 (0)  	\\ \hline
				\cellcolor[HTML]{C0C0C0}\textbf{......}  & . & .........   & ......... 	& .........   & .........   & .........  & .........   & .........  	\\ \hline
				\cellcolor[HTML]{C0C0C0}\textbf{Byte 55} & 1 & X	   & X	 	& X	      & X	    & X		 & Dato17 (15) & Dato17 (14)	\\ \hline
			\end{tabularx}
			\caption{\emph{UART de pulsos}. Palabra de cuentas}
			\label{tab:FPGAUartCont}
			\begin{tabularx}{\hsize}{|c|X|}
		  		\hline
				\rowcolor[HTML]{C0C0C0} 
		  		Commando & Descripción                            \\\hline
		  		0x00     & Configura multiplexor para aparato 1   \\\hline
		  		0x01     & Configura multiplexor para aparato 2   \\\hline
		  		0x02     & Configura multiplexor para aparato 3   \\\hline
		  		0x03     & Configura multiplexor para aparato 4   \\\hline
		  		0x10     & Reset general del sistema              \\\hline
		  		0x11     & Solicita la transmisión de las cuentas \\\hline
			\end{tabularx}
			\caption{\emph{UART de pulsos}. Commandos}
			\label{tab:FPGAUartComm}
		\end{table}
		\begin{figure}[h]
			\centering
			\includegraphics[keepaspectratio, width=1\textwidth]{./img/fpgawave.png}
			\caption{FPGA. Diagrama de eventos.}
			\label{fig:fpgaWave}
		\end{figure}
	\subsection{Multiplexor}
		Tal y como explicamos en el capítulo de introducción, el sistema de adquisición debe monitorizar la presión atmosférica, el
		funcionamiento de las fuentes de tensión y en muchos casos la temperatura ambiente. Al igual que muchos otros dispositivos que pueden
		ser necesarios, la mayoría de estos sensores hacen uso de un puerto serie. Estos son manejados según el patron \emph{Master-Slave} y
		generalmente intercambian poca información, por consiguiente, pueden compartir un puerto serie. El mecanismo que resuelve este
		problema está implmentado en el \emph{IP core}. Este consiste en una UART a la que nos referiremos como \emph{UART de extensión}. Esta
		está conectada a un multiplexor, que es el encargado de realizar la conmutación entre los cuatro dispositivos soportados. El estado
		del multiplexor puede cambiarse enviando comandos por la \emph{UART de pulsos}, de acuerdo a la especificación presentada en la tabla
		\ref{tab:FPGAUartComm}. Ejemplo de dispositivos que requiren un puerto serie son el barómetro \emph{BM35} utilizado en CALMA, varias
		fuentes de alimentación comerciales, estaciones meteorológicas, GPS, etc.
	\subsection{Reset}
		El \emph{IP core} permite realizar un Reset del estado interno de la FPGA. Todas las variables son puestas a sus valores iniciales,
		excepto el multiplexor que mantiene su estado. Para la realización de dicho Reset son exportadas tres interfaces. La primera es un
		boton hardware accesible de forma física. La segunda es un commando trasmitido por la \emph{UART de pulsos}, de acuerdo a la
		especificación presentada en la tabla \ref{tab:FPGAUartComm}. Finalmente, la tercera es una señal digital activa a bajo nivel. Dicha
		señal definida como \emph{BS1}, está conectada al pin \emph{P9\_42(GPIO\_7)} de la BeagleBone Black, por consiguiente, es accesible
		desde nuestro software de adquisición.

\chapter{Software de adquisición}
\label{cap2}

Texto explicación introductiva.

\section{Entorno Hardware}
	En un sistema empotrado el software y hardware están muy vinculados. Para entender el funcionamiento del software es muy importante estar
	familiarizado con el diseño y funcionalidad del hardware. Impulsados por esta razón a continuación procederemos a hacer una descripción de
	los aspectos más importantes del hardware que compone nuestro sistema de adquisición. Volvemos a enfatizar que el autor de este trabajo no
	ha formado parte  en la realización de los módulos hardware que serán descritos a continuación.
	\subsection{BeagleBone Black}
		La BeagleBone Black es un computador empotrado, open-source y single board. La placa viene con Linux, distribución Angstrom y versión
		de núcleo 3.8.  Lo más característico de la placa son los 2x46 pines de extensión disponibles. Estos pines son los utilizados para
		integrar este componente con el resto del sistema de adquisición. La mayoría de estos pines son multipropósito, pueden ser
		configurados para tener funcionalidades diversas y además esta configuración puede ser realizada de forma dinámica.
		\par
		La configuración de los pines se realiza mediante el “Device Tree Overlay”, estructura de datos que se utiliza para describir el
		hardware. Para la realización de este trabajo no tendremos que lidiar con este mecanismo, utilizaremos una librería que nos facilite
		el trabajo. La librería utilizada es “Adafruit BeagleBone IO Python”. Esta librería nos ayudara a configurar dinámicamente los pines
		de extensión, además la librería ofrece métodos para operar sobre estos una vez configurados. En la figura 1 podemos ver las
		funcionalidades que los pines de la BeagleBone Black ofrecen y las que vamos a utilizar para este trabajo.
		\begin{table}[h]
	  		\begin{tabular}{|l|l|}
	    			\hline
	    			\rowcolor[HTML]{C0C0C0} 
	    			{\color[HTML]{000000} \textbf{Disponibles}} & {\color[HTML]{000000} \textbf{Usados}} \\ \hline
	    				7 Analog Pins                               & 4 Analog Pins                          \\ \hline
	    				65 Digital Pins at 3.3V                     & 1 Digital Pin                          \\ \hline
	    				2x I2C                                      & 0x I2C                                 \\ \hline
	    				2x SPI                                      & 0x SPI                                 \\ \hline
	    				4 Timers                                    & 0 Timers                               \\ \hline
	    				4x UART                                     & 2x UART                                \\ \hline
	    				8x PWM                                      & 0x PWM                                 \\ \hline
	    				A/D Converter                               & Not used                               \\ \hline
	  		\end{tabular}
			\centering
			\caption{BeagleBone Black Headers.}
			\label{fig:BBBPins}
		\end{table}
	\subsection{FPGA}
		Intro FPGA. La FPGA es el componente que más interactuar con la BeagleBone Black.
		Intro, Reset, UARTS 

\section{Configuración del sistema}
	System Services. Como funcionan. Como las usamos.
	\subsection{WatchDog}
	\subsection{Time Sync}
	\subsection{Software de adquisición}

\section{Sensor Manager}
	\subsection{Pressure}
	\subsection{HVPS}

\section{Reader}
	Explicar funcionamiento.

\section{Counts Pettioner}
	Intro funcionamineto.
	\subsection{Deriva local}
	\subsection{Petición de cuentas}
	\subsection{Petición de datos sensores}
	\subsection{Guardar datos}

\section{DbUpdater}

\section{testing}
	Porque son importantes los unit tests.
	\subsection{moking}
	\subsection{Coverage}

\section{Mantenimiento}
	Dejar el sistema de adquisición de datos funciónando algun tiempo. Hacer analisis de los datos. Describir experiencia.  TODO dejar para el capítuo de Conclusiones.

%\chapter{Ámbito y objetivos}
\label{cap2} 

\section{¿Por qué este Trabajo?}

Lorem ipsum dolor sit amet, consectetur adipiscing elit. Quisque et ipsum ac
sem venenatis fermentum. Mauris tincidunt ante nibh, faucibus pellentesque elit
mattis sed. Phasellus sollicitudin faucibus nulla, ullamcorper tempus nisl
hendrerit at. Suspendisse aliquam quam non nisi hendrerit, vitae mattis eros
semper. Aenean sollicitudin dapibus enim nec dictum. Sed porta diam sit amet
sem eleifend aliquet. In dapibus lacinia mauris, sed vulputate purus.
Pellentesque efficitur imperdiet efficitur. Mauris venenatis, arcu ac
vestibulum efficitur, massa augue fringilla sem, eu aliquam nisl metus nec
magna.

Pellentesque rhoncus mi lorem. Donec odio diam, dignissim non ex a,
pellentesque lacinia libero. Aenean ornare orci sit amet nulla sodales, non
finibus tellus sollicitudin. Cum sociis natoque penatibus et magnis dis
parturient montes, nascetur ridiculus mus. Sed auctor at nulla sit amet rutrum.
Duis tempor metus id magna tincidunt lacinia. Maecenas placerat velit et luctus
auctor. Nulla elit erat, condimentum in lacinia eget, volutpat non orci. Donec
et lacinia ipsum, non tempor massa. Nam ligula ante, euismod non sapien eget,
tincidunt aliquet sem. Morbi auctor elit nec mi efficitur, eget pellentesque
leo faucibus.

Nunc dictum, lacus sodales auctor iaculis, lectus turpis pulvinar dolor, in
gravida nibh enim sit amet elit. Vestibulum sit amet volutpat sapien. Curabitur
nec ultricies massa. Aliquam erat volutpat. Donec maximus volutpat maximus. In
eu elementum ante. Pellentesque neque libero, pellentesque quis sagittis ac,
tempor vel mauris. Praesent et arcu nisi. Maecenas non iaculis mauris.

Sed at nunc suscipit, congue ante quis, accumsan urna. Donec eget venenatis
est. Fusce nec eros nec tellus tempor rutrum ut nec urna. Pellentesque quis
pulvinar lacus. Nullam ac mollis turpis. Aliquam eu porta odio, sed commodo
ante. Maecenas luctus orci ut odio bibendum tincidunt. Nam lacus nisi,
condimentum nec condimentum vitae, euismod sed purus. Quisque erat sem, mattis
a tortor et, imperdiet tempor nibh. Phasellus maximus neque imperdiet velit
cursus aliquet. Maecenas dictum id leo et pharetra.

Proin feugiat felis diam, sit amet accumsan est finibus nec. Nulla pellentesque
accumsan tempor. Phasellus convallis volutpat eros, ac accumsan metus eleifend
vehicula. Fusce imperdiet, leo in tincidunt dignissim, lacus libero semper
mauris, eu maximus mi nisl quis enim. Aliquam hendrerit lacinia justo ut
gravida. Nulla nec mattis ligula. Curabitur varius pharetra tellus at auctor.
Integer a urna nibh. Vivamus consectetur lorem et nisl egestas maximus at vel
sem.

\section{Objetivos}

Proin feugiat felis diam, sit amet accumsan est finibus nec. Nulla pellentesque
accumsan tempor. Phasellus convallis volutpat eros, ac accumsan metus eleifend
vehicula. Fusce imperdiet, leo in tincidunt dignissim, lacus libero semper
mauris, eu maximus mi nisl quis enim. Aliquam hendrerit lacinia justo ut
gravida. Nulla nec mattis ligula. Curabitur varius pharetra tellus at auctor.
Integer a urna nibh. Vivamus consectetur lorem et nisl egestas maximus at vel
sem.

\begin{enumerate}
    \item El sistema a implementar será multiplataforma,
    \item Será un sistema dividido en módulos, para su fácil modificación y ampliación posterior,
    \item Interfaz amigable y de fácil manejo,
    \item Será una aplicación multiusuario,
    \item No será necesario realizar ningún tipo de instalación,
    \item Toda la información será almacenada en una base de datos,
    \item El sistema tendrá gestión de usuarios y limitación de privilegios,
    \item Debe ser posible georreferenciar los elementos patrimoniales a inventariar,
    \item La aplicación generará informes en PDF.
\end{enumerate}

\section{Descripción general del trabajo}

Nunc dictum, lacus sodales auctor iaculis, lectus turpis pulvinar dolor, in
gravida nibh enim sit amet elit. Vestibulum sit amet volutpat sapien. Curabitur
nec ultricies massa. Aliquam erat volutpat. Donec maximus volutpat maximus. In
eu elementum ante. Pellentesque neque libero, pellentesque quis sagittis ac,
tempor vel mauris. Praesent et arcu nisi. Maecenas non iaculis mauris.
 
Nunc dictum, lacus sodales auctor iaculis, lectus turpis pulvinar dolor, in
gravida nibh enim sit amet elit. Vestibulum sit amet volutpat sapien. Curabitur
nec ultricies massa. Aliquam erat volutpat. Donec maximus volutpat maximus. In
eu elementum ante. Pellentesque neque libero, pellentesque quis sagittis ac,
tempor vel mauris. Praesent et arcu nisi. Maecenas non iaculis mauris.

\section{Cómo se ha estructurado este libro}

Lorem ipsum dolor sit amet, consectetur adipiscing elit. Quisque et ipsum ac
sem venenatis fermentum. Mauris tincidunt ante nibh, faucibus pellentesque elit
mattis sed. Phasellus sollicitudin faucibus nulla, ullamcorper tempus nisl
hendrerit at. Suspendisse aliquam quam non nisi hendrerit, vitae mattis eros
semper. Aenean sollicitudin dapibus enim nec dictum. Sed porta diam sit amet
sem eleifend aliquet. In dapibus lacinia mauris, sed vulputate purus.
Pellentesque efficitur imperdiet efficitur. Mauris venenatis, arcu ac
vestibulum efficitur, massa augue fringilla sem, eu aliquam nisl metus nec
magna.

Pellentesque rhoncus mi lorem. Donec odio diam, dignissim non ex a,
pellentesque lacinia libero. Aenean ornare orci sit amet nulla sodales, non
finibus tellus sollicitudin. Cum sociis natoque penatibus et magnis dis
parturient montes, nascetur ridiculus mus. Sed auctor at nulla sit amet rutrum.
Duis tempor metus id magna tincidunt lacinia. Maecenas placerat velit et luctus
auctor. Nulla elit erat, condimentum in lacinia eget, volutpat non orci. Donec
et lacinia ipsum, non tempor massa. Nam ligula ante, euismod non sapien eget,
tincidunt aliquet sem. Morbi auctor elit nec mi efficitur, eget pellentesque
leo faucibus.

Nunc dictum, lacus sodales auctor iaculis, lectus turpis pulvinar dolor, in
gravida nibh enim sit amet elit. Vestibulum sit amet volutpat sapien. Curabitur
nec ultricies massa. Aliquam erat volutpat. Donec maximus volutpat maximus. In
eu elementum ante. Pellentesque neque libero, pellentesque quis sagittis ac,
tempor vel mauris. Praesent et arcu nisi. Maecenas non iaculis mauris.

Sed at nunc suscipit, congue ante quis, accumsan urna. Donec eget venenatis
est. Fusce nec eros nec tellus tempor rutrum ut nec urna. Pellentesque quis
pulvinar lacus. Nullam ac mollis turpis. Aliquam eu porta odio, sed commodo
ante. Maecenas luctus orci ut odio bibendum tincidunt. Nam lacus nisi,
condimentum nec condimentum vitae, euismod sed purus. Quisque erat sem, mattis
a tortor et, imperdiet tempor nibh. Phasellus maximus neque imperdiet velit
cursus aliquet. Maecenas dictum id leo et pharetra.

Proin feugiat felis diam, sit amet accumsan est finibus nec. Nulla pellentesque
accumsan tempor. Phasellus convallis volutpat eros, ac accumsan metus eleifend
vehicula. Fusce imperdiet, leo in tincidunt dignissim, lacus libero semper
mauris, eu maximus mi nisl quis enim. Aliquam hendrerit lacinia justo ut
gravida. Nulla nec mattis ligula. Curabitur varius pharetra tellus at auctor.
Integer a urna nibh. Vivamus consectetur lorem et nisl egestas maximus at vel
sem.

%\chapter{Análisis de requisitos}
\label{cap3}

\section{Sobre la fase de análisis}

El análisis del problema generalemente describe el ``qué hay que hacer'',
mientras que el diseño dice ``cómo se va a hacer''. De esta forma pueden
existir muchos diseños diferentes solucionar un mismo problema descrito a
través del análisis del mismo.

En el análisis que se llevará a cabo se formalizarán los requisitos generales
del sistema, en los que se habla de los aspectos que éste debe cubrir y de las
distintas funcionalidades que debe poseer para llevar a cabo todas las
operaciones necesarias para cada entidad o usuario del sistema. Lo no
formalizado en estos requisitos se considera fuera del alcance del proyecto.

\section{Requisitos de usuario}

Nunc dictum, lacus sodales auctor iaculis, lectus turpis pulvinar dolor, in
gravida nibh enim sit amet elit. Vestibulum sit amet volutpat sapien. Curabitur
nec ultricies massa. Aliquam erat volutpat. Donec maximus volutpat maximus. In
eu elementum ante. Pellentesque neque libero, pellentesque quis sagittis ac,
tempor vel mauris. Praesent et arcu nisi. Maecenas non iaculis mauris.

\section{Casos de uso}

\section{Actores}

\section{Relaciones}

\section{Casos de uso de la aplicación}

A continuación se detallan los posibles casos de uso de la aplicación:

\subsection{Crear un elemento desde la aplicación}
\begin{description}
\item [Precondiciones:]No existen.
\item [Actores:]Podrá acceder cualquiera de los posibles actores antes
    definidos para la aplicación.
\item [Descripción:] Se encontrará un sistema de autenticación que deberá
    superar para acceder a la aplicación. Tras esto, deberá elegir el tipo de
    elemento que desea añadir (siempre que tenga los privilegios adecuados) y
    entrará a la ventana principal. Tras rellenar un formulario con los datos
    relacionados del elemento en cuestión, deberá guardarlo, de forma que
    quedará almacenado en base de datos.
\end{description}

\subsection{Modificar un elemento desde la aplicación}

Nunc dictum, lacus sodales auctor iaculis, lectus turpis pulvinar dolor, in
gravida nibh enim sit amet elit. Vestibulum sit amet volutpat sapien. Curabitur
nec ultricies massa. Aliquam erat volutpat. Donec maximus volutpat maximus. In
eu elementum ante. Pellentesque neque libero, pellentesque quis sagittis ac,
tempor vel mauris. Praesent et arcu nisi. Maecenas non iaculis mauris.


\section{Requisitos no funcionales}

Éstos hacen referencia a las características que deberá tener el sistema que no
hacen referencia a su funcionalidad, sino a sus características. Es decir, no
hacen referencia a la manera en que se guardan los datos en el sistema ni las
funciones que realiza la aplicación.

Los requisitos no funcionales principales serán:

\begin{enumerate}
    \item Accesibilidad a través de internet. Uno de los requisitos no
        funcionales de este sistema es la posibilidad que se nos brindará de
        acceder desde cualquier sitio a la aplicación sin necesidad de realizar
        instalaciones en un ordenador.
    \item Almacenamiento centralizado. La información se encontrará guardada en
        un servidor, así se tendrá todo bien organizado y a la vista de todo el
        mundo.
    \item Escalabilidad. En un futuro y si así se desea, la aplicación podrá
        ser ampliada para posibles mejoras.
    \item Eficiencia. Se buscará que la aplicación web sea lo más ligera
        posible, evitando así tiempos de espera y de transacciones para el
        usuario.
    \item Precio. El gasto inicial será bastante bajo.
    \item Fácil de mantener. Se buscará sencillez en el mantenimiento de la
        aplicación.
    \item Portabilidad. La aplicación deberá ser portable a otras plataformas
        sin problemas.
\end{enumerate}

%\chapter{Diseño del sistema}
\label{cap4}

\section{Arquitectura de Software}

La manera en la que se estructure nuestra aplicación será fundamental a la hora
del diseño, ya que se ha elegido una plataforma Cliente-Servidor, explicada más
adelante. La arquitectura hace referencia a la manera en la que está diseñada
la aplicación. En la arquitectura se pueden especificar detalles físicos
(hardware) y lógicos (software). Lo ideal es que la arquitectura esté
modularizada y las diferentes capas sean independientes, ya que se buscará al
máximo la portabilidad, la escalabilidad y la flexibilidad.

\subsection{Modelo Cliente-Servidor}

Es aquel utilizado para describir una aplicación en la que hay al menos dos
procesos separados trabajando con el fin de completar una misma tarea. El
cliente solicita al servidor la ejecución de una operación particular, llevando
a cabo un proceso cooperativo.

Nunc dictum, lacus sodales auctor iaculis, lectus turpis pulvinar dolor, in
gravida nibh enim sit amet elit. Vestibulum sit amet volutpat sapien. Curabitur
nec ultricies massa. Aliquam erat volutpat. Donec maximus volutpat maximus. In
eu elementum ante. Pellentesque neque libero, pellentesque quis sagittis ac,
tempor vel mauris. Praesent et arcu nisi. Maecenas non iaculis mauris.

\subsection{Patrones de diseño software}

Es la base de la que parte la búsqueda de soluciones para problemas comunes de
desarrollo software u otros ámbitos relacionados con el diseño y desarrollo
software.

Partimos de un problema de diseño. A partir de él, diseñamos un patrón que
posea una serie de características que proporcionen una solución estándar a
raíz de un problema dado.

Para que un diseño pueda ser denominado patrón, tendrá que tener como
característica fundamental la reusabilidad, ya que deberá ser aplicable a
diferentes problemas de diseño en distintas situaciones.

\subsection{Patrón Modelo-Vista-Controlador}

Nunc dictum, lacus sodales auctor iaculis, lectus turpis pulvinar dolor, in
gravida nibh enim sit amet elit. Vestibulum sit amet volutpat sapien. Curabitur
nec ultricies massa. Aliquam erat volutpat. Donec maximus volutpat maximus. In
eu elementum ante. Pellentesque neque libero, pellentesque quis sagittis ac,
tempor vel mauris. Praesent et arcu nisi. Maecenas non iaculis mauris.

\section{Arquitectura general de la aplicación}

Nunc dictum, lacus sodales auctor iaculis, lectus turpis pulvinar dolor, in
gravida nibh enim sit amet elit. Vestibulum sit amet volutpat sapien. Curabitur
nec ultricies massa. Aliquam erat volutpat. Donec maximus volutpat maximus. In
eu elementum ante. Pellentesque neque libero, pellentesque quis sagittis ac,
tempor vel mauris. Praesent et arcu nisi. Maecenas non iaculis mauris.

\section{Diseño del software del cliente}

Nunc dictum, lacus sodales auctor iaculis, lectus turpis pulvinar dolor, in
gravida nibh enim sit amet elit. Vestibulum sit amet volutpat sapien. Curabitur
nec ultricies massa. Aliquam erat volutpat. Donec maximus volutpat maximus. In
eu elementum ante. Pellentesque neque libero, pellentesque quis sagittis ac,
tempor vel mauris. Praesent et arcu nisi. Maecenas non iaculis mauris.

\subsection{Componentes utilizados, jerarquía e interconexión}

Nunc dictum, lacus sodales auctor iaculis, lectus turpis pulvinar dolor, in
gravida nibh enim sit amet elit. Vestibulum sit amet volutpat sapien. Curabitur
nec ultricies massa. Aliquam erat volutpat. Donec maximus volutpat maximus. In
eu elementum ante. Pellentesque neque libero, pellentesque quis sagittis ac,
tempor vel mauris. Praesent et arcu nisi. Maecenas non iaculis mauris.

\section{Diseño del software del servidor}

Nunc dictum, lacus sodales auctor iaculis, lectus turpis pulvinar dolor, in
gravida nibh enim sit amet elit. Vestibulum sit amet volutpat sapien. Curabitur
nec ultricies massa. Aliquam erat volutpat. Donec maximus volutpat maximus. In
eu elementum ante. Pellentesque neque libero, pellentesque quis sagittis ac,
tempor vel mauris. Praesent et arcu nisi. Maecenas non iaculis mauris.

\section{Arquitectura del sistema informático}

Nunc dictum, lacus sodales auctor iaculis, lectus turpis pulvinar dolor, in
gravida nibh enim sit amet elit. Vestibulum sit amet volutpat sapien. Curabitur
nec ultricies massa. Aliquam erat volutpat. Donec maximus volutpat maximus. In
eu elementum ante. Pellentesque neque libero, pellentesque quis sagittis ac,
tempor vel mauris. Praesent et arcu nisi. Maecenas non iaculis mauris.

\subsection{Servidor WEB (Apache)}

Nunc dictum, lacus sodales auctor iaculis, lectus turpis pulvinar dolor, in
gravida nibh enim sit amet elit. Vestibulum sit amet volutpat sapien. Curabitur
nec ultricies massa. Aliquam erat volutpat. Donec maximus volutpat maximus. In
eu elementum ante. Pellentesque neque libero, pellentesque quis sagittis ac,
tempor vel mauris. Praesent et arcu nisi. Maecenas non iaculis mauris.

\subsection{Intérprete de PHP (mod-php)}

Nunc dictum, lacus sodales auctor iaculis, lectus turpis pulvinar dolor, in
gravida nibh enim sit amet elit. Vestibulum sit amet volutpat sapien. Curabitur
nec ultricies massa. Aliquam erat volutpat. Donec maximus volutpat maximus. In
eu elementum ante. Pellentesque neque libero, pellentesque quis sagittis ac,
tempor vel mauris. Praesent et arcu nisi. Maecenas non iaculis mauris.

\begin{lstlisting}
function db_open($server,$user,$pass){
	$link=mysql_connect($server,$user,$pass);
    if(!$link){
    	$error=mysql_error();
        print ("<H1>No esta autorizado para acceder al servidor: $server</H1><BR>");
        print ("<P>Error: $error<P>");
        exit;
    }
    return $link;
}
\end{lstlisting}


%\chapter{Implementación}
\label{cap5}

\section{Descripción general de la implementación}

En esta sección se describen las herramientas y procedimientos que se han
utilizado para la construcción del diseño realizado previamente. Se describirán
aquí tanto herramientas hardware tales como emuladores, entrenadores, placas de
evaluación etc., como herramientas software como por ejemplo editor,
compiladores, bibliotecas, sistemas operativos, etc. También aquí se debe de
especificar si se ha utilizado alguna metodología por ejemplo agile, o técnicas
de control de versiones como git, subversión, etc.


\section{La tarjeta de evaluación XM233.}

Nunc dictum, lacus sodales auctor iaculis, lectus turpis pulvinar dolor, in
gravida nibh enim sit amet elit. Vestibulum sit amet volutpat sapien. Curabitur
nec ultricies massa. Aliquam erat volutpat. Donec maximus volutpat maximus. In
eu elementum ante. Pellentesque neque libero, pellentesque quis sagittis ac,
tempor vel mauris. Praesent et arcu nisi. Maecenas non iaculis mauris.

\section{El entorno de Eclipse V3.3}

Nunc dictum, lacus sodales auctor iaculis, lectus turpis pulvinar dolor, in
gravida nibh enim sit amet elit. Vestibulum sit amet volutpat sapien. Curabitur
nec ultricies massa. Aliquam erat volutpat. Donec maximus volutpat maximus. In
eu elementum ante. Pellentesque neque libero, pellentesque quis sagittis ac,
tempor vel mauris. Praesent et arcu nisi. Maecenas non iaculis mauris.

\section{Sistema de control de versiones del software.}

Nunc dictum, lacus sodales auctor iaculis, lectus turpis pulvinar dolor, in
gravida nibh enim sit amet elit. Vestibulum sit amet volutpat sapien. Curabitur
nec ultricies massa. Aliquam erat volutpat. Donec maximus volutpat maximus. In
eu elementum ante. Pellentesque neque libero, pellentesque quis sagittis ac,
tempor vel mauris. Praesent et arcu nisi. Maecenas non iaculis mauris.

\section{Distribución Debian del sistema operativo Linux.}

Nunc dictum, lacus sodales auctor iaculis, lectus turpis pulvinar dolor, in
gravida nibh enim sit amet elit. Vestibulum sit amet volutpat sapien. Curabitur
nec ultricies massa. Aliquam erat volutpat. Donec maximus volutpat maximus. In
eu elementum ante. Pellentesque neque libero, pellentesque quis sagittis ac,
tempor vel mauris. Praesent et arcu nisi. Maecenas non iaculis mauris.



%\chapter{Manuales}\label{cap6}
\section{Manual de usuario}

A lo largo de este capítulo presentamos todas las posibles acciones que puede
realizar el usuario para acceder a la información de las fichas patrimoniales,
cómo realizar búsquedas detalladas, cómo interactuar con el módulo GIS,
asociación de fichas patrimoniales y cómo crear nuevos patrimonios relacionados
con la Vía de la Plata. 

De esta manera, se intentará ilustrar de la manera más sencilla posible el
manejo de la aplicación y de cada una de sus principales peculiaridades.

\subsection{Estructura de las ventanas}

La aplicación posee una estructura muy similar en cada una de sus secciones
principales. Todas las ventanas de la aplicación web constan de un panel
principal en el que se muestran todos los campos disponibles para ser
visualizados, modificados o creados y en la parte superior, una barra de
herramientas con el fin de facilitar al operador el acceso a la información.

Todas las secciones de acceso a fichas patrimoniales poseen similar estructura.
Para cada una de las opciones se contempla la opción de añadir una nueva ficha
patrimonial del tipo que corresponda o modificar una ya existente. Uno de los
menús desplegados tiene este aspecto:

\subsection{Creación de un elemento.}

Nunc dictum, lacus sodales auctor iaculis, lectus turpis pulvinar dolor, in
gravida nibh enim sit amet elit. Vestibulum sit amet volutpat sapien. Curabitur
nec ultricies massa. Aliquam erat volutpat. Donec maximus volutpat maximus. In
eu elementum ante. Pellentesque neque libero, pellentesque quis sagittis ac,
tempor vel mauris. Praesent et arcu nisi. Maecenas non iaculis mauris.

\subsection{Obtención de un histograma de respuesta.}

Nunc dictum, lacus sodales auctor iaculis, lectus turpis pulvinar dolor, in
gravida nibh enim sit amet elit. Vestibulum sit amet volutpat sapien. Curabitur
nec ultricies massa. Aliquam erat volutpat. Donec maximus volutpat maximus. In
eu elementum ante. Pellentesque neque libero, pellentesque quis sagittis ac,
tempor vel mauris. Praesent et arcu nisi. Maecenas non iaculis mauris.

\section{Manual de Instalación}

Hacer una correcta instalación del sistema es fundamental para un correcto
funcionamiento, por eso se detalla de aquí en adelante la instalación de
\emph{WampServer}, que es un paquete de herramientas que le facilitará al
administrador del sistema la instalación y configuración del servidor en caso
de que tenga que hacer una nueva instalación.

Nunc dictum, lacus sodales auctor iaculis, lectus turpis pulvinar dolor, in
gravida nibh enim sit amet elit. Vestibulum sit amet volutpat sapien. Curabitur
nec ultricies massa. Aliquam erat volutpat. Donec maximus volutpat maximus. In
eu elementum ante. Pellentesque neque libero, pellentesque quis sagittis ac,
tempor vel mauris. Praesent et arcu nisi. Maecenas non iaculis mauris.


\chapter{Conclusiones y trabajos futuros}
\label{capx}

\section{Conclusiones}
\section{Trabajos futuros}
	\subsection{Debian}
	\subsection{Script configuración}
	\subsection{WatchDog pro version}



%Incluyo la bibliografía
\addcontentsline{toc}{chapter}{Referencias bibliográficas}
%\let\antiguaLM\leftmark
%\let\antiguaRM\rightmark
%\renewcommand\leftmark{}
\nocite{*}
\bibliographystyle{plain}
\bibliography{Bibliografia}

%Acronyms
\addcontentsline{toc}{chapter}{Acrónimos}
\printglossary[type=\acronymtype]
\newpage


%Incluyo un glosario
%\addcontentsline{toc}{chapter}{Glosario}
%\let\antiguaLM\leftmark
%\let\antiguaRM\rightmark
%\renewcommand\leftmark{}
%\renewcommand\rightmark{Glosario}
%\include{glosario}

\end{document}
