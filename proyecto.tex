\documentclass[twoside,spanish,a4paper,openright,11pt]{book}
\usepackage[spanish]{babel}
\usepackage[utf8]{inputenc}
\usepackage{graphics}
\usepackage{graphicx}
\usepackage{url}
\usepackage[T1]{fontenc}
\usepackage{fancyhdr}
\usepackage{hyperref}
\usepackage{color}
\usepackage{colortbl}
\usepackage{listings}
\usepackage{colourlist}
\usepackage{booktabs}
\usepackage{amssymb}
\usepackage[acronym, nonumberlist]{glossaries}
\usepackage[table,xcdraw]{xcolor}
\usepackage{rotating}
\usepackage{tabularx}
\usepackage[justification=centering]{caption}
\usepackage{textcomp}
\usepackage[stretch=90,shrink=10]{microtype}
\usepackage{wrapfig}
\usepackage{enumitem}



\newcommand{\cc}{{\textquotesingle}}

\makeglossaries

%Configuración del paquete hyperref
% colores definidos
\definecolor{mygray}{rgb}{.86,.86,.86}
\definecolor{myblue}{rgb}{.20,.20,.6}
\definecolor{gray97}{gray}{.97}
\definecolor{gray75}{gray}{.75}
\definecolor{gray45}{gray}{.45}
\definecolor{whitesmoke}{rgb}{0.96,0.96,0.96}

% configuración del paquete hyperref
\hypersetup{   pdfauthor = {Hristo Ivanov Ivanov}
               pdftitle = {Software empotrado de control y gestión de un monitor de neutrones},
               pdfkeywords = {NMDB, CALMA, TFC},
               pdfsubject = {Proyecto Final de Carrera},
               colorlinks={true},
               pdfstartview={FitV},
               linkcolor={myblue},
               citecolor={myblue},
               urlcolor={myblue}
}





%Nos permite definir los márgenes como queramos del documento completo
\usepackage{anysize}
\marginsize{3.5cm}{3cm}{2cm}{2cm}

%Con esta definición pordemos cambiar los márgenes sobre la marcha utilizando
% \begin{changemargin}{7cm}{0cm}
%	.. .. ..
% \end{changemargin}
\def\changemargin#1#2{\list{}{\rightmargin#2\leftmargin#1}\item[]}
\let\endchangemargin=\endlist





% Control de lneas viudas y huérfanas
\clubpenalty=100000
\widowpenalty=10000
\displaywidowpenalty=1000
\looseness=1

%-----------------------------------------------------------------------------%
%
% Formato de las cabeceras
%
\pagestyle{fancyplain}

\lhead[
        \fancyplain{}
	    {\textrm{\textbf{\thepage}\hspace{4mm}\small\leftmark}}]
        {\fancyplain{}
        {\textrm{\footnotesize\ GSO}}}
\rhead[
        \fancyplain{}
        {\textrm{\footnotesize\ Hristo Ivanov Ivanov}}]
        {\fancyplain{}
        {\textrm{\small\rightmark\hspace{4mm}\normalsize\textbf{\thepage}}}}
\cfoot[]{} % En el pie de página no ponemos nada
\renewcommand{\headrulewidth}{0.1mm}

\graphicspath{{img/}}



%Definiciónes para la representación de consola, ficheros y código
\definecolor{pythonComment}{rgb}{.5,.2,.0}
\definecolor{pythonKey}{rgb}{0,0,.5}
\definecolor{pythonKey2}{rgb}{0,.5,.2}
\definecolor{pythonString}{rgb}{.5,0,0}
\usepackage{listings}
\lstdefinestyle{myPython}{
	language	= Python,
	frame 		= lines,
	basicstyle	= \ttfamily,
	keywordstyle	= \color{pythonKey},
	otherkeywords	= {self, True, False},
	keywords	= [2]{True, False},
	keywordstyle	= [2]\color{pythonKey2},
	commentstyle	= \em \color{pythonComment},
	stringstyle	= \color{pythonString},
%
	numbers=left,
	numbersep=15pt,
	numberstyle=\tiny,
	numberfirstline = false,
	breaklines=true
}

\lstdefinestyle{myBash}{
	language	= bash,
	frame 		= lines,
	basicstyle	= \ttfamily,
	keywordstyle	= \color{pythonKey},
	commentstyle	= \em \color{pythonComment},
	stringstyle	= \color{pythonString},
%
	numbers=left,
	numbersep=15pt,
	numberstyle=\tiny,
	numberfirstline = false,
	breaklines=true
}



 
% minimizar fragmentado de listados
\lstnewenvironment{listing}[1][]
   {\lstset{#1}\pagebreak[0]}{\pagebreak[0]}
 
\lstdefinestyle{consola}
   {basicstyle=\scriptsize\bf\ttfamily,
    backgroundcolor=\color{gray75},
   }

\lstdefinestyle{Fichero}
   {basicstyle=\scriptsize\bf\ttfamily,
    backgroundcolor=\color{whitesmoke},
   }
 
\lstdefinestyle{C}
   {language=C
   }
\lstdefinestyle{Java}
   {language=Java,
moredelim={**[il][\rlap{\textcolor{azure2}{\rule[-4pt]{\linewidth}{\baselineskip}}}]{--}}
%Esta última línea se pone para que cuando se añade delante de una línea de código se resalta 
%poniéndola en el color indicado
   }


\begin{document}

\newacronym{cme}{CME}{Eyección de masa coronal}

\newacronym{nmdb}{NMDB}{Neutron Monitor Database}

\newacronym{gle}{GLE}{Ground level enhancements}

\newacronym{calma}{CALMA}{Monitor de neutrones de Castilla-La Mancha}

\newacronym{fpga}{FPGA}{Field Programmable Gate Array}

\newacronym{mvc}{MVC}{Modelo–vista–controlador}

\newacronym{gpio}{GPIO}{General Purpose Input/Output}

\newacronym{uart}{UART}{Universal Asynchronous Receiver-Transmitter}

\newacronym{api}{API}{Application programming interfaces}

\newacronym{rest}{REST}{Representational state transfer}

\newacronym{url}{URL}{Localizador de recursos uniforme}

\newacronym{rpc}{RPC}{Remote Procedure Call}

\glsaddall


\enlargethispage{1cm}
\thispagestyle{empty}

\begin{center}
\includegraphics[width=2cm,keepaspectratio]{../img/logoUAH.jpg}
\par\end{center}

\begin{center}
\textbf{\huge Universidad de Alcalá}
\par\end{center}{\huge \par}

\begin{center}
{\Large Escuela Politécnica Superior}
\par\end{center}{\Large \par}

\medskip{}


\begin{center}
\textbf{\textsc{\huge Grado en Ingeniería de Computadores }}
\par\end{center}{\huge \par}

\medskip{}


\begin{center}
\textbf{\Large Trabajo de Fin de Grado}
\par\end{center}{\Large \par}

\begin{center}
\textbf{\textsc{\LARGE 
  Software empotrado de control y gestión de un monitor de neutrones
}}
\par\end{center}{\LARGE \par}

\medskip{}


\begin{center}
\textbf{\large Autor:}{\large{} Hristo Ivanov Ivanov}
\par\end{center}{\large \par}
\begin{center}
\textbf{\large Tutor:}{\large{} Óscar García Población }
\par\end{center}{\large \par}

\vspace{1.0cm}
\textbf{\Large TRIBUNAL:}{\Large \par}

\vspace{2.0cm}
\textbf{\Large Presidente:}{\Large{} \line(1,0){287}  }{\Large \par} %%TODO

\vspace{2.0cm}
\textbf{\Large Vocal 1:}{\Large{} \line(1,0){307}  }{\Large \par} %%TODO

\vspace{2.0cm}
\textbf{\Large Vocal 2:}{\Large{} \line(1,0){307}  }{\Large \par} %%TODO

\vspace{0.8cm}
\textbf{ Fecha: \line(1,0){75} \hspace{1cm}Calificación: \line(1,0){150}  }{ \par}




\newpage{\pagestyle{empty}\cleardoublepage}


\pagenumbering{arabic}
\setcounter{page}{3}

\setcounter{tocdepth}{3}

\thispagestyle{empty}
\vspace*{4cm}
\begin{flushright}           %%TODO
\textit{Dedicatorias varias \\,   
varias varias varias varias varias varias varias varias varias varias.}
\end{flushright}




\cleardoublepage
\chapter*{Agradecimientos}  %%TODO


\vspace{10mm}
\textit{Muchos agradecimientos a StackOverflow.}



\tableofcontents
\newpage
\listoffigures % Índice de figuras
\newpage
\renewcommand{\listtablename}{Índice de tablas}
\listoftables % Índice de tablas
\newpage

\chapter{Introducción}
\label{cap1}

El propósito de este documento es describir el \emph{Trabajo de Fin de Grado} realizado por el autor. Este es parte del desarrollo de un nuevo sistema de
adquisición de datos para un monitor de neutrones, instrumento que permite medir los niveles de radiación cósmica. Los objetivos del trabajo son la
realización del software encargado de gestionar la adquisición de datos y el desarrollo de una herramienta Web dirigida a la gestión del sistema.	
\par
Empezamos este capítulo haciendo una breve introducción a los conceptos teóricos y el entorno en el que se desarrolla este trabajo. Continuamos
estableciendo los objetivos y el diseño preliminar. Finalmente aclaramos algunas peculiaridades referentes al proceso de adquisición de datos. El
segundo capítulo describe el entorno hardware en el que se desarrolla el software de adquisición. Los siguientes capítulos describen en detalle el
trabajo realizado. El primero de estos es dedicado al software de adquisición. Debido a la separación impuesta por el modelo \emph{Cliente-Servidor}
se utilizarán dos capítulos para describir la herramienta Web, el primero dedicado al \emph{Servidor} y el segundo al \emph{Cliente}. El último
capítulo de este documento recoge las conclusiones obtenidas del trabajo y la visión futura de este. El documento también incluye un apéndice, en este
son descritos los pasos a seguir para implantar el software de adquisición y la herramienta Web. En este apéndice también son reflejados los aspectos
más importantes del proceso de implementación. Finalmente el documento incluye una lista de referencias bibliográficas a las fuentes de información
utilizadas en el trabajo. 

\section{Introducción a los rayos cósmicos}
	A principios del siglo \emph{XX} se descubrió la existencia de los rayos cósmicos. Estos son partículas subatómicas provenientes del espacio
	exterior. Estas partículas, en su mayoría protones y núcleos de Helio, son muy energéticas debido a su gran velocidad. El origen de estas no
	está muy claro, pero sabemos que proceden del espacio exterior. Raras veces la actividad solar puede producir partículas tan energéticas. 
	\par
	Muchas de estas partículas inciden en la atmósfera terrestre. En las capas altas de la atmósfera se producen las primeras interacciones, estas
	partículas colisionan con las partículas que forman la atmósfera. Esta colisión es muy violenta y causa la división de las partículas
	originales en partículas secundarias. Estas a su vez pueden colisionar con otras partículas de la atmósfera para así formar aún más partículas
	secundarias. Vemos como una sola partícula proveniente del espacio produce el fenómeno denominado \emph{cascadas atmosféricas}. Como es de
	esperar con cada choque consecutivo se pierde parte de la energía. Normalmente las partículas secundarias que alcanzan la superficie terrestre
	tan solo tienen una pequeña fracción de la energía inicial. Si una partícula no posee la energía suficiente la cascada que es originada no se
	propaga hasta la superficie terrestre.
	\par
	Como hemos dicho la mayor parte de la radiación cósmica proviene de fuera de nuestro sistema solar, pero está fuertemente relacionada con los
	ciclos solares. Los ciclos solares de 11 años aproximadamente afectan la actividad solar pasando por un mínimo y un máximo, donde los cambios
	son apreciables en la luminosidad y el campo magnético. Es este segundo, el campo magnético solar, el que afecta a la llegada de radiación
	cósmica a la Tierra. Al ser mayormente partículas con carga eléctrica en la presencia de un fuerte campo magnético estas son desviadas. A
	continuación detallamos los sucesos más comunes que pueden ser observados indirectamente a consecuencia de estudiar la cantidad de radiación
	cósmica.
	\begin{itemize}
		\item	Ciclo solar. Como hemos explicado existe una fuerte relación entre la cantidad de radiación cósmica y la actividad solar. La
			radiación cósmica es un buen indicador de la actividad solar donde la relación es inversa. Menos radiación generalmente
			significa una actividad solar elevada.
		\item	\emph{Forbush decrease}\cite{Forbush1938}. Estos sucesos consisten en un descenso rápido de los niveles de radiación cósmica medida
			en la Tierra. Estos descensos son consecuencia de CME's(Eyección de masa coronal). La materia expulsada en un CME al ser en su
			mayoría plasma, extiende e intensifica el campo magnético solar. Como ya hemos explicado el aumento del campo magnético solar
			conlleva al descenso de radiación cósmica.
		\item	\emph{Ground level enhancements}(GLE). Eventualmente la actividad solar es tan elevada que el Sol es capaz de emitir partículas muy
			energéticas. Estas partículas son a veces tan energéticas que pueden generar cascadas atmosféricas que alcanzan la superficie
			terrestre. Estos sucesos son muy raros, entre 10 y 15 por década. A pesar de ser muy raros estos pueden tener un gran impacto
			en nuestras vidas cotidianas, pueden afectar el funcionamiento de la electrónica sensible que está en orbita e incluso la que
			está en tierra.   
	\end{itemize}

\section{Monitor de neutrones}
	Un monitor de neutrones es una estación terrestre que monitoriza la llegada de partículas extraterrestres de forma indirecta a partir de las
	cascadas atmosféricas. Este está compuesto por cuatro capas especialmente diseñadas para capturar las partículas secundarias producidas en las
	cascadas atmosféricas. A continuación procedemos a explicar estas cuatro capas.
	\begin{itemize}
		\item	\textbf{Reflector}. La primera capa, la exterior, consiste en un escudo reflector que tan solo deja pasar las partículas con
			energía alta.  De esta manera todas las partículas generadas por el entorno inmediato que tienen baja energía rebotan y no
			influyen en la medición.
		\item	\textbf{Productor}. Esta capa compuesta generalmente de material denso tiene como objetivo conseguir algo parecido a las
			cascadas atmosféricas. La idea es tener un material denso para que aumente la probabilidad de que las partículas secundarias
			impacten con las partículas del material y como resultado se produzcan aún más partículas. A las partículas generadas en esta
			capa se les da el nombre de neutrones de evaporación. Son estas partículas las que finalmente serán medidas por el
			instrumento, también son las que le dan nombre. Los neutrones producidos tienen menos energía, por lo que son más fáciles de
			medir.
		\item	\textbf{Moderador}. A pesar de que las partículas que tenemos a este nivel tienen tan solo una fracción de la energía
			original, estas aún siguen siendo demasiado energéticas para ser capturadas. Esta capa tiene como objetivo ralentizar,
			disminuir la energía, de las partículas para así poder capturarlas.
		\item	\textbf{Contador}. Un contador proporcional o tubo contador generalmente está relleno de gas con propiedades específicas.
			Cuando el gas interacciona con los neutrones de evaporación este es ionizado, en el proceso son liberados electrones. Debido
			al campo eléctrico que atraviesa constantemente el gas estos electrones son acelerados hacia el ánodo, hilo conductor que
			atraviesa el tubo contador en su centro. Conforme los electrones ganan energía, estos pueden producir ionizaciones
			secundarias. El número total de electrones que llegan al ánodo se mantiene, sin embargo, proporcional a la energía de la
			partícula inicial. Al llegar al ánodo estos electrones producen una corriente eléctrica. 

	\end{itemize}
	\par
	Los sistemas de adquisición están diseñados para recoger estas pequeñas corrientes eléctricas y medirlas. Tradicionalmente la medida que se
	realiza son eventos por minuto, las señales son capturadas, amplificadas y registradas en un contador que se reinicia cada minuto. A lo largo
	de este trabajo muchas veces nos referiremos a esta medición de eventos por minuto con el nombre de \emph{cuentas}. 

\section{NMDB}
	NMDB\cite{NMDB2011} o \emph{Neutron Monitor Database} es una red mundial de monitores de neutrones. Antes de proceder a hablar sobre la red en
	concreto expondremos las ventajas y objetivos de una red como esta.
	\begin{itemize}
		\item 	Espectro de energías. Al igual que el Sol, la Tierra tiene campo magnético. Este campo magnético repele con mayor fuerza en
		  	las regiones ecuatoriales que en los polos. Esto implica que solo las partículas más energéticas son perceptibles en las
			zonas ecuatoriales, mientras que en los polos las partículas no necesitan ser tan energéticas para alcanzar la superficie.
			Combinando datos de estaciones que se encuentran a diferentes latitudes podemos construir espectrogramas basados en la energía
			de las partículas.
		\item 	Anisotropía. Tener estaciones en diferentes lugares del globo terráqueo implica estar orientado en diferentes direcciones del
		  	espacio. Esto implica poder realizar estudios sobre la procedencia de eventos.
		\item 	Redundancia. Tener muchas estaciones implica detectar el mismo evento en más de una estación. Esto permite comparar los datos
		  	entre estaciones y descartar fluctuaciones grandes, rápidas y asiladas en una sola estación.
		\item 	Cooperación. Estar en una red implica mejorar la comunicación entre las diferentes estaciones. De esta manera los resultados
		  	son mejores y el avance más rápido. 
	\end{itemize}
	\par
	Como ya hemos comentado NMDB es una red mundial, impulsada por la Comisión Europea. Actualmente la red supera las veinte estaciones y
	proporciona datos en tiempo real con resolución de un minuto. Los formatos de los datos están estandarizados entre las diferentes estaciones,
	esto ayuda al análisis científico de estos. Los datos en tiempo real son utilizados para la elaboración de un sistema de alarma
	GLE\cite{GleAlarm}. Como hemos explicado un GLE fuerte puede tener un gran impacto negativo en nuestras vidas. Es beneficioso poder detectar
	estos eventos lo antes posible y este es uno de los objetivos principales del NMDB. 

\section{CALMA}
	CALMA\cite{Medina2013} o \emph{Castilla la Mancha Neutron Monitor} es el primer y único monitor de neutrones en España. Este forma parte del
	NMDB, el equipo técnico responsable de la estación está profundamente implicado en desarrollar sistemas y herramientas que mejoran la red. Un
	ejemplo es el sistema de adquisición implantado en la estación, también implantado en otras estaciones de la red. La estación empezó a operar
	de forma plena en diciembre de 2012 y desde entonces lleva haciéndolo ininterrumpidamente con pequeñas excepciones. Desde su puesta en marcha
	la estación ha registrado 18 Forbush decreases. Desafortunadamente aún no ha habido ningún GLE que detectar, aunque este tendría que ser muy
	energético para ser detectado en una estación con tan poca latitud. A continuación procedemos a hablar más a fondo del estado actual de CALMA,
	hablaremos del sistema de adquisición, base de datos, herramientas y técnicas que son utilizadas.
	\subsection{Sistema de adquisición}
		Como hemos mencionado el sistema de adquisición que está implantado en CALMA es producto del propio equipo\cite{Garcia2014}. Este está
		basado en un sistema empotrado Linux. Si las señales generadas por los tubos contadores superan un determinado umbral, estas son
		covertidas en pulsos digitales por los amplificadores. A continuación, un circuito de adaptación se encarga de elevar los niveles de
		tensión de los pulsos digitales a niveles \emph{TTL} y de adaptar la impedancia de la línea de comunicaciones si el amplificador
		así lo requiere. Seguidamente los pulsos son procesados por una FPGA que es la encargada de medir los eventos por minuto de los 18
		canales. Aunque la estación tan solo tiene 15 tubos contadores el sistema está diseñado para soportar 18, este número es un estándar
		histórico. El software que se ejecuta en el sistema Linux tiene como tarea comunicarse con la FPGA, su labor es recoger las
		\emph{cuentas} de cada minuto y guardar estas en una base de datos. El software hace uso del \emph{Network Time Protocol} para
		asegurar la correcta sincronización temporal entre estaciones, esta sincronización permite contrastar los datos de diferentes
		estaciones. En la figura \ref{fig:acqsis} podemos ver un diagrama de bloques que representa el sistema de adquisición.
		\begin{figure}[h]
			\centering
			\includegraphics[keepaspectratio, width=1\textwidth]{./img/AcqSis.png}
			\caption{Sistema de adquisición.}
			\label{fig:acqsis}
		\end{figure}
		\par
		Actualmente el equipo de CALMA está desarrollando un nuevo sistema de adquisición de datos. El esquema presentado en la figura
		\ref{fig:acqsis} es aplicable al nuevo sistema. Las diferencias entre los dos sistemas de adquisición radican en los componentes. La
		primera de estas está en los amplificadores. Los utilizados en el sistema actual generan pulsos fijos, mientras que los amplificadores
		utilizados en este nuevo sistema de adquisición generan pulsos cuyo ancho es proporcional a la energia de la señal original. El fin es
		medir la energía de las señales generadas por los tubos contadores, esta magnitud es de gran interés técnico.
		\par
		La segunda diferencia está en el circuito de adaptación. Este cambia a fin de soportar los nuevos amplificadores que acabamos de
		explicar. La compatibilidad con los amplificadores utilizados en el sistema de adquisición actual también se mantiene. Esto da mucha
		flexibilidad a la estación, permitiéndole incorporar diferentes amplificadores a la vez.
		\par
		La siguiente diferencia está en la FPGA, pero sobre todo en el \emph{IP core} integrado en cada uno de los dos sistemas. Recordamos
		que en el sistema de adquisición actual este tan solo registra los eventos por minuto para los 18 canales. El nuevo \emph{IP core}
		debe realizar la misma tarea, pero aparte debe poder medir el ancho de los pulsos generados por los amplificadores. Los dos sistemas
		de adquisición utilizan un puerto serie para la comunicación entre software y FPGA. En el nuevo sistema este puerto opera a una
		velocidad mayor a fin de poder transmitir toda la información extra.
		\par
		El nuevo sistema también está pensado para ofrecer una mayor compatibilidad. El fin es poder implantar este en diferentes estaciones
		de forma fácil y rápida. El hardware y software deben ser diseñados con este requisito en mente.
	\subsection{Bases de datos}
		La base de datos que genera CALMA está diseñada de acuerdo con el estándar impuesto por NMDB. Los eventos por minuto de los 18 canales
		se guardan en la tabla \texttt{binTable}. En una segunda tabla, \texttt{CALM\_ori}, se guarda el valor global de la estación, calculado
		a partir de los datos de todos los tubos. Junto a este valor global también se guardan las correcciones que se realizan sobre este. El
		valor global y las correcciones se explican en la última sección de este capítulo. Los datos de esta segunda tabla son revisados
		por un operador humano y finalmente guardados en una tercera tabla cuyo nombre es \texttt{CALM\_rev}. Los datos de estas dos últimas
		tablas se envian al NMDB. Este envío es realizado por un proceso que se ejecuta cada minuto. El envío consiste en calcular las
		diferencias entre los dos conjuntos de datos y tan solo enviar estas.
	\subsection{Herramientas y técnicas}
		El equipo de CALMA hace uso de herramientas que les ayudan a analizar la información desde el punto de vista científico. Actualmente
		no existe ninguna herramienta que proporcione información sobre el estado técnico de la estación, todas las labores relacionadas con
		el mantenimiento son realizadas de forma manual.

\section{Objetivos}
	\textbf{El objetivo de este proyecto es desarrollar el software para el nuevo sistema de adquisición de datos y también desarrollar una
	herramienta que permite operar y monitorizar el estado de la estación}. A continuación procedemos a hacer una descripción más detallada de los
	objetivos de este trabajo.  
	\subsection{Software de adquisición}
		Tal y como hemos explicado en secciones anteriores el equipo de CALMA está desarrollando un nuevo sistema de adquisición. Actualmente
		la mayoría de módulos hardware incluyendo la FPGA están listos. El propósito de este trabajo es realizar el software de adquisición. A
		continuación exponemos algunos de los requisitos más relevantes.
		\begin{itemize}
			\item 	El software debe ser capaz de realizar una correcta comunicación con la FPGA. Esto implica enviar los comandos
				apropiados y ser capaz de interpretar los mensajes de datos trasmitidos por esta. En el capítulo \ref{entornoHW} se
				puede obtener más información sobre la interfaz de comunicación con la FPGA.
			\item 	El software debe ser capaz de calcular el valor global de la estación y las correcciones que se realizan sobre este.
			  	Estos valores son explicados en la última sección de este capítulo.
			\item	El software debe ser capaz de almacenar en local tanto los eventos por minuto de los 18 canales como el dato global y
			  	las correcciones de este.
			\item 	El software también debe ser capaz de mantener una réplica remota de los datos.
			\item 	Al igual que el hardware, el software debe ser capaz de adaptarse fácilmente a diferentes estaciones. Para este
				propósito este debe ser diseñado de tal forma que sea fácil de extender.
			\item 	El software debe ser capaz de poner en marcha la estación completa de forma automática ante la presencia de corriente
			  	eléctrica. 
			\item 	El software debe ser capaz de detectar estados anómalos y actuar conforme a estos. En una estación de este tipo la
				mayoría de veces un funcionamiento anómalo se traduce en no generar datos o generar datos irregulares. Ante la
				presencia de un estado anómalo debe generarse una alerta. El software también debe reaccionar ante dichos estados
				anómalos, la mayoría de problemas se solucionan realizando un reinicio del sistema.
		\end{itemize}
		Aparte de la realización del software en este trabajo también contemplamos el proceso de implantación y mantenimiento de este.
		Volvemos a recordar que los demás módulos que componen el sistema de adquisición ya están desarrollados por el equipo de CALMA. Las
		interfaces de estos son descritas, dado que es necesario para entender este trabajo.
	\subsection{Herramienta Web}
		El segundo objetivo de este trabajo es el desarrollo de una herramienta que facilite la gestión de una estación. La idea de esta
		es del equipo de CALMA. Procedemos a detallar las funcionalidades de la herramienta tal y como las concibe este.
		%TODO Citar el artículo de la herramienta.
		\begin{itemize}
			\item	\textbf{Spike Tool}. Módulo que permitirá la detección de Spikes. Un Spike es un dato anómalo, un dato
				anormalmente grande o pequeño. Usando los datos proporcionados por el sistema de adquisición, este módulo generará
				gráficos. Éstos serán interactivos y su propósito será hacer más fácil la detección de Spikes. Los Spikes detectados
				podrán ser marcados como nulos en el conjunto de datos revisados. 
			\item 	\textbf{Configuración de los parámetros la estación}. Este módulo permitirá la reconfiguración de los parámetros de la
				estación. Esta reconfiguración será llevada a cabo sin interrumpir el proceso de adquisición. Nótese que el software
				de adquisición deberá evolucionar para hacer posible esta funcionalidad.
			\item	\textbf{Alertas}. La herramienta visualizará las alertas producidas por el software de adquisición.
			\item 	\textbf{Histogramas con la intensidad de los eventos}. El nuevo sistema de adquisición proporciona información sobre
				la energía de las partículas detectadas. Este módulo generará histogramas con estos datos. Estos histogramas
				permitirán hacer mejores diagnósticos sobre el funcionamiento de los tubos contadores. 
		\end{itemize}
		Podemos ver que la herramienta ofrece un gran abanico de funcionalidades. Por desgracia en este trabajo tan solo nos centraremos en el
		primer módulo, realizar los demás módulos está fuera del alcance de un trabajo como este. También cabe destacar que tan solo nos
		centraremos en la implementación, no implantaremos ni mantendremos la herramienta. La herramienta será una herramienta Web intuitiva y
		altamente interactiva.

\section{Diseño preliminar}
	En esta sección procedemos a especificar un diseño preliminar para el software de adquisición y para la herramienta Web. La primera subsección
	es dedicada al software de adquisición, mientras que la segunda es reservada para la herramienta Web.
	\subsection{Software de adquisición}
		El sistema de adquisición que se está desarrollando es un sistema empotrado donde hardware y software están muy vinculados. Esto en
		gran medida condiciona el diseño del software que queremos realizar. El software debe ejecutarse sobre una \emph{BeagleBone
		Black}\cite{Beagle} que está integrada con el resto de componentes hardware. La distribución Linux elegida para este trabajo es
		\emph{Angstrom}, la distribución por defecto. El lenguaje de programación elegido es \emph{Python}\cite{Python}, lenguaje interpretado de alto nivel
		con tipado dinámico y sintaxis centrada en producir código legible. Actualmente \emph{Python} es muy popular y existen muchas bibliotecas de
		las que haremos uso. El uso de bibliotecas reduce la carga de trabajo y normalmente resulta en software más robusto. Para la gestión
		de la base de datos hemos elegido \emph{Sqlite3}\cite{Sqlite}, una elección popular en sistemas empotrados. Este es un sistema ligero, de
		alto rendimineto que implementa la base de datos en un único archivo. Esto junto al API bien definido simplifica la administración.
		\par
		En la figura \ref{fig:soft_control_preliminar} podemos ver el diseño preliminar del software de adquisición. Ante la presencia de
		corriente eléctrica este debe ser capaz de inicializarse automáticamente. El primer paso que debe hacer es realizar la configuración
		necesaria para el correcto funcionamiento del sistema de adquisición. Seguidamente debe continuar con el funcionamiento nominal. Este
		consiste de tres pasos que se repiten cíclicamente. El primero es solicitar la información a la FPGA, seguidamente el software debe
		interpretar dicha información y finalmente la información debe ser guardada. Los posibles estados anómalos deben ser detectados y
		contrarrestados. 
		\begin{figure}[h]
			\centering
			\includegraphics[keepaspectratio, width=1\textwidth]{./img/soft_control_preliminar.png}
			\caption{Software de adquisición. Diseño preliminar.}
			\label{fig:soft_control_preliminar}
		\end{figure}
	\subsection{Herramienta Web}
		En la figura \ref{fig:herramienta_web_preliminar} podemos ver el diseño preliminar de la herramienta Web. Como podemos observar el
		diseño está fuertemente basado en el modelo \emph{Cliente-Servidor}\cite{MVCWiki}. En este trabajo utilizamos los términos
		\emph{back-end} y \emph{front-end} para referirnos al \emph{Cliente} y al \emph{Servidor} respectivamente. A continuación explicamos los componentes
		básicos.
		\begin{description}
			\item[Base de Datos.]    
				En este componente residen los datos de nuestra estación. Para la gestión de estos utilizamos un servidor
				\emph{MySQL}\cite{MySql}.
			\item[\emph{Back-End}.]
				Este componente es el encargado de recibir y procesar los mensajes de petición provenientes del \emph{Front-End}.
				Estas peticiones pueden ser de consulta o de acción. En ambos casos este componente procede a comunicarse con la base
				de datos a fin de satisfacer la petición. Finalmente el resultado es trasmitido al \emph{Front-End} en un mensaje de
				respuesta. En el caso de una petición de consulta son devueltos los datos especificados. En el caso de una petición de
				acción es devuelto un mensaje de estado. Para la implementación de este componente hemos utilizado
				\emph{ZendFramework}\cite{ZF} y \emph{Apygility}\cite{Apigility}.
			\item[\emph{Front-end}.] 
				Este componente implementa la interfaz de nuestra aplicación. Es el encargado de presentar la información y manejar
				las peticiones del usuario. El módulo está basado en el patrón \emph{Modelo-Vista-Controlador}. Para implementar la
				\emph{Vista} encargada de presentar los datos hemos utilizado \emph{HighStock}\cite{HighStock}, para el
				\emph{Controlador} responsable de gestionar los eventos empleamos \emph{ExtJs}\cite{ExtJs} y para el modelo encargado
				de manejar los datos utilizamos peticiones \emph{Ajax}\cite{AjaxWiki}.
		\end{description}
		\begin{figure}[h]
			\centering
			\includegraphics[keepaspectratio, width=1\textwidth]{./img/herramienta_web_preliminar.png}
			\caption{Herramienta Web. Diseño preliminar}
			\label{fig:herramienta_web_preliminar}
		\end{figure}
\section{Proceso de adquisición}
	El propósito de este punto es explicar algunos aspectos del proceso de adquisición que son relevantes para este trabajo.
	\par
	Hasta este momento siempre hemos hablado de tubos contadores, en plural, detrás de esto hay una razón. Normalmente las estaciones se componen
	de varios tubos contadores, donde 18 tubos contadores es un estándar. En el proceso de adquisición están envueltos muchos factores
	probabilísticos, esto conlleva a que las medidas en un tubo tengan una gran dispersión. La solución de este problema es tener muchos tubos,
	cuantos más mejor. Combinando datos de diferentes tubos conseguimos reducir esta dispersión. 
	\par
	Tener los datos de muchos tubos contadores permite reducir la dispersión de los datos, sin embargo no es muy práctico trabajar con esos datos
	en crudo. El software del sistema de adquisición actual calcula un valor global a partir de los datos de todos los tubos contadores. El
	software de adquisición que desarrollamos en este trabajo también debe calcular este valor global.  Para este propósito utilizamos el Median
	Algorithm\cite{MedianAlgr}.
	\par
	Anteriormente en este capítulo explicamos las cascadas atmosféricas que son originadas por los rayos cósmicos. Estas cascadas atmosféricas
	dependen de la presión atmosférica. Cuando esta es elevada, es necesaria más energía para que la cascada se propague hasta el nivel terrestre,
	por consecuente los monitores de neutrones registran menos eventos. El valor de la presión atmosférica es monitorizado por los monitores de
	neutrones. Este valor es utilizado para realizar una corrección por presión sobre el valor global de la estación, a lo largo de este trabajo
	utilizamos el término de \emph{valor corregido por presión} para referirnos a este valor. El software de adquisición elaborado para este
	trabajo debe ser capaz de leer el valor de presión desde un barómetro y debe poder calcular esta corrección. 
	\par
	También es realizada una \emph{corrección por eficiencia}. Esta es muy simple y consiste en aplicar un factor multiplicativo al \emph{valor
	corregido por presión}. Este valor es utilizado para para solventar problemas técnicos. Cambios en el entorno inmediato del instrumento o
	cambios en la electrónica utilizada pueden afectan a la cantidad de eventos medidos. Estos cambios son identificados, evaluados y finalmente
	contrarrestados con esta corrección por eficiencia. Este valor dota la estación de consistencia histórica, de esta manera pueden ser
	comparados datos de diferentes intervalos temporales. El software también debe ser capaz de realizar esta corrección. 
	\par
	Al principio de este capítulo explicamos que los tubos contadores están rellenos de gas con propiedades especiales. Sobre este gas es aplicado
	un campo eléctrico. Para la generación de dicho campo son utilizadas fuentes de alimentación de alta tensión. Cambios en la tensión generada
	pueden afectar a la cantidad de eventos registrados. Esto ha conllevado a que el funcionamiento de las fuentes sea monitorizado. El software
	para el sistema de adquisición debe ser capaz de realizar esta operación. Es de esperar que la tensión sea constante, en caso de variaciones
	los datos generados son considerados no consistentes.
	\par
	Un \emph{Spike} es un dato anómalo, un dato anormalmente grande o pequeño. Son datos producidos por mal funcionamientos de la instrumentación,
	cambios bruscos en el entorno inmediato u otros factores desconocidos. Estos están presentes en todas las estaciones. Con la elaboración del
	nuevo sistema de adquisición se pretende reducir el número de \emph{Spikes} generados. Con la elaboración de la herramienta Web pretendemos
	ofrecer un método fácil de identificar y descartar dichos datos.

\chapter{Entorno Hardware}
\label{entornoHW}

En un sistema empotrado el software y hardware están muy vinculados. Para entender el funcionamiento del software de adquisición es muy importante
estar familiarizado con el diseño y funcionalidad del hardware. Impulsados por esta razón en este capítulo procederemos a hacer una descripción de los
aspectos más importantes del hardware que compone el sistema de adquisición. Volvemos a enfatizar que el autor de este trabajo no ha formado parte en
la realización de los módulos hardware que serán descritos a continuación.
\section{BeagleBone Black}
	BeagleBone Black\cite{Beagle}\cite{BeagleWiki} es un computador empotrado, open-source y single board. La placa viene con Linux, distribución
	Angstrom y versión de núcleo 3.8. En la figura \ref{fig:beaglebone} podemos ver un diagrama de bloques que refleja los componentes hardware
	que componen la BeagleBone. A continuación vamos a hacer una breve descripción de los módulos que utilizamos para este trabajo.
	\begin{itemize}
		\item 	ARM CORTEX A8\cite{BeagleCore} y SDRAM. El procesador es suficientemente potente para soportar la distribución Linux y
			satisfacer las necesidades de nuestro software. Los 512MB DDR3 de memoria RAM son suficientes para los propósitos de este
			trabajo.
		\item 	Ethernet Connector. El conector RJ-45 permite establecer una conexión a Internet. Una vez establecida esta conexión los datos
			pueden ser transmitidos al NMDB. También es posible establecer una conexión a la placa vía SSH, de esta manera un operador
			puede realizar operaciones de mantenimiento de forma remota.
		\item	eMMC y microSD. Utilizamos la memoria integrada (eMMC) de 4GB para albergar el sistema operativo y el software de
			adquisición. La microSD es utilizada para guardar los datos y logs producidos por el software de adquisición.
		\item 	Analog Pins. Los 7 pines analógicos permiten usar sensores cuyo output es una señal analógica. Ejemplo de estos sensores son
			las fuentes de alimentación analógicas o los sensores de temperatura, en algunas estaciones la temperatura ambiente se
			monitoriza también. Estos sensores producen señales cuya tensión eléctrica es proporcional al valor de la magnitud medida. 
		\item 	GPIOs. Los \emph{General Purpose Input/Output} permiten trabajar con señales digitales, tanto de entrada como de salida. Un
			ejemplo de uso es la señal de Reset de la FPGA, señal digital activa a bajo nivel.
		\item	UARTs. La placa ofrece 4 puertos serie de los que utilizamos 2 para comunicarnos con la FPGA.
	\end{itemize}
	\begin{figure}[h]
		\centering
		\includegraphics[keepaspectratio, width=1\textwidth]{./img/beaglebone.png}
		\caption{BeagleBone Black. Diagrama de bloques hardware}
		\label{fig:beaglebone}
	\end{figure}
	\par
	Fijándonos en la figura \ref{fig:beaglebone} podemos ver que muchos de los módulos son accesibles mediante los conectores de
	expansión\cite{BeagleWikiExp}. Estos conectores tienen 2x46 pines, de los cuales muchos son multipropósito. Estos pueden ser configurados para tener
	funcionalidades diversas y además esta configuración puede ser realizada de forma dinámica. La configuración de los pines se realiza mediante
	el \emph{Device Tree Overlay}, estructura de datos que se utiliza para describir el hardware. En la realización de este trabajo no lidiamos
	con este mecanismo, sino que utilizamos una librería que facilita el trabajo. La librería utilizada es \emph{Adafruit BeagleBone IO
	Python}\cite{AdaFruitGit}. Esta ayuda a configurar dinámicamente los pines de expansión, además ofrece métodos para
	operar sobre estos una vez configurados.

\section{FPGA}
	La FPGA es el núcleo fundamental del sistema, ya que, se encarga de procesar las señales y transmitir la información al software de
	adquisición. Una FPGA es un circuito integrado, que está formado por unidades lógicas cuya interconexión y funcionalidad pueden ser
	configurados. Esta configuración está definida por el \emph{IP core} integrado, por lo tanto, es este el que establece la funcionalidad de una
	FPGA. En esta sección vamos a hacer una descripción funcional de la FPGA, por consiguiente será explicado el \emph{IP core} de esta.  En la
	figura \ref{fig:fpga} podemos ver un diagrama de bloques que refleja los módulos funcionales y interfaces que el este posee.
	\begin{figure}[h]
		\centering
		\includegraphics[keepaspectratio, width=1\textwidth]{./img/fpga.png}
		\caption{FPGA. Diagrama de bloques.}
		\label{fig:fpga}
	\end{figure}	
	\subsection{Procesado de pulsos}
		Como hemos comentado la FPGA es la encargada de procesar las señales y trasmitir la información a la BeagleBone Black. Esta transmisión
		se realiza mediante una línea serie asíncrona de 1Mbps. La línea está controlada por una UART(Universal Asynchronous
		Receiver-Transmitter) a la que nos referiremos como \emph{UART de pulsos}. En las tablas \ref{tab:FPGAUartPulso},
		\ref{tab:FPGAUartOver} y \ref{tab:FPGAUartCont} podemos ver el formato de los datos transmitidos. Vemos que hay tres mensajes
		diferentes que la FPGA puede transmitir.
		\begin{itemize}
			\item	En la tabla \ref{tab:FPGAUartPulso} podemos ver el formato del primer mensaje que vamos a discutir. Este mensaje se
				compone de tres bytes que dan información sobre el ancho de un pulso. El mensaje permite identificar el canal en el
				que se ha producido el pulso, el nivel (alto/bajo) y la longitud de este. Los pulsos de nivel alto representan la
				detección de una partícula por los tubos contadores y el ancho del pulso es proporcional a la energía de la partícula
				detectada. La longitud de los pulsos de nivel bajo también se mide para poder identificar pulsos con pequeña
				separación temporal, que podrían indicar la presencia de un fenómeno físico denominado multiplicidad.
			\item 	En la siguiente tabla \ref{tab:FPGAUartOver} podemos apreciar el mensaje de estado que la FPGA genera. Dado que la UART usa
				una FIFO de tamaño fijo es posible que la esta se llene y se pierdan mensajes. En este caso la FPGA genera un mensaje
				de estado que es transmitido para indicarle al software que se han producido perdidas de datos.
			\item	Además de medir el ancho de los pulsos la FPGA lleva la cuenta de pulsos recibidos para cada canal. Esta información
				puede ser solicitada por la BeagleBone Black utilizando el comando apropiado(ver tabla \ref{tab:FPGAUartComm}). En la
				tabla \ref{tab:FPGAUartCont} podemos ver el formato en el que se transmite la información de las cuentas. Los bytes 2,
				3 y 4 son transmitidos 18 veces, una vez para cada canal. Después de transmitir la información de las cuentas la FPGA
				reinicia los contadores para cada canal.
		\end{itemize}
		En la figura \ref{fig:fpgaWave} podemos observar un ejemplo de la secuencia de datos transmitidos por la \emph{UART de pulsos}, junto a
		estos podemos apreciar los estímulos que los causan. 

		\begin{table}[h]
			\tiny
			\begin{tabularx}{\textwidth}{|l|c|c|X|c|c|c|c|c|}
				\hline
				\rowcolor[HTML]{C0C0C0} 
				\multicolumn{1}{|r|}{\textbf{Bit}}    	& 7 & 6          & 5 				& 4 	       & 3 	     & 2 	  & 1          & 0 	     \\ \hline
				\cellcolor[HTML]{C0C0C0}\textbf{Byte 1} & 0 & 1          & 0  				& \multicolumn{5}{c|}{Canal (0-17)}				     \\ \hline
				\cellcolor[HTML]{C0C0C0}\textbf{Byte 2} & 1 & Dato (6)	 & Dato (5)      		& Dato (4)     & Dato (3)    & Dato (2)   & Dato (1)   & Dato (0)    \\ \hline
				\cellcolor[HTML]{C0C0C0}\textbf{Byte 3} & 1 & X          & Nivel ('1'->alto, '0'->bajo) & Dato (11)    & Dato (10)   & Dato (9)   & Dato (8)   & Dato (7)    \\ \hline
			\end{tabularx}
			\caption{\emph{UART de pulsos}. Palabra de ancho de pulso}
			\label{tab:FPGAUartPulso}
			\begin{tabularx}{\textwidth}{|l|c|X|X|X|X|X|X|X|}
				\hline
				\rowcolor[HTML]{C0C0C0} 
				\multicolumn{1}{|r|}{\textbf{Bit}} 	& 7 & 6 		       & 5 		       & 4		       & 3 		       & 2		       & 1          	       & 0			\\ \hline
				\cellcolor[HTML]{C0C0C0}\textbf{Byte 1} & 0 & 0                        & OverFlow FIFO Tubo 5  & OverFlow FIFO Tubo 4  & OverFlow FIFO Tubo 3  & OverFlow FIFO Tubo 2  & OverFlow FIFO Tubo 1  & OverFlow FIFO Tubo 0	\\ \hline
				\cellcolor[HTML]{C0C0C0}\textbf{Byte 2} & 1 & OverFlow FIFO Tubo 12    & OverFlow FIFO Tubo 11 & OverFlow FIFO Tubo 10 & OverFlow FIFO Tubo 9  & OverFlow FIFO Tubo 8  & OverFlow FIFO Tubo 7  & OverFlow FIFO Tubo 6	\\ \hline
				\cellcolor[HTML]{C0C0C0}\textbf{Byte 3} & 1 & Almost Full FIFO General & OverFlow FIFO General & OverFlow FIFO Tubo 17 & OverFlow FIFO Tubo 16 & OverFlow FIFO Tubo 15 & OverFlow FIFO Tubo 14 & OverFlow FIFO Tubo 13	\\ \hline
			\end{tabularx}
			\caption{\emph{UART de pulsos}. Palabra de estado}
			\label{tab:FPGAUartOver}
			\begin{tabularx}{\textwidth}{|l|X|c|c|c|c|c|c|c|}
				\hline
				\rowcolor[HTML]{C0C0C0} 
				\multicolumn{1}{|r|}{\textbf{Bit}}    	 & 7 & 6           & 5 		& 4 	      & 3 	    & 2 	 & 1           & 0 	     	\\ \hline
				\cellcolor[HTML]{C0C0C0}\textbf{Byte 1}  & 0 & 1           & 1  	& X	      & X	    & X	  	 & X	       & X	     	\\ \hline
				\cellcolor[HTML]{C0C0C0}\textbf{Byte 2}  & 1 & Dato0 (6)   & Dato0 (5) 	& Dato0 (4)   & Dato0 (3)   & Dato0 (2)  & Dato0 (1)   & Dato0 (0)  	\\ \hline
				\cellcolor[HTML]{C0C0C0}\textbf{Byte 3}  & 1 & Dato0 (13)  & Dato0 (12)	& Dato0 (11)  & Dato0 (10)  & Dato0 (9)  & Dato0 (8)   & Dato0 (7)  	\\ \hline
				\cellcolor[HTML]{C0C0C0}\textbf{Byte 4}  & 1 & X	   & X	 	& X	      & X	    & X		 & Dato0 (15)  & Dato0 (14)	\\ \hline
				\cellcolor[HTML]{C0C0C0}\textbf{Byte 5}  & 1 & Dato1 (6)   & Dato1 (5) 	& Dato1 (4)   & Dato1 (3)   & Dato1 (2)  & Dato1 (1)   & Dato1 (0)  	\\ \hline
				\cellcolor[HTML]{C0C0C0}\textbf{......}  & . & .........   & ......... 	& .........   & .........   & .........  & .........   & .........  	\\ \hline
				\cellcolor[HTML]{C0C0C0}\textbf{Byte 55} & 1 & X	   & X	 	& X	      & X	    & X		 & Dato17 (15) & Dato17 (14)	\\ \hline
			\end{tabularx}
			\caption{\emph{UART de pulsos}. Palabra de cuentas}
			\label{tab:FPGAUartCont}
			\begin{tabularx}{\hsize}{|c|X|}
		  		\hline
				\rowcolor[HTML]{C0C0C0} 
		  		Commando & Descripción                            \\\hline
		  		0x00     & Configura multiplexor para aparato 1   \\\hline
		  		0x01     & Configura multiplexor para aparato 2   \\\hline
		  		0x02     & Configura multiplexor para aparato 3   \\\hline
		  		0x03     & Configura multiplexor para aparato 4   \\\hline
		  		0x10     & Reset general del sistema              \\\hline
		  		0x11     & Solicita la transmisión de las cuentas \\\hline
			\end{tabularx}
			\caption{\emph{UART de pulsos}. Commandos}
			\label{tab:FPGAUartComm}
		\end{table}
		\begin{figure}[h]
			\centering
			\includegraphics[keepaspectratio, width=1\textwidth]{./img/fpgawave.png}
			\caption{FPGA. Diagrama de eventos.}
			\label{fig:fpgaWave}
		\end{figure}
	\subsection{Multiplexor}
		Tal y como explicamos en el capítulo de introducción, el sistema de adquisición debe monitorizar la presión atmosférica, el
		funcionamiento de las fuentes de tensión y en muchos casos la temperatura ambiente. Al igual que muchos otros dispositivos que pueden
		ser necesarios, la mayoría de estos sensores hacen uso de un puerto serie. Estos son manejados según el patron \emph{Master-Slave} y
		generalmente intercambian poca información, por consiguiente, pueden compartir un puerto serie. El mecanismo que resuelve este
		problema está implmentado en el \emph{IP core}. Este consiste en una UART a la que nos referiremos como \emph{UART de extensión}. Esta
		está conectada a un multiplexor, que es el encargado de realizar la conmutación entre los cuatro dispositivos soportados. El estado
		del multiplexor puede cambiarse enviando comandos por la \emph{UART de pulsos}, de acuerdo a la especificación presentada en la tabla
		\ref{tab:FPGAUartComm}. Ejemplo de dispositivos que requiren un puerto serie son el barómetro \emph{BM35} utilizado en CALMA, varias
		fuentes de alimentación comerciales, estaciones meteorológicas, GPS, etc.
	\subsection{Reset}
		El \emph{IP core} permite realizar un Reset del estado interno de la FPGA. Todas las variables son puestas a sus valores iniciales,
		excepto el multiplexor que mantiene su estado. Para la realización de dicho Reset son exportadas tres interfaces. La primera es un
		boton hardware accesible de forma física. La segunda es un commando trasmitido por la \emph{UART de pulsos}, de acuerdo a la
		especificación presentada en la tabla \ref{tab:FPGAUartComm}. Finalmente, la tercera es una señal digital activa a bajo nivel. Dicha
		señal definida como \emph{BS1}, está conectada al pin \emph{P9\_42(GPIO\_7)} de la BeagleBone Black, por consiguiente, es accesible
		desde nuestro software de adquisición.

\chapter{Software de adquisición}
\label{cap2}

Texto explicación introductiva.

\section{Entorno Hardware}
	En un sistema empotrado el software y hardware están muy vinculados. Para entender el funcionamiento del software es muy importante estar
	familiarizado con el diseño y funcionalidad del hardware. Impulsados por esta razón a continuación procederemos a hacer una descripción de
	los aspectos más importantes del hardware que compone nuestro sistema de adquisición. Volvemos a enfatizar que el autor de este trabajo no
	ha formado parte  en la realización de los módulos hardware que serán descritos a continuación.
	\subsection{BeagleBone Black}
		La BeagleBone Black es un computador empotrado, open-source y single board. La placa viene con Linux, distribución Angstrom y versión
		de núcleo 3.8.  Lo más característico de la placa son los 2x46 pines de extensión disponibles. Estos pines son los utilizados para
		integrar este componente con el resto del sistema de adquisición. La mayoría de estos pines son multipropósito, pueden ser
		configurados para tener funcionalidades diversas y además esta configuración puede ser realizada de forma dinámica.
		\par
		La configuración de los pines se realiza mediante el “Device Tree Overlay”, estructura de datos que se utiliza para describir el
		hardware. Para la realización de este trabajo no tendremos que lidiar con este mecanismo, utilizaremos una librería que nos facilite
		el trabajo. La librería utilizada es “Adafruit BeagleBone IO Python”. Esta librería nos ayudara a configurar dinámicamente los pines
		de extensión, además la librería ofrece métodos para operar sobre estos una vez configurados. En la figura 1 podemos ver las
		funcionalidades que los pines de la BeagleBone Black ofrecen y las que vamos a utilizar para este trabajo.
		\begin{table}[h]
	  		\begin{tabular}{|l|l|}
	    			\hline
	    			\rowcolor[HTML]{C0C0C0} 
	    			{\color[HTML]{000000} \textbf{Disponibles}} & {\color[HTML]{000000} \textbf{Usados}} \\ \hline
	    				7 Analog Pins                               & 4 Analog Pins                          \\ \hline
	    				65 Digital Pins at 3.3V                     & 1 Digital Pin                          \\ \hline
	    				2x I2C                                      & 0x I2C                                 \\ \hline
	    				2x SPI                                      & 0x SPI                                 \\ \hline
	    				4 Timers                                    & 0 Timers                               \\ \hline
	    				4x UART                                     & 2x UART                                \\ \hline
	    				8x PWM                                      & 0x PWM                                 \\ \hline
	    				A/D Converter                               & Not used                               \\ \hline
	  		\end{tabular}
			\centering
			\caption{BeagleBone Black Headers.}
			\label{fig:BBBPins}
		\end{table}
	\subsection{FPGA}
		Intro FPGA. La FPGA es el componente que más interactuar con la BeagleBone Black.
		Intro, Reset, UARTS 

\section{Configuración del sistema}
	System Services. Como funcionan. Como las usamos.
	\subsection{WatchDog}
	\subsection{Time Sync}
	\subsection{Software de adquisición}

\section{Sensor Manager}
	\subsection{Pressure}
	\subsection{HVPS}

\section{Reader}
	Explicar funcionamiento.

\section{Counts Pettioner}
	Intro funcionamineto.
	\subsection{Deriva local}
	\subsection{Petición de cuentas}
	\subsection{Petición de datos sensores}
	\subsection{Guardar datos}

\section{DbUpdater}

\section{testing}
	Porque son importantes los unit tests.
	\subsection{moking}
	\subsection{Coverage}

\section{Mantenimiento}
	Dejar el sistema de adquisición de datos funciónando algun tiempo. Hacer analisis de los datos. Describir experiencia.  TODO dejar para el capítuo de Conclusiones.

\chapter{Herramienta Web. Back End}
\label{backend}

Fijándonos en el diseño preliminar de la figura \ref{fig:herramienta_web_preliminar} podemos ver que nuestra aplicación Web está dividida en
\emph{front-end} y \emph{back-end}. Esta separación entre módulos es una técnica popular en diseño software. El \emph{front-end} es el encargado de
la capa de presentación, sobre este hablaremos más en el próximo capítulo. En este capítulo nos centraremos en explicar el \emph{back-end}. Este es el
encargado de procesar las peticiones provenientes del \emph{front-end} y devolver le a este la información solicitada. 
\par
El \emph{back-end} será implementado en PHP\cite{PHP}. Este es un lenguaje diseñado para desarrollo Web y además es una elección muy popular. 
Utilizaremos ZendFramework\cite{ZF}, este es un framework orientado al desarrollo de aplicaciones Web. Junto al framework utilizaremos 
Apigility\cite{Apigility}, herramienta que simplifica la creación y mantenimiento de APIs.
\par
Hemos elegido basarnos en la técnica de arquitectura software REST\cite{Rest}, serie de directrices y métodos para crear aplicaciones Web. Los
aspectos básicos de REST a los que más nos ceñiremos son:
\begin{itemize}
	\item	Cliente Servidor. La aplicación debe seguir el modelo Cliente-Servidor. En nuestro caso estas dos identidades son representadas por el
	  	\emph{back-end} y el \emph{front-end}.
	\item	Sin estado. La comunicación entre Cliente y Servidor no mantiene ningún tipo de sesión. La respuesta del Servidor tan solo depende de
	  	la estimulación proporcionada por el Cliente y no es influenciada por los eventos de comunicación anteriores.
	\item	Conjunto bien definido de recursos y operaciones aplicables a esos recursos. En REST por recurso entendemos elementos de información.
	  	Estos recursos serán los solicitados en las peticiones del Cliente. Estos recursos tienen asociada una URL exclusiva que se usa para
		identificarlos.
\end{itemize}
Si nos volvemos a fijar en la figura \ref{fig:herramienta_web_preliminar} podemos ver que en el \emph{back-end} existe la separación entre
\emph{Modelo} y \emph{Controlador}. Esta separación sigue el patrón de diseño MVC\cite{MVCWiki}. Podemos ver que el componente de \emph{Vista} no
existe, en este caso el \emph{front-end} será nuestro componente de \emph{Vista}.
\par
Volviendo a los recursos REST, cada petición recibida por el \emph{back-end} apuntara a un recurso y especificara una operación sobre este. Con el fin
de responder a las peticiones cada recurso tendrá su parte de \emph{Controlador} y \emph{Modelo}. En este capítulo procederemos a explicar los
servicios REST, pero primero explicaremos algunos aspectos técnicos como la base de datos o el ZendFramework.
\section{Protocolo de comunicación}
  	Hemos hablado que entre el \emph{back-end} y \emph{front-end} existe una comunicación. Siendo esta una herramienta Web como es de esperar el
	protocolo de comunicación es HTTP\cite{HTTP}. El \emph{back-end} recibirara \emph{HTTP request} y responderá con \emph{HTTP response}. Las
	peticiones especifican su recurso mediante el URL, al que son anexados los parámetros necesitados. El protocolo ofrece una serie de
	métodos, los más utilizados en este trabajo son \emph{GET} y \emph{POST}. 
	\par
	El método \emph{GET} pide una respuesta del recurso especificado. En función de los parámetros anexados al URL la respuesta cambia.
	\par
	El método \emph{POST} no suele tener parámetros, sin embargo tiene un campo especial. En este cambo se encuentran una serie de datos que serán
	procesados por el \emph{back-end}. Estos datos acabaran creando o actualizando entradas en la base de datos. 
	\par
	Haciendo una pequeña abstracción podemos ver los métodos \emph{GET} y \emph{POST} como \emph{leer y escribir} el recurso especificado.
\section{ZendFramework y Apigility}
	Para la realización de este trabajo utilizaremos ZendFramework, este es un framework orientado al desarrollo de aplicaciones y servicios Web.
	Basado en PHP 5.3+ el framework sigue un diseño orientado a objetos. Esta enfocado para crear aplicaciones siguiendo el patrón MVC. Ofrece
	abstracciones para las bases de datos, autenticación y validación de parámetros. El framework también disfruta de una amplia comunidad de 
	usuarios. Todas estas ventajas nombradas anteriormente son la razón para elegir este framework en nuestro trabajo.
	\par
 	Para  los principiantes el framework ofrece la aplicación esqueleto. Esta aplicación consiste del código mínimo necesario para construir una
	aplicación usando ZendFramework, no tiene ninguna funcionalidad y esta pensada para ser extendida. En este trabajo hemos empezado con esta
	aplicación esqueleto.
  	\par 
  	Apigility es una herramienta creada por el equipo responsable de ZendFramework. La herramienta puede utilizarse sin necesidad del framework,
	sin embargo esta se integra muy bien con este. La herramienta facilita la creación y mantenimiento de aplicaciones Web. En nuestro caso la
	herramienta nos ayudara crear los recursos y métodos de nuestra aplicación REST. La aplicación ofrece un entorno gráfico accesible desde un
	navegador Web. El uso de la herramienta es muy fácil e intuitivo.
\section{Bases de datos}
	Tal y como hemos explicado al principio del capítulo la función del \emph{back-end} es  procesar las peticiones del \emph{front-end} y
	responderle con la información solicitada. Esta información es guardada en dos bases de datos. Para la gestión de las bases de datos
	utilizamos MySql\cite{MySql}. ZendFramework ofrece abstracciones para las bases de datos, esto facilita el trabajo con estas. En la
	figura \ref{fig:tablas} podemos ver el esquema de las tablas con las que vamos a trabajar.
	\par
	En la tabla \texttt{binTable} guardamos la información de las cuentas de cada canal. Junto a esa información guardamos las lecturas del
	barómetro, las fuentes de alta tensión y la fecha y hora actuales. La resolución de los datos es de un minuto.
	\par
	En la tabla \texttt{CALM\_ori} guardamos el valor global y  las correcciones sobre este. La lectura de presión atmosférica también es guardada.
	La resolución de estos datos es también de un minuto.
	\par
	En la tabla \texttt{CALM\_rev} guardamos la revisión de los datos de \texttt{CALM\_ori}. Vemos que la tabla tiene el mismo esquema con dos
	campos adicionales. En estos dos campos se guardan la fecha de última modificación y la versión. Los datos en esta tabla son introducidas por
	un operario. Cuando este encuentra un dato corrupto en los datos originales, puede crear una entrada en esta base de datos para señalarlo.
	\begin{figure}[h]
		\centering
		\includegraphics[keepaspectratio, width=1\textwidth]{./img/tablas.png}
		\caption{Esquema de las tablas.}
		\label{fig:tablas}
	\end{figure}
\section{Servicios REST}
  	Es esta sección procederemos a explicar uno a uno los servicios REST que nuestro \emph{back-end} ofrece. Cada servicio REST tiene su parte de
	\emph{Modelo} y \emph{Controlador}. La parte del \emph{Modelo} es la encargada de extraer la información necesaria de la base de datos. El
	\emph{Controlador} es el encargado de procesar las solicitudes, pedirle los datos al controlador y enviar la respuesta al \emph{front-end}.


\chapter{Herramienta Web. \emph{Front-End}}
\label{frontend}
Nuestra aplicación Web está dividida en \emph{back-end} y \emph{front-end}. En el capítulo anterior se describió el \emph{back-end}. El propósito de
este capítulo es describir el \emph{front-end}, que es el encargado de la capa de presentación.
\par
El \emph{front-end} se implementa en JavaScript\cite{JavaScript}. Este es un lenguaje de programación soportado por la mayoría de navegadores, que
permite dotar de funcionalidad extra a las páginas Web. Actualmente el uso de este lenguaje está tan extendido y avanzado que permite crear
aplicaciones enteras para navegadores Web. Existen múltiples frameworks escritos en JavaScript que facilitan la creación de aplicaciones Web. En este
trabajo utilizamos dos: \emph{Sencha ExtJs}\cite{ExtJs} y \emph{HighStock}\cite{HighStock}.
\par
\emph{Sencha ExtJs} es un framework orientado a la creación de aplicaciones Web interactivas. Se trata de un framework de propósito general que ofrece
abstracciones para gestionar nuestros datos, arquitectura MVC, componentes gráficos de control y otros. 
\par
\emph{HighStock} es un framework con un propósito específico, que es la creación de gráficos. Podemos elegir entre diferentes tipos de gráficos, que pueden
ser: lineas, barras, áreas, \emph{candlestick} y otros. Los gráficos generados son altamente interactivos lo que permite ocultar series, navegar,
realizar \emph{zoom} y mucho más.
\par
Empezaremos este capítulo aclarando los aspectos técnicos relacionados con estos dos frameworks. Explicaremos como usarlos, que funcionalidad nos
proporcionan, consideraciones que debemos tener en cuenta.
\par
Como podemos ver en la figura \ref{frontend} nuestro \emph{front-end} está basado en el patrón de diseño
\emph{modelo-vista-controlador}\cite{MVCWiki}. En la segunda parte de este capítulo hablaremos más a fondo sobre la división impuesta por este patrón
de diseño.
\begin{figure}[h]
	\centering
	\includegraphics[keepaspectratio, width=1\textwidth]{./img/frontend.png}
	\caption{\emph{Front-end}. Patrón MVC.}   
	\label{fig:frontend}
\end{figure}
\par
En nuestra aplicación también hacemos una división funcional, diferenciamos entre tres módulos que son \texttt{Spike}, \texttt{SpikeRevised} y
\texttt{ChannelStats}. En las secciones finales de este capítulo hablaremos sobre estos módulos. 

\section{Sencha ExtJs}
	El propósito de esta sección es explicar alguno de los aspectos básicos del framework. Empezaremos explicando como crear una aplicación básica
	con \emph{ExtJs}. En la mayoría de los casos existe un único documento HTML\cite{HTML} que contiene toda la aplicación. En él tenemos que cargar dos
	\emph{scripts} de la siguiente forma.
    		\begin{center} \texttt{<script type="text/javascript" src=\textquotedbl extjs/ext-all-debug.js\textquotedbl ></script>}  \end{center}
    		\begin{center} \texttt{<script type="text/javascript" src=\textquotedbl app.js\textquotedbl ></script>}  \end{center}
	El primer \emph{script} contiene el framework que queremos utilizar. Es conveniente destacar que esta es una versión concebida para el proceso
	de desarrollo. Para la versión final es conveniente usar el \texttt{ext-all.js}, que es una versión comprimida del mismo \emph{script}.
 	\par
	El segundo \emph{script} es el que contiene la lógica de nuestra aplicación. A continuación podemos ver un ejemplo del código mínimo que este
	\emph{script} debe contener. El código presentado se explicará a fondo.
	\begin{lstlisting}[style=myJs]
Ext.application({
   name: 'HelloExt',
   launch: function() {
      Ext.create('Ext.container.Viewport', {
         layout: 'fit',
            items: [{
               title: 'Hello Ext',
               html : 'Hello! Welcome to Ext JS.'}] 
      }); 
   } 
});
	\end{lstlisting}
	\par
	En la primera línea hacemos uso del singleton \texttt{Ext}. Este es un objeto que encapsula todas las clases y métodos proporcionados por el
	framework. La función utilizada \texttt{Ext.application} carga e inicializa una instancia de la clase \texttt{Ext.app.Application}. Esta clase
	representa una aplicación \emph{ExtJs} \emph{single-page}. La llamada a esta función crea la variable global \texttt{MyApp}, que debe contener todas las
	clases e instancias de nuestra aplicación.
 	\par
	En la segunda línea declaramos el nombre de nuestra aplicación. Seguidamente definimos la función \texttt{launch}. Esta función es ejecuta
	cuando se lanza la aplicación. La primera función invocada es \texttt{Ext.create} que crea una instancia de la clase proporcionada, en este
	caso \texttt{Ext.container.Viewport}. \texttt{Viewport} es un \emph{contenedor} que representa el área de aplicación y puede haber tan sólo
	uno por aplicación. 
 	\par
	La interfaz de usuario en una aplicación \emph{ExtJs} se compone de \emph{componentes}. Un \emph{contenedor} es un \emph{componente} especial que
	contiene otros \emph{componentes}. En la figura \ref{fig:comps} podemos ver un ejemplo que ilustra esta jerarquía.
	\begin{figure}[h]
		\centering
		\includegraphics[keepaspectratio, width=1\textwidth]{./img/comps.png}
		\caption{\emph{Sencha ExtJs}. Jerarquía de componentes.}   
		\label{fig:comps}
	\end{figure}
 	\par
	El \texttt{Viewport} es el \emph{contenedor} que gestiona el área de representación del navegador Web. Los \emph{compenentes} de un
	\emph{contenedor} se especifican en el campo \texttt{items} que es una lista. En el ejemplo presentado tan sólo tenemos un \emph{componente},
	pero pueden añadirse más.
 	\par
	Fijándonos en el código de ejemplo podemos ver que antes de definir el campo \texttt{items} definimos el campo \texttt{layout}. El
	\texttt{layout} especifica la forma en la que se posicionan y ajustan los \emph{componentes} hijos dentro del padre. En este caso el
	\texttt{layout} especificado es \texttt{\cc fit\cc}, donde un único hijo ocupa todo el espacio del padre.
	\subsection{Component Manager}
		\emph{ExtJs} ofrece el singleton \texttt{Ext.ComponentManager} que provee un registro con todos los componentes. Este facilita la referencia
		a elementos desde cualquier punto del código. El propósito de esta sección es explicar como hacer uso de esta facilidad.
		\par
		Empezaremos especificando un requisito que  los componentes deben cumplir, estos deben definir el atributo \texttt{itemId}. Este
		atributo es el identificador del componente. El ámbito que tiene este identificador es local al contenedor que contiene el componente.
		Esto hace que los identificadores no tengan que ser únicos para toda la aplicación. 
		\par
		Es el singleton \texttt{Ext.ComponentQuery} el que permite realizar búsquedas de componentes en función del \texttt{itemId}.
		Concretamente es \texttt{Ext.ComponentQuery.query()} el método que permite realizar las búsquedas. Este método acepta dos parámetros.
		El primer parámetro, \texttt{selector}, es un String y especifica el \texttt{itemId} que queremos buscar. El segundo parámetro,
		\texttt{root} es opcional y especifica el contenedor dentro de cual queremos hacer la búsqueda. Si el segundo parámetro se omite la
		búsqueda se realizará entre todos los componentes. La función devuelve un array con todos los componentes que reúnen las condiciones,
		pudiendo ser un array vacío. Eventualmente podemos hacer búsquedas más complejas, que nos permiten buscar por tipo, atributos y mucho
		más. En este trabajo tan sólo hemos utilizado consultas simples.
		\par
		La clase \texttt{Ext.container.Container} implementa las funciones \texttt{down()} y \texttt{query()}. Estas dos realizan una llamada
		a \texttt{Ext.ComponentQuery.query()} teniéndose a si mismo como parámetro \texttt{root}. Esto les permite hacer una búsqueda entre
		sus hijos.
		\par
		También existe el método \texttt{up()} implementado por \texttt{Ext.Component}. Este permite hacer una búsqueda similar a las
		anteriormente descritas. En este caso se navega hacia arriba en la jerarquía de componentes y se busca por componentes que reúnan los
		criterios. Esta función puede ser invocada sin ningún parámetro, caso en el que devuelve el padre inmediato del componente.
		\par
		Es conveniente citar el parámetro \texttt{id} y la función \texttt{Ext.getCmp()}. Estos dos ofrecen una funcionalidad parecida a la
		anteriormente explicada, pero no deben ser usados con ese propósito. Estan considerados como obsoletos y pueden dar lugar a
		colisiones. 
	\subsection{Layouts}
		El propósito de esta sección es explicar los \texttt{layouts} utilizados en este trabajo. Tal y como explicamos anteriormente, el
		\texttt{layout} especifica la forma en la que se posicionan y ajustan los componetes hijos dentro del padre. Tan sólo explicaremos la
		funcionalidad de estos sin centrarnos en el uso que les hemos dado, este será explicado en las secciones venideras cuando proceda. En
		la figura \ref{fig:layouts} podemos ver una representación de los \texttt{layouts} que describiremos a continuación.
		\begin{description}[style=unboxed,leftmargin=0cm,labelwidth=1cm]
			\item[\texttt{Absolute}] Los \emph{componentes} hijo son posicionados mediante el uso de coordenadas \texttt{X,Y}. Las
			  dimensiones de estos también se definen de forma estática mediante el uso de dos atributos, \texttt{height} and
			  \texttt{width}. La posición y tamaño de los componentes se puede cambiar mediante el uso de las funciones
			  \texttt{setPosition()} y \texttt{setSize()}.
			\item[\texttt{Accordion}] Este layout se asemeja a un acordeón. Los \emph{componentes} hijo pueden ser expandidos y
			  colapsados, teniendo tan sólo uno expandido a la vez. Los \emph{componentes} son ordenados en vertical ocupando todo el
			  espacio disponible. Las funciones \texttt{expand} y \texttt{collapse} permiten expandir y colapsar los paneles.
			  También están presentes multitud de manejadores de eventos que podemos sobrescribir. En nuestro caso hacemos uso del
			  \texttt{beforeExpand} que se dispara antes de expandir un \emph{componente}. 
			\item[\texttt{Border}] Este layout permite tener hasta cinco \emph{componentes} dispuestos de la manera que podemos ver en la
			  figura \ref{fig:layouts}. Los componentes hijos deben especificar el atributo \texttt{region} que determina la posición que
			  tendrán. El atributo puede tomar los siguientes valores: \texttt{[north, east, south, west, center]}. No es necesario que
			  estén presentes los cinco hijos, podemos omitir los que no sean necesarios.
			\item[\texttt{Card}] Este layout maneja multiples \emph{componentes} hijo. Los hijos ocupan todo el espacio disponible, por lo
			  que se solapan entre ellos. Esto hace que solamente uno sea visible a la vez. El layout asemeja una baraja de cartas donde
			  solamente una carta puede estar en la parte superior. La función que permite hacer visible un hijo es \texttt{setActiveItem()}.
			  Esta función acepta el \emph{componente}, el \texttt{itemId} o el índice de este.
			\item[\texttt{Fit}] Este es un layout muy simple, que tan sólo acepta un hijo, y este ocupa todo el espacio que el padre ofrece.
			  El \emph{componente} hijo se expande automáticamente para ajustarse al tamaño del padre.
			\item[\texttt{HBox y VBox}] Estos son dos layouts diferentes, pero muy parecidos. \texttt{HBox} organiza sus elementos hijos
			  de forma horizontal a lo largo del \emph{componente} padre. \texttt{VBox} hace lo mismo, pero la organización es de forma
			  vertical. Los componentes hijos pueden especificar el atributo \texttt{flex}, que determina como será repartido el
			  espacio disponible entre los diferentes hijos. Para ilustrar el funcionamiento del \texttt{flex} expondremos un ejemplo.
			  Teniendo dos \emph{componentes} hijo con \texttt{1} y \texttt{2} de \texttt{flex} respectivamente, el primero ocupará 1/3 y
			  el segundo 2/3 del espacio total.
		\end{description}
		\begin{figure}[h]
			\centering
			\includegraphics[keepaspectratio, width=1\textwidth]{./img/layouts.png}
			\caption{\emph{Sencha ExtJs}. Layouts.}   
			\label{fig:layouts}
		\end{figure}

	\subsection{Lazy instantiation}
		La inicialización perezosa es una técnica que consiste en retrasar la creación de recursos hasta el momento en el que sea necesario su
		uso. Como hemos visto en la sección dedicada a los layouts en muchos casos los componentes no son visibles. Tener que crear todos los
		componentes al inicializar la aplicación no es muy eficiente, dado que tan sólo unos pocos son visibles. Crear tan sólo los
		componentes visibles según se vayan necesitando supone un incremento en el rendimiento. En páginas pequeñas con pocos componentes el
		aumento no es apreciable, sin embargo en páginas que manejan multitud de componentes el incremento puede ser drástico. 
		\par
		Para implementar esta inicialización perezosa \emph{ExtJs} se basa en la jerarquía de componentes previamente explicada. Cuando un componente
		es creado también se crean todos los componentes hijos que deben ser visibles en ese momento. Tal y como explicamos los componentes
		hijos se definen en el campo \texttt{items}. A continuación podemos ver un pequeño ejemplo. 
		\begin{lstlisting}[style=myJs]
items:[
   Ext.create('Ext.form.field.Text',{
      fieldLabel:'Foo'}),
   Ext.create('Ext.form.field.Text', {
      fieldLabel: 'Bar'})]
		\end{lstlisting}
		\par
		El ejemplo presentado anteriormente no habilita la inicialización perezosa. Para este propósito tenemos que evitar el uso de la
		función \texttt{Ext.create()}. Tenemos que especificar los componentes hijos como objetos de configuración. A continuación se presenta
		un ejemplo.
		\begin{lstlisting}[style=myJs]
items: [{
   xtype: 'textfield',
   fieldLabel: 'Foo'},
{
   xtype: 'textfield',
   fieldLabel: 'Bar'}]
		\end{lstlisting}
		\par
		La diferencia que podemos apreciar entre los dos ejemplos es el atributo \texttt{xtype}. Este especifica la clase que tendrá el
		componente.  Como podemos ver no se especifica el nombre completo de la clase. \texttt{xtype} acepta una serie de atajos para
		referenciar a las clases. En nuestro caso la clase \texttt{Ext.form.field.Text} es referenciada por el \texttt{xtype:\cc
		textfield\cc}.
		\par
		La lista completa de \texttt{xtypes} aceptados se puede consultar en la documentación de \emph{ExtJs}\cite{ExtJsDoc}. Eventualmente podemos
		extender esta lista declarando nuestros propios \texttt{xtypes}. Esto se hace especificando el campo \texttt{xtype} a la hora de
		definir nuestra clase. A continuación presentamos un pequeño ejemplo.
		\begin{lstlisting}[style=myJs]
Ext.define('MyApp.PressMeButton', {
   extend: 'Ext.button.Button',
   xtype: 'pressmebutton',
   text: 'Press Me'});
		\end{lstlisting}

	\subsection{Events}
		Las clases y componentes de \emph{ExtJs} disparan multitud de eventos a lo largo de su ciclo de vida. Los eventos se disparan cuando algo
		interesante le ocurre a nuestro componente. Un ejemplo es el evento \texttt{afterrender} que se dispara después de mostrar en pantalla
		el componente.
		\par
		Los eventos son muy interesantes porque podemos definir funciones, \texttt{listeners}, que se ejecutan cuando se producen dichos
		eventos. A continuación presentamos un ejemplo en el que podemos ver cómo manejar el evento \texttt{click}, que se dispara al hacer
		\emph{click} sobre un componente. 
		\begin{lstlisting}[style=myJs]
Ext.create('Ext.Button',{
   text: 'Click Me',
   renderTo: Ext.getBody(),
   listeners:{
      click: function(){
         Ext.Msg.alert('I was clicked!');}
   }
});
		\end{lstlisting}
		\par
		\emph{ExtJs} también permite configurar eventos propios. Estos se declaran como si fuesen eventos normales. La peculiaridad está en que somos
		nosotros los que debemos dispararlos haciendo uso de la función \texttt{fireEvent}. Esta acepta como parámetros el evento que queremos
		disparar y todos los parámetros que la función encargada de este necesite. A continuación podemos ver un pequeño ejemplo de como
		declarar y disparar un evento propio, que tiene como nombre \texttt{myEvent}.
		\begin{lstlisting}[style=myJs]
var button = Ext.create('Ext.Button',{
   text: 'Custom Event',
   renderTo: Ext.getBody(),
   listeners:{
      myEvent: function(number){
         Ext.Msg.alert('Custom event fired, number: '+number);}
   }
});
button.fireEvent('myEvent', 42);
		\end{lstlisting}
	\subsection{Componentes}  %Muy mal. %Si sobra contenido esta sección tiene que ser eliminada.
		\emph{ExtJs} ofrece multitud de componentes, que son subclases del \texttt{Ext.Component}, cada uno con un propósito diferente. El objetivo
		de esta sección es describir los componentes usados en este trabajo. Nos centraremos en los aspectos técnicos sin especificar el uso
		concreto que les hemos dado. Este será explicado en secciones futuras cuando proceda. 
		\begin{description}[style=unboxed,leftmargin=0cm]
			\item[\texttt{Ext.button.Button xtype:button}] Este componente permite crear un botón. El propósito de un botón es ofrecer un
			  medio fácil al usuario para interactuar con la aplicación. Los botones que \emph{ExtJs} ofrece son altamente configurables, tanto
			  funcionalmente como estéticamente. Podemos tener botones normales, botones \emph{toggle}, botones que despliegan menús y
			  mucho más. Para dar funcionalidad a estos botones tenemos que escuchar los eventos que estos disparan. Los eventos más
			  comunes son \texttt{click}, \texttt{toggle}, \texttt{mouseover} y \texttt{mouseout}.
			\item[\texttt{Ext.form.field.Date xtype:datefield}] Este componente proporciona un campo de entrada de fecha. Es posible
			  especificar el formato deseado en el que introducir la fecha, y si este formato no es respetado el campo es subrayado en
			  color rojo indicando al usuario la inconsistencia. Eventualmente este campo puede desplegar un selector de fechas que
			  facilita el proceso de introducir la fecha deseada. Al igual que los demás componentes este dispara multitud de eventos que
			  podemos manejar.
			\item[\texttt{Ext.form.field.Time xtype:timefield}] Este componente es muy similar al anteriormente descrito, y proporciona
			  un campo de entrada de tiempo. También acepta una configuración del formato deseado y advierte  al usuario cuando este no es
			  respetado. El componente puede desplegar un selector de tiempo que facilita la tarea de introducir el valor deseado.
			\item[\texttt{Ext.form.Label xtype:label}] Este componente permite generar una etiqueta de texto. Este componente es muy
			  similar al tag HTML <label>. Este componente permite configurar múltiples aspectos y manejar múltiples eventos.
			\item[\texttt{Ext.panel.Panel xtype:panel}] Este componente es una parte fundamental en la creación de aplicaciones con
			  \emph{ExtJs}.
			  Esta pensado para ser utilizado como un contenedor, contener otros componentes. Este componente es altamente configurable
			  pero el aspecto más importante es el \texttt{layout} que determina como se posicionan los componentes hijos. 
			\item[\texttt{Ext.tab.Panel xtype:tabpanel}] Este componente es una extensión del componente anterior. Este ofrece múltiples
			  pestañas en la que organizar los componentes hijos.  Esto permite incluir una cantidad mayor de contenido en el mismo
			  espacio. La barra de pestañas que el componente ofrece es altamente configurable. 
			\item[\texttt{Ext.grid.Panel xtype:gridpanel}] Este componente es un contenedor capaz de presentar una tabla. Las tablas son
			  altamente interactivas, estas permiten al usuario ordenar los datos, hacer búsquedas, eliminar datos, añadir datos y mucho
			  más.
		\end{description}
	\subsection{History stack}
		La aplicación Web que estamos desarrollando es una aplicación \emph{single-page}. Este modelo rompe con el patrón tradicional de
		navegación por el historial usando los botones \emph{Forward/Back}. Esto presenta un problema de usabilidad cuando el usuario utiliza
		el botón de navegación hacia atrás, ya que lo esperado es volver al estado anterior de la aplicacióin. Sin embargo se carga la
		página anterior del historial y nuestra aplicación \emph{single-page} es descartada.
		\par
		La solución que la mayoría de aplicaciones \emph{single-page} han adoptado se basa en el \emph{Fragment Identifier}  o \emph{hash},
		nombre que recibe porque va precedido del símbolo \textbf{\#}. El \emph{hash} es una parte opcional de la URL que históricamente se
		utilizaba para navegar por documentos largos. Un cambio en el \emph{hash} hace que se muestren diferentes partes del documento.
		\par
		Actualmente el \emph{hash} es usado por aplicaciones \emph{single-page} para manipular la pila de historial. El hash es cambiado
		acuerdo al estado de la aplicación. Esto hace que los cambios en la aplicación sean registrados en la pila de historial.
		\par
		Al pulsar el botón de navegación hacia atrás el \emph{hash} de la URL cambia a su estado anterior. Nuestra aplicación debe ser capaz
		de detectar los cambios en el \emph{hash} y actuar acuerdo a estos. De esta manera podemos preservar la funcionalidad de los botones
		de navegación \emph{Forward/Back}. 
		\par
		Para implementar la funcionalidad anteriormente descrita \emph{ExtJs} ofrece el singleton \texttt{Ext.util.History}. Para inicializar este
		módulo tenemos que invocar la función \texttt{Ext.History.init()}. Una vez inicializado este módulo podemos hacer cambios en el
		\emph{hash} invocando la función \texttt{Ext.History.add()}. Esta función acepta como parámetro el valor del \emph{hash}. 
		\par
		Para detectar y actual acorde a los cambios del \emph{hash} tenemos que inicializar un \texttt{listener}. El evento que vamos a
		capturar es \texttt{change}. A continuación podemos ver un pequeño código de como inicializar el \texttt{listener}.
		\begin{lstlisting}[style=myJs]
Ext.History.on('change', function(hash){
   switch(hash){
      case: 'Foo':
         //Actuar acuerdo al hash Foo
         break;
      case: 'Bar'
         //Actuar acuerdo al hash Bar
         break;
   };
});
		\end{lstlisting}
		\par
		El propósito de esta sección ha sido explicar el problema que surge con el historial de navegación y como prevenirlo haciendo uso del
		\texttt{Ext.History}. Del uso concreto que hemos hecho de este módulo hablaremos en secciones futuras de este capítulo.

\section{HighStock}
	El propósito de esta sección es explicar alguno de los aspectos básicos del framework. Este permite añadir gráficos interactivos de forma
	simple a páginas o aplicaciones Web. Es conveniente destacar la diferencia entre \emph{HighCharts} y \emph{HighStock}. El primero tiene un propósito más
	general, dado que ofrece mayor variedad de gráficos. El segundo tiene un propósito más específico, ya que está enfocado a la representación
	gráfica de valores que evolucionan respecto al tiempo. A continuación presentaremos el código mínimo necesario para crear un gráfico con
	\emph{HighStock}. Basándose en este código explicaremos los aspectos técnicos de \emph{HighStock}.
	\begin{lstlisting}[style=myJs]
<script src="http://code.highcharts.com/stock/highstock.js"></script>  // Cargar el framework

myChart = new Highcharts.StockChart({
   chart: {
      renderTo: container}, //HTML element reference
   series: [{
      name: 'My First Series',
      data: myData // predefined JavaScript array
   }]
});
	\end{lstlisting}
	\par
	Como podemos ver para crear un gráfico invocamos la función \texttt{StockChart()}. Esta función acepta como parámetro un objeto que
	define la configuración del gráfico. El formato de este objeto puede verse en la documentación del framework\cite{HighStockDoc}. En nuestro
	caso tan sólo especificaremos dos atributos. La mayoría de los atributos tienen un valor por defecto por lo que no es necesario especificarlos
	explícitamente. 
	\par
	El primer atributo que fijamos es \texttt{chart}. Este atributo es un objeto que a su vez determina aspectos básicos del gráfico. El
	atributo que especificamos es \texttt{renderTo}, el elemento HTLM en el que nuestro gráfico será presentado.
	\par
	El segundo atributo es \texttt{series}. Este es un array de objetos que definen las series de nuestro gráfico. En este
	ejemplo tenemos tan sólo un objeto, por lo que nuestro gráfico tendrá tan sólo una serie. Definiendo el atributo \texttt{name} le damos un
	nombre a nuestra serie y con el atributo \texttt{data} le damos datos a la misma. El atributo \texttt{data} tiene que ser un
	array que contenga los datos. Este array puede tener múltiples formatos que se pueden utilizar según convenga. Para ello se puede consultar la
	documentación proporcionada\cite{HighStockDoc}.
	\par
	En este punto volveremos al \texttt{renderTo} a fin de explicar la integración entre \emph{HighStock} y \emph{Sencha ExtJs}. En secciones anteriores
	explicamos que la interfaz de usuario de una aplicación \emph{ExtJs} se constituye mediante componentes. Estos componentes son abstracciones con
	funcionalidades muy extendidas pero en el fondo son elementos HTML. El atributo \texttt{renderTo} puede tomar el valor de una de las etiquetas
	HTML, a fin de mostrar un gráfico \emph{HighStock} en un componente de una aplicación \emph{ExtJs}. El elemento HTML de un componente es accesible de la
	siguiente forma.
    		\begin{center} \texttt{Ext.Component.getEl().dom}  \end{center}
	\subsection{Agrupación de datos y \emph{Candlestick}} 
		\begin{wrapfigure}{R}{0.42\textwidth}
			\centering
	        	\includegraphics[keepaspectratio, width=0.40\textwidth]{./img/candlestick.png}
			\caption{Gráfico Candlestick.}
			\label{fig:candlestick}
		\end{wrapfigure}
		El objetivo de esta sección es explicar el gráfico \emph{Candlestick} o gráfico de velas. Primero explicaremos el motivo por el que
		utilizamos este tipo de gráfico y seguidamente como interpretarlo.
		\par
		El problema radica en la gran cantidad de datos que tienen que ser representados. El sistema de adquisición genera una entrada de
		datos cada minuto, 1440 datos al día o 43200 datos al mes aproximadamente. Estas cantidades de datos son excesivas comparadas con las
		resoluciones de pantalla actuales. Consideremos una pantalla estándar con 1280 píxeles de ancho, es imposible representar 43200 datos.
		\par
		Para dar solución a este problema hemos recurrido a agrupar los datos. El fin es crear un número de grupos que no exceda el ancho de
		píxeles de nuestra pantalla. Para cada grupo tenemos que calcular un valor representativo. Una técnica común es calcular el valor
		medio de los grupos.  Esto tiene el inconveniente de aplanar los Spikes, algo que queremos resaltar.
		\par
		Para evitar aplanar los Spike para cada grupo calcularemos valores más representativos que la media. En concreto calcularemos cuatro
		valores, máximo, mínimo, \emph{open} y \emph{close}. El \emph{open} y \emph{close} son la media más la desviación típica y la media
		menos la desviación típica respectivamente.
		\par
		El agrupamiento de datos y cálculo de estos cuatro valores para cada grupo está implementado en el \emph{back-end}. Los servicios RPC
		que nos permiten acceder a estos datos son \texttt{nmdbOriginalGroup} y \texttt{nmdbRevisedGroup}. Estos servicios fueron explicados
		en el capítulo dedicado al \emph{back-end}. En este punto tan sólo nos interesa destacar el argumento \texttt{points} que estos dos
		servicios aceptan. Este argumento especifica el número de conjuntos en los que serán agrupados los datos, que será proporcional al
		número de píxeles que tenemos disponibles para representar nuestro gráfico. 
		\par
		En análisis técnico bursátil los valores \emph{open} y \emph{close} tienen su significado propio, en este trabajo representan la media
		más la desviación típica y la media menos la desviación típica respectivamente. Como podemos ver en la figura \ref{fig:candlestick} el
		\emph{open} y \emph{close} son utilizados para construir el cuerpo de las velas. Este representa los datos que se encuentran dentro de
		la desviación típica. La línea que cruza el cuerpo de la vela representa el valor máximo y mínimo para cada grupo.
		\par
		Las velas con cuerpos alargados indican grupos en los que la desviación típica es muy grande, por contrario velas con cuerpos
		contraídos indican grupos con una desviación típica pequeña. Los dos extremos, datos que fluctúan mucho o datos que no fluctúan,
		pueden ser indicativo de un mal funcionamiento. Los valores máximo o mínimo que se alejan mucho del cuerpo de la vela nos indican la
		presencia de un Spike en dicho grupo. 	
	\subsection{Ampliando HighStock}
		\emph{HighStock} es construido de forma modular para facilitar la ampliación por parte de los desarrolladores. En este trabajo hemos ampliado
		el framework con dos funcionalidades adicionales. El propósito de esta sección es describir estas dos funcionalidades, pero primero
		explicaremos el proceso para ampliar el framework.
		\par
		La primera consideración que tenemos que tener es el ámbito o \emph{scope}, nombre que utilizaremos a lo largo de este trabajo.
		Debemos tener cuidado de no contaminar el \emph{scope} global con variables. Para este propósito debemos utilizar una función
		\emph{self-invoking}. En JavaScript esta es una función que se ejecuta inmediatamente y crea su propio \emph{scope}, de esta manera
		evitamos contaminar el \emph{scope} global. A continuación presentamos un ejemplo de como declarar una de estas funciones.
		\begin{lstlisting}[style=myJs]
(function (H) {
   var localVar,	// Variable local
   Series = H.Series;
   doSomething();
}(Highcharts));
		\end{lstlisting}
		\par
		El framework ofrece la función \texttt{wrap} que permite añadir código en una función existente. Podemos añadir dicho código antes o
		después del código inicial de la función. La función acepta tres argumentos, el objeto padre como primer argumento, el nombre de la
		función como segundo argumento y la función de sustitución como tercer argumento. La función original se pasa como el primer
		argumento de la función de sustitución. Es conveniente fijarnos en el ejemplo de código a continuación presentado para entender mejor
		el uso de la función \texttt{wrap}.
		\begin{lstlisting}[style=myJs]
H.wrap(H.Series.prototype, 'drawGraph', function (proceed) {
// Antes de la función original.
console.log("We are about to draw the graph: ", this.graph);
// Invocar la función original usando todos los argumentos menos el primero.
proceed.apply(this, Array.prototype.slice.call(arguments, 1));
// Después de la función original.
console.log("We just finished drawing the graph: ", this.graph);
});
		\end{lstlisting}
		\par  
		A continuación explicaremos las dos ampliaciones que hemos realizado sobre el framework. La primera permite realizar \emph{zoom out} y la
		segunda permite resaltar las series de un gráfico para poder inspeccionarlas mejor.
		\subsubsection{Zoom Out}
			Tal y como explicamos los gráficos creados con el framework son altamente interactivos. Entre las diferentes funcionalidades
			están las que permiten navegar por el conjunto de datos. Con navegar nos referimos a mostrar diferentes fragmentos de los
			datos. Un ejemplo de estas funcionalidades es el \emph{zoom in}, que permite ampliar una sección de los datos. Haciendo \emph{click}
			y arrastrando, sin importar la dirrección, podemos seleccionar un intervalo para realizar el \emph{zoom in}. Esto es muy útil para
			inspeccionar los datos, el problema esta en que el framework no ofrece la posibilidad de realizar un \emph{zoom out}.
			\par
			En este trabajo hemos extendido el framework para incorporar esta funcionalidad. Para este propósito utilizamos la dirección
			de arrastre al realizar una selección. Un arrastre de izquierda a derecha realiza un \emph{zoom in}, mientras que un arrastre de
			derecha a izquierda realiza un \emph{zoom out}. 
			\par
			Para poder registrar la dirección de arrastre hemos hecho uso de dos eventos, \texttt{mousedown} y \texttt{mouseup}. En cada
			evento registramos la posición de este y usamos la diferencia entre las posiciones para determinar la dirección de arrastre. A
			continuación presentamos el código. Es la variable \texttt{selectDirection} la que indica la dirección de arrastre.
			\begin{lstlisting}[style=myJs]
H.addEvent(container, 'mousedown', function (e) {
   selectFrom = chart.pointer.normalize(e).chartX;
});

H.addEvent(container, 'mouseup', function (e) {
   selectTo = chart.pointer.normalize(e).chartX;
   selectDirection = selectTo - selectFrom;													            
});
			\end{lstlisting}
			\par
			El evento que se dispara al realizar una selección es \texttt{selection}, por lo que tenemos que registrar un manejador para
			este. La primera acción es comprobar el estado del \emph{selectDirection}, la variable que indica la dirección de arrastre. En
			el caso de que la variable tenga un valor positivo no tomamos acción, dejamos que se realice la accion por defecto que es
			\emph{zoom in}. Si por contrario la variable tiene un valor negativo tenemos que prevenir la acción por defecto. Seguidamente
			calculamos los valores que definen el nuevo intervalo temporal y, para terminar aplicamos el nuevo intervalo al gráfico. A
			continuación se puede ver el código que implementa la funcionalidad previamente descrita.
			\begin{lstlisting}[style=myJs]
H.addEvent(chart, 'selection', function (e) {
   if (selectDirection < 0) {
      e.preventDefault();
      //Calcular newMin y newMax
      xAxis.setExtremes(newMin, newMax);
   }
});
			\end{lstlisting}
		\subsubsection{Resaltar series}
			En la mayoría de casos los monitores de neutrones se componen por 18 tubos contadores. Los gráficos que representan la
			información de los tubos por separado se vuelven confusos debido al gran número de series. Este problema crea la necesidad de
			tener un mecanismo que permita resaltar una serie del gráfico para poder inspeccionarla mejor. Para crear dicho mecanismo
			utilizaremos la leyenda del gráfico.
			\par
			\emph{HighStock} permite añadir una leyenda a los gráficos. Esta leyenda consiste en un elemento gráfico que facilita el identificar
			e interpretar las diferentes series del gráfico. La leyenda, al igual que los gráficos, es altamente interactiva. Haciendo
			\emph{click} sobre los elementos de una leyenda podemos mostrar u ocultar la serie correspondiente. En este trabajo
			preservamos esta funcionalidad y añadimos la funcionalidad de resaltar una serie. Para este propósito utilizamos los eventos
			\texttt{mouseover} y \texttt{mouseout}, de forma que la serie se resalta al posicionar el puntero sobre el elemento
			correspondiente de la leyenda. 
			\par
			El proceso de resaltar una serie para poder visualizarla mejor consiste en bajar la opacidad de todas las demás series, de
			esta manera la serie que queremos resaltar es más visible. En la figura \ref{fig:resalto} podemos ver el resultado que
			obtenemos. A la izquierda podemos ver el gráfico antes de resaltar una serie, vemos que las mediciones de los 18 canales se
			solapan y es difícil examinarlas. A la derecha podemos ver el canal 13, \emph{ch13}, resaltado.
			\begin{figure}[h]
				\centering
				\includegraphics[keepaspectratio, width=1\textwidth]{./img/resalto.png}
				\caption{Ampliando \emph{HighStock}. Resaltar series.}   
				\label{fig:resalto}
			\end{figure}
	
\section{Modelo}
	El \emph{modelo} es el encargado de manejar los datos de una aplicación. En el caso del \emph{front-end} los datos de la aplicación deben ser
	servidos por el \emph{back-end}. El \emph{modelo} es el encargado de realizar la comunicación con el \emph{back-end}. 
	\par
	Tal y como especificamos en el  capítulo anterior el protocolo para la comunicación entre los dos módulos es \emph{HTTP}. Nuestro
	\emph{front-end} es el que empieza la comunicación enviando un mensaje de petición y el \emph{back-end} responde a esa peticion con un mensaje
	de respuesta. El \emph{modelo} es el encargado de enviar los mensajes de petición y después interpretar los mensajes de respuesta.
	\par
	Para dotar al \emph{modelo} de la funcionalidad necesaria utilizamos las facilidades que \emph{ExtJs} ofrece, concretamente utilizamos el singleton
	\texttt{Ext.Ajax}. \emph{Ajax}\cite{AjaxWiki} es una técnica de desarrollo Web donde cliente y servidor mantienen una comunicación asíncrona en
	segundo plano. \texttt{Ext.Ajax} es un singleton de la clase \texttt{Ext.data.Connection}, clase que encapsula la lógica necesaria para
	realizar una comunicación \emph{Ajax}. 
	\par
	Concretamente hacemos uso de la función \texttt{Ext.data.Connection.request}. Esta función envía una petición \emph{HTTP} a un servidor
	remoto. La función acepta únicamente un parámetro que es un objeto cuyas propiedades definen el comportamiento de la función. Las propiedades de las
	que nosotros hacemos uso están detalladas a continuación. 
	\begin{itemize}
		\item	\texttt{url}. La URL a la que enviaremos la petición. En nuestro caso \emph{back-end} y \emph{front-end} estarán albergados en
			el mismo \emph{host}. Esto nos permite no especificar el \emph{host}, la petición se hará al \emph{host} utilizado para cargar
			la aplicación. En la URL solamente hay que especificar el servicio deseado y los parámetros que acompañan a este.
			Seguidamente presentamos un ejemplo del valor que puede tomar el campo \texttt{url}.
    				\begin{center} \texttt{url: \cc/nmdadb/channel/stats/default/default\cc}  \end{center}
		\item	\texttt{method}. El método que especifica nuestro mensaje de petición. Los mensajes \emph{HTTP} de petición pueden especificar
			un método. Si este campo se deja vacío el método utilizado por defecto es \texttt{GET}. La mayoría de los servicios ofrecidos
			por el \emph{back-end} aceptan un método \texttt{GET}, pero el servicio \texttt{nmdbMarkNull} acepta un método \texttt{POST}.
		\item	\texttt{success}. La función a ser invocada al completar con éxito la petición. Esta función a su vez acepta como parámetro
		  	\texttt{response}. Este parámetro contiene los datos del mensaje de respuesta.
		\item	\texttt{failure}. La función a ser invocada al completar sin éxito la petición. Esta función también acepta el parámetro
		  	\texttt{response} que podemos utilizar para identificar la causa del fallo.
		\item	\texttt{timeout}. El número de milisegundos en los que el \emph{back-end} debe responder. Si el tiempo expira la solicitud se
		  	considera como fallida. 
	\end{itemize}
	Más allá de \texttt{Ext.Ajax} el framework ofrece abstracciones de un nivel superior. La clase \texttt{Ext.data.Model} es una representación
	de un objeto utilizado por nuestra aplicación. Estos modelos son usados por la clase \texttt{Ext.data.Store}, que encapsula instancias de
	estos. Estas abstracciones son muy útiles, pero algo complejas. Por esta razón la mayoría de los datos se manejan usando el
	\texttt{Ext.data.Connection}, mientras que hemos utilizado el \texttt{Ext.data.Store} en casos muy específicos como los datos que necesitan
	ser mostrados en una tabla. Las tablas en \emph{ExtJs} deben tener asociado un \texttt{Ext.data.Store} cuyos datos mostrar. A continuación podemos
	ver un ejemplo de como declarar un \texttt{Ext.data.Model} y un \texttt{Ext.data.Store} que enlazaremos a un \texttt{Ext.grid.Panel}.
	\begin{lstlisting}[style=myJs]
Ext.define('MyApp.model.MyModel',{
   extend: 'Ext.data.Model',
   fields: [	{name: 'myName1', type: 'string'},
   		{name: 'myName2', type: 'string'}]
});

Ext.define('MyApp.store.MyStore',{
   extend: 'Ext.data.Store',
   model: 'MyApp.model.MyModel',
   proxy: {
      type: 'ajax',
      url: '/nmdadb/channel/stats/default/default',
      reader: {
         type: 'json'}}
});
var myStore = Ext.create('MyApp.store.MyStore');
myStore.load();

Ext.define('MyApp.grid.MyGrid',{
   extend: 'Ext.grid.Panel',
   title: 'MyCoolGrid',
   store: myStore,
   fields: [	{text: 'Name1', dataIndex: 'myName1'},
		{text: 'Name2', dataIndex: 'myName2'}]
});
Ext.create('MyApp.grid.MyGrid', {
   renderTo: Ext.getBody()
});
	\end{lstlisting}
	\par
	Podemos ver que en la declaración del \texttt{Ext.data.Store} especificamos \texttt{proxy}. El objeto especificado se utiliza para crear una
	instancia de la clase \texttt{Ext.data.proxy.Proxy}. Esta clase es una abstracción que agrupa diferentes técnicas de comunicación como
	\emph{Ajax}, \emph{JsonP}, \emph{Rest} y \emph{Direct}.

\section{Controlador}
	El Controlador en una aplicación \emph{modelo-vista-controlador}\cite{MVCWiki} hace de intermediario entre la Vista y el Modelo. Normalmente
	este responde a los eventos producidos por el usuario. Las respuestas consisten en peticiones al Modelo para cargar datos o bien en comandos
	dirigidos a la Vista para realizar un cambio en la forma de mostrar esos datos. El Controlador contiene toda la lógica de la aplicación por lo
	que podemos decir que es el módulo más interesante. El propósito de esta sección es hacer una descripción de este.
	\par
	\emph{Sencha ExtJs} ofrece la clase \texttt{Ext.app.Controller}, que representa la abstracción de un Controlador. Esta clase contiene la
	funcionalidad mínima necesaria, está pensada para ser extendida y dotada de funcionalidad por el usuario. A continuación podemos ver un
	pequeño ejemplo de como declarar un Controlador básico en \emph{Sencha ExtJs}.
	\begin{lstlisting}[style=myJs]
Ext.define('MyApp.controller.Users', {
   extend: 'Ext.app.Controller',
   onLaunch: function() {
      console.log('Podemos empezar..');}
});
	\end{lstlisting}
	\par
	La función \texttt{onLaunch()} es una función especial que es invocada después de inicializar la aplicación. La Vista y Modelo ya están
	inicializados, por lo que es un buen punto para empezar a actuar sobre estos elementos.
	\par
	Antes de empezar a discutir los Controladores que hemos implementado en este trabajo es conveniente destacar que el framework ofrece una
	facilidad similar al \texttt{Ext.ComponentQuery}, pero para los Controladores. Podemos hacer consultas para acceder a los Controladores desde
	cualquier punto del código. A continuación podemos ver como hacer uso de esta facilidad.
    		\begin{center} \texttt{MyApp.app.getController("Nombre\_del\_Controlador")}  \end{center}
	\par
	En nuestra aplicación hemos definido cinco Controladores, que se describirán en las las subsecciones venideras.
	\subsection{\texttt{HighStockExtend}}
		El propósito de este Controlador es inicializar las dos extensiones de \emph{HighStock} que discutimos previamente en este capítulo. Este
		Controlador se invoca antes de empezar a construir los gráficos de la aplicación para que las extensiones estén listas y sean
		aplicadas a estos.
	\subsection{\texttt{Navigation}}
		Este Controlador es el encargado de inicializar el \texttt{Ext.History},que es el mecanismo que soluciona el problema de navegación por el
		historial en aplicaciones \emph{single-page}. El problema y el mecanismo fueron descritos en secciones anteriores de este capítulo.
		\par
		El Controlador define la función \texttt{onLaunch()}, que se invoca una vez creados la Vista y el Modelo de la aplicación.
		Empieza invocando la función \texttt{Ext.History.init()}, a fin de inicializar este módulo. Seguidamente define la
		función encargada de manejar el evento \texttt{change}, que se dispara al cambiar el \emph{hash} de la URL. Esta función soporta tres
		valores diferentes como \emph{hash} que son \texttt{[Spike, SpikeRevised, ChannelStats]}. Estos tres valores representan los tres
		módulos funcionales entre los que hacemos distinción. Cambiando el \emph{hash} cambiamos el módulo funcional mostrado en
		pantalla.
		\par
		Podemos ver que el soporte de navegación por el historial es pobre en funcionalidad, por esta razón proponemos como trabajo futuro
		extender y ampliar la funcionalidad de este.
		\par
		También debemos tener en cuenta que al inicializarse la aplicación puede haber un \emph{hash} presente. Una vez especificado el
		comportamiento ante un evento \texttt{change} procedemos a evaluar el \emph{hash} actual. Este puede no estar presente, caso en el que
		cargamos el módulo funcional \texttt{Spike}. En el caso de la presencia de un \emph{hash} actuamos conforme a este.
	\subsection{\texttt{Spike}}
		Este Controlador es directamente relacionado al módulo funcional con el mismo nombre, \texttt{Spike}. Este Controlador es algo más
		complicado que los anteriores, dado que se compone de múltiples funciones. A continuación procedemos a explicar estas
		funciones.
		\begin{description}[style=unboxed,leftmargin=0cm]
			\item[\texttt{Launch()}] Inicializa el módulo funcional \texttt{Spike}. Es conveniente destacar que esta función no es
			  \texttt{onLauch}, y por lo tanto no es invocada automáticamente al inicio de la aplicación, sino que somos nosotros los que
			  la invocamos. Esta función inicializa el objeto \texttt{app}, que debe contener todas las variables usadas en este
			  módulo. La mayoría de las demás funciones usaran este objeto para acceder a estas variables. Después de inicializar este
			  objeto se invoca \texttt{loadInitialData()}.
			\item[\texttt{loadInitialData()}] Esta función carga los datos necesarios para construir el módulo. Los datos son cargados
			  realizando una petición \emph{Ajax} al \emph{back-end}. El servicio RPC que invocamos es \texttt{nmdbOriginalGroup}. Al cargar los
			  datos con éxito se invocann a su vez las funciones \texttt{initCandleChart()} y \texttt{initChannelChart()}. 
			\item[\texttt{initCandleChart()}] Esta función es la encargada de crear el gráfico \emph{Candlestick} con los datos de la
			  estación. Hablaremos más a fondo sobre este gráfico en las secciones futuras donde discutiremos la herramienta desde un
			  enfoque funcional.
			\item[\texttt{initChannelChart()}] Esta función es la encargada de crear el gráfico que representa los datos de los tubos
			  contadores por separado. Igual que en el caso anterior hablaremos más sobre este gráfico en secciones futuras.
			\item[\texttt{LineOrOhcl(start, finish, N\_points)}] Anteriormente en este capítulo discutimos el problema que surge al
			  intentar representar un gran número de datos en un gráfico convencional y que el gráfico \emph{Candlestick} da solución a
			  este problema. Esta función determina si debemos utilizar un gráfico de línea o bien uno \emph{Candlestick}. La cantidad de
			  datos se define por los parámetros \texttt{start} y \texttt{finish}, mientras que el parámetro \texttt{N\_points} representa
			  el número de píxeles que tenemos a disposición para representar los datos.
			\item[\texttt{updateMode(start, finish)}] Esta función es invocada cuando ocurre algún cambio en el gráfico \emph{Candlestick}
			  y calcula el modo en que deben ser representados los datos. Esta función hace un uso intensivo de la función
			  \texttt{LineOrOhcl()}, pero tiene en cuenta más cosas como la serie actual.
			\item[\texttt{changeSeries(series)}] Esta función cambia la serie que es presentada en el gráfico \emph{Candlestick}. El
			  parámetro \texttt{series} es un entero que puede tomar los siguientes valores \texttt{[1, 2, 3]}, donde estos representan
			  \emph{uncorrected}, \emph{corrected for pressure} y \emph{corrected for efficiency}.
			\item[\texttt{updateSeries()}] Esta función simplemente refresca los datos de las series del gráfico \emph{CandleStick}, y se
			  invoca cada vez que ocurre algún cambio en este para refrescar los datos.
			\item[\texttt{updateCandleData(start, finish)}] Esta función carga los datos para el gráfico \emph{Candlestick} en función de
			  la petición realizada por el usuario. Para cargar los datos la función realiza una petición \emph{Ajax} al \emph{back-end}. La
			  función invoca el servicio RPC \texttt{nmdbOriginalGroup} o \texttt{nmdbOriginalRaw} en función del valor devuelto por la
			  función \texttt{LineOrOhcl()}. El intervalo de datos que será solicitado se define por los parámetros \texttt{start} y
			  \texttt{finish}. 
			\item[\texttt{updateChannelData()}] Esta función carga los datos para el gráfico que representa los datos de los tubos
			  contadores por separado. Estos datos se descargan desde el \emph{back-end} mediante una petición \emph{Ajax} que invoca el servicio
			  RPC \texttt{nmdadbRawData}.
			\item[\texttt{searchInterval(start, finish)}] Esta función permite hacer una búsqueda por intervalos temporales. El intervalo
			  es definido por los parámetros \texttt{start} y \texttt{finish}.
			\item[\texttt{getTimestamp(str)}] Esta función acepta una cadena de texto que debe seguir un formato determinado y devuelve un
			  objeto de la clase \texttt{Date}. 
			\item[\texttt{showCandle()}] Esta función muestra el gráfico \texttt{Candlestick} creado por este Controlador. Junto al
			  gráfico también se asegura de mostrar los controles vinculados a este. Si es la primera vez que intentamos mostrar este
			  gráfico el módulo entero no estará creado, por lo que es invocada la función \texttt{Launch()}.
			\item[\texttt{showChannel()}] Esta función muestra el gráfico de los tubos contadores por separado. Junto al gráfico también
			  se asegura de mostrar los controles vinculados a este. Es invocada la función \texttt{updateChannelData()} para asegurar que
			  son mostrados los datos correctos.
		\end{description}
	\subsection{\texttt{SpikeRevised}}
		Este Controlador es muy similar al anteriormente descrito, \texttt{Spike}. Las diferencias como indica el nombre radican en que este
		utiliza los datos revisados, no los originales. Además de utilizar un conjunto de datos distinto este Controlador también ofrece
		funcionalidad extendida, permite marcar datos como inválidos. Los datos marcados como inválidos serán considerados nulos en el
		conjunto de datos revisados.
		\par
		A continuación procederemos a explicar las funciones que este Controlador define.
		\begin{description}[style=unboxed,leftmargin=0cm]
			\item[\texttt{Launch()}]
				Inicializa el módulo funcional \texttt{SpikeRevised}. Esta función inicializa el objeto \texttt{app} que debe contener
				todas las variables asociadas a este módulo. La estructura de este objeto es idéntica al objeto creado en la función
				\texttt{Launch()} del Controlador \texttt{Spike} a excepción de una serie de atributos extra. 
				\par
				Seguidamente se importan todas las funciones que podemos reutilizar del Controlador \texttt{Spike}, de esta forma
				se evita duplicar código idéntico. Las funciones importadas son las siguientes:
					\begin{center} \texttt{	initCandleChart, initChannelChart, changeSeries, searchInterval, updateChannelData,
					  			LineOrOhcl, updateMode, updateSeries, getTimestamp}
					\end{center}
				Finalmente se llama a la función \texttt{loadInitialData()}.
		    	\item[\texttt{loadInitialData()}]
				Esta función es muy parecida a la función con el mismo nombre del Controlador \texttt{Spike}. La función carga los
				datos necesarios para construir el módulo. La diferencia radica en el servicio RCP utilizado. Esta función utiliza el
				servicio \texttt{nmdbRevisedGroup}, tal y como explicamos este módulo utiliza los datos revisados.
				\par
				Al cargar los datos con éxito son invocadas las funciones \texttt{initCandleChart} y \texttt{initChannelChart} que
				crean los dos gráficos, estas son función importadas del Controlador \texttt{Spike}. Finalmente se invoca la función
				\texttt{extendChandleChart} que se explicará a continuación.
			\item[\texttt{extendCandleChart()}]
				Volvemos a recordar que este módulo es muy similar al \texttt{Spike}, pero ofrece alguna funcionalidad extra. Esta
				función es la encargada de extender la funcionalidad del gráfico \emph{Candlestick}. La función define el
				comportamiento de las series del gráfico ante el evento \texttt{click}. Este comportamiento consiste en marcar el dato
				con un flag y añadirlo a una tabla. Posteriormente los datos en la tabla podrán ser clasificados como inválidos. 
			\item[\texttt{updateCandleData(start, finish)}]
				Esta función es muy parecida a la función con el mismo nombre del Controlador \texttt{Spike}. La función carga los
				datos para el gráfico Candlestick en función de la petición realizada por el usuario. La diferencia radica en los
				servicios RPC utilizados. Esta función utiliza lo servicios \texttt{nmdbRevisedGroup} o \texttt{nmdbRevisedRaw} en
				función del valor devuelto por la función \texttt{LineOrOhcl()}.El intervalo de datos que es solicitado es definido
				por los parámetros \texttt{start} y \texttt{finish}.
			\item[\texttt{removeFromGrid(rowIndex)}]
				Los datos marcados son guardados en una tabla. Esta función permite eliminar un dato de esta tabla. El parámetro que
				esta función acepta especifica el índice del dato que será eliminado de la tabla. 
			\item[\texttt{submitGrid()}]
				Esta función permite clasificar como inválidos los datos en la tabla. La función invoca el servicio
				\texttt{nmdbMarkNull} del \emph{back-end} a fin de cumplir con su propósito. Una vez anulados correctamente los datos,
				la tabla se vacía y el gráfico se refresca con el fin de mostrar los cambios.
			\item[\texttt{showHideGrid()}]
				Esta función permite mostrar u ocultar la tabla que contiene los datos marcados, de esta manera los datos pueden ser
				inspeccionados.
			\item[\texttt{clearGrid()}]
				Esta función permite vaciar la tabla que contiene los datos marcados, de esta manera estos son descartados y no se
				toma ninguna acción.
			\item[\texttt{showCandle()}]
				Esta función es muy parecida a la función con el mismo nombre del Controlador \texttt{Spike}. La función muestra el
				gráfico \emph{Candlestick} y todos los controles vinculados a este. La función \texttt{Launch()} es invocada en el
				caso de que el módulo no esté creado.
			\item[\texttt{showChannel()}]
				Esta función es muy parecida a la función con el mismo nombre del Controlador \texttt{Spike}. La función muestra el
				gráfico de los tubos contadores por separado. Junto al gráfico también se asegura de mostrar los controles vinculados
				a este. Es invocada la función \texttt{updateChannelData()} para asegurar que son mostrados los datos correctos.
		\end{description}
	\subsection{\texttt{ChannelStats}}
		Este Controlador está directamente relacionado con el módulo funcional con el mismo nombre, \texttt{ChannelStats}. Este Controlador se
		compone de varias funciones que discutiremos a continuación.
		\begin{description}[style=unboxed,leftmargin=0cm]
			\item[\texttt{Launch()}]
				Inicializa el módulo funcional \texttt{ChannelStats}. Esta función inicializa el objeto \texttt{app}, que debe
				contener todas las variable usadas en este módulo. Después de iniciar este objeto se invoca la función
				\texttt{loadInitialData()}.
			\item[\texttt{loadInitialData()}]
				Esta función carga los datos necesarios para construir el módulo. Los datos son cargados realizando una petición
				\emph{Ajax} al \emph{back-end}. Se invocan dos servicios RPC, \texttt{nmdadbChannelStats} y
				\texttt{nmdadbChannelHistogram}. El primero devuelve una serie de estadísticas descriptivas de los canales. Estos
				datos se representan en un gráfico y también se muestran en una tabla. El segundo carga los datos necesarios para
				construir un histograma de la distribución de cuentas para los 18 canales.
				\par
				Al cargar los datos con éxito se invocan las funciones \texttt{initStatsChart()} y \texttt{initHistogramChart()}
				que  crean los dos gráficos: el gráfico con las estadísticas y el histograma. Los datos de la tabla son actualizados
				automáticamente, por lo que no tenemos que tomar ninguna acción.
			\item[\texttt{initStatsChart()}]
				Esta función es la encargada de crear el gráfico a partir de los datos devueltos por el serviocio RPC
				\texttt{nmdadbChannelStats}. 
			\item[\texttt{initChannelHistogram()}]
				Esta función es la encargada de crear el histograma que representa la distribución de cuentas para los 18 canales de
				la estación.
			\item[\texttt{searchInterval()}]
				La función que carga los datos iniciales carga los datos del último mes, esta función permite solicitar otro intervalo
				temporal. Este intervalo se define por los dos parámetros que esta función acepta, \texttt{start} y \texttt{finish}.
				La función invoca los servicios RPC \texttt{nmdadbChannelStats} y \texttt{nmdadbChannelHistogram} para solicitar los
				datos al \emph{back-end}.
			\item[\texttt{showModule()}]
				Esta función muestra en pantalla este módulo funcional. En el caso de que este módulo no esté creado se invoca la
				función \texttt{Launch()}.
		\end{description}
	\subsection{Visión general}
		En la figura \ref{fig:controllers} podemos ver un diagrama que representa los Controladores que acabamos de explicar. En color azul
		podemos ver las funciones, gris en caso de que sean funciones importadas desde otro Controlador. Las flecha azules que unen las
		funciones muestras como estas se invocan las unas a las otras. Algunas funciones son invocadas al producirse un cierto evento, estos
		pueden verse en verde dentro de las funciones. Todos los eventos son producidos por los elementos que forman la Vista de la
		aplicación, a excepción del evento \texttt{Ext.History.change} que se dispara al realizar un cambio en la pila del historial de
		navegación.
		\begin{figure}[h]
			\centering
			\includegraphics[keepaspectratio, width=1\textwidth]{./img/controllers.png}
			\caption{Visión general de los Controladores.}   
			\label{fig:controllers}
		\end{figure}
\section{Vista}
	El componente de Vista en una aplicación \emph{modelo-vista-controlador}\cite{MVCWiki} es el responsable de presentar la información de la
	aplicación junto a todos los componentes que componen la interfaz de usuario. El propósito de esta sección es describir los aspectos básicos
	de este componente.
	\begin{figure}[h]
		\centering
		\includegraphics[keepaspectratio, width=1\textwidth]{./img/vista.png}
		\caption{\emph{Front-end}. Vista.}   
		\label{fig:vista}
	\end{figure}
	\par
	La Vista en una aplicación \emph{ExtJs} se compone de una jerarquía de \emph{componentes}. En nuestro caso en lo más alto de esta jerarquía
	tenemos un \texttt{ViewPort} al que hemos asignado el nombre \texttt{MyViewPort}. Este es un \emph{componente} que se ajusta al área de
	aplicación disponible. El layout que determina como se posicionan los hijos de este \emph{componente} es \texttt{border}. Los
	\emph{componentes} hijos deben especificar el atributo \texttt{region} que determina su posicionamiento.
	\par
	El primer hijo con \texttt{region:north} es un \texttt{Ext.container.Container}, que especifica un \emph{contenedor} base. Actualmente este
	\emph{contenedor} no tiene ninguna funcionalidad. En un futuro se plantea visualizar en él mensajes de alerta.
	\par
	El segundo hijo del \texttt{MyViewPort} con \texttt{region:west} es también una instancia de la clase \texttt{Ext.container.Container}. El
	\texttt{itemId} que define es \texttt{Controls}. Este \emph{contenedor} tiene tres hijos que son instancias de la misma clase. Estos tres
	hijos contienen los controles vinculados a los tres módulos funcionales. Los \texttt{itemId} que estos tres hijos especifican son:
    		\begin{center} \texttt{[SpikeControls, SpikeRevisedControls, ChannelStatsControls]}  \end{center}
	\texttt{Controls} usa el layout \texttt{accordion}, donde solamente uno de los hijos puede estar expandido y los demás están
	colapsados, de esta manera no pueden ser visibles lo controles de dos módulos funcionales a la vez. Cada vez que se expanden los
	controles de un módulo funcional se dispara el evento \texttt{Ext.panel.Panel.beforeexpand}. El handler asociado a este evento
	invoca la función apropiada de los Controladores para mostrar el módulo funcional al que pertenecen los controles que acaban de ser
	expandidos.
	\par
	La tercer hijo del \texttt{MyViewPort} con \texttt{region:center} es también una instancia de la clase \texttt{Ext.container.Container}. El
	\texttt{ítemId} que identifica a este componente es \texttt{NavigationPanel}. El layout utilizado es \texttt{card} donde los hijos se solapan
	unos a otros y tan sólo uno puede ser visible a la vez. Los \emph{componentes} hijos contienen todos los gráficos y tablas que nuestra
	aplicación ofrece. Estos se muestran y ocultan en función del estado de los controles.
	\par
	En la figura \ref{fig:vista} podemos ver como se organizan los componentes de la Vista que acabamos de explicar.
\section{Descripción funcional}
	El propósito de esta sección es hacer una descripción funcional de la herramienta. Diferenciamos entre tres módulos funcionales, que son
	\texttt{Spike}, \texttt{SpikeRevised} y \texttt{ChannelStats}.
	\subsection{\texttt{Spike}}
		El propósito de este módulo es ofrecer un gráfico interactivo que permite localizar los \emph{spikes} de forma fácil. Recordamos
		que estos son datos anormalmente grandes o pequeños. La figura \ref{fig:spikeRevised} presenta el módulo funcional
		\texttt{SpikeRevised}. Dado que los dos módulos son muy similares podemos fijarnos en esta figura.
		\par
		Inicialmente el módulo tiene que representar todos los datos de la estación, razón por la que se utiliza el gráfico
		\emph{Candlestick}. En este gráfico los valores máximo y mínimo para cada grupo son fáciles de distinguir. Eventualmente cuando se
		solicita  un intervalo temporal para el que no es necesario agrupar los datos, el gráfico automáticamente cambia a un gráfico de
		línea. Por contrario si se vuelve a solicitar un intervalo que requiere agrupar los datos, el gráfico cambia a \emph{Candlestick}.
		\par
		Para solicitar un intervalo diferente de datos tenemos varias formas. La primera es el \emph{zoom} interactivo que podemos realizar
		haciendo \emph{click} y arrastrando sobre un intervalo del gráfico. Dependiendo de la dirección de arrastre este será un \emph{zoom
		in} u \emph{out}. La segunda alternativa es utilizar los campos de entrada disponibles en los controles para realizar una búsqueda por
		intervalo.  En estos campos podemos especificar el inicio y fin del intervalo y la búsqueda será realizada al accionar el botón
		\texttt{Search}. La tercera opción es utilizar el navegador que podemos encontrar en la parte inferior del gráfico. Finalmente podemos
		utilizar los botones en la parte superior izquierda del gráfico, \texttt{1m} y \texttt{All}, que permiten fijar un intervalo de un mes
		o volver a mostrar todos los datos.
		\par
		Inicialmente se muestran los datos \emph{sin corregir}, utilizando el botón \texttt{Choise a Series} podemos mostrar los datos
		\emph{corregidos por presión} o los \emph{corregidos por eficiencia}. En la parte superior del gráfico se representan los valores
		de la presión atmosférica, que permiten relacionar los datos respecto a esta magnitud.
		\par
		Cuando no es necesario agrupar los datos el botón \texttt{Channel Chart} es habilitado. Este botón muestra un gráfico secundario con
		los valores de los tubos contadores por separado. El intervalo de datos es marcado por el intervalo del gráfico principal. Este
		gráfico permite observar el comportamiento de los diferentes tubos. El botón \texttt{Candle Chart} permite volver al gŕafico
		principal.
	\subsection{\texttt{SpikeRevised}}
		Este módulo incorpora toda la funcionalidad del anterior, la diferencia radica en que este utiliza un conjunto de
		datos diferente y además ofrece alguna funcionalidad extra. \texttt{Spike} utiliza los datos originales, mientras que este módulo
		utiliza los datos revisados. La funcionalidad extra de este módulo es la que permite añadir datos al conjunto de datos revisados.
		\par
		Para satisfacer dicha funcionalidad este módulo ofrece una tabla. Esta tabla se muestra en una ventana separada que puede ser mostrada
		u ocultada utilizando el botón \texttt{Selected Points}. Los datos de esta tabla pueden ser enviados al conjunto de datos revisados
		utilizando el botón \texttt{Submit}, que está presente en la misma ventana. Los datos añadidos a este conjunto serán considerados como
		nulos en el futuro. Este botón además de enviar los datos, actualiza el gráfico para reflejar los cambios que acaban de producirse.
		\par
		Cuando los datos son representados en modo línea podemos hacer \emph{click} sobre un dato para añadir este a la tabla anteriormente
		descrita. En el gráfico se añade un \emph{flag}, la letra que aparece en dicho \emph{flag} se corresponde con el campo
		\texttt{Label} de la tabla.
		\begin{figure}[h]
			\centering
			\includegraphics[keepaspectratio, width=1\textwidth]{./img/spikeRevised.png}
			\caption{Descripción funcional. SpikeRevised}   
			\label{fig:spikeRevised}
		\end{figure}
	\subsection{\texttt{ChannelStats}}
		El propósito de este módulo funcional es ofrecer información de los diferentes canales por separado, de esta manera seremos capaces de
		identificar el mal funcionamiento en canales aislados. El módulo ofrece una tabla en la que podemos inspeccionar la media, desviación
		típica, máximo, mínimo y último valor para cada canal. Estos datos también son representados en un gráfico \emph{Candlestick}.
		\par
		Junto a estos datos el módulo ofrece un histograma con la distribución de cuentas para los diferentes canales. Dicho histograma puede
		consultarse en una pestaña separada.
		\par
		Inicialmente los datos y el histograma son calculados para el último mes. En los controles podemos encontrar los campos de entrada que
		permiten solicitar un intervalo de tiempo determinado. La búsqueda por intervalo está limitada a un mes dado a que las consultas
		necesarias para obtener estos datos no son muy efficientes, estas agrupan los datos por campos que no están indexados. Más sobre este
		problema podemos encontrar en el capítulo dedicado al \emph{back-end}.

\chapter{Conclusiones y trabajos futuros}
\label{capx}

\section{Conclusiones}
\section{Trabajos futuros}
	\subsection{Debian}
	\subsection{Script configuración}
	\subsection{WatchDog pro version}



%Incluyo la bibliografía
\addcontentsline{toc}{chapter}{Referencias bibliográficas}
%\let\antiguaLM\leftmark
%\let\antiguaRM\rightmark
%\renewcommand\leftmark{}
\nocite{*}
\bibliographystyle{plain}
\bibliography{Bibliografia}

%Acronyms
\newpage
\printglossary[type=\acronymtype]
\addcontentsline{toc}{chapter}{Acrónimos}
\end{document}
