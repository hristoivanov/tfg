\chapter{Herramienta Web. Back End}
\label{backend}

Fijándonos en el diseño preliminar de la figura \ref{fig:herramienta_web_preliminar} podemos ver que nuestra aplicación Web está dividida en
\emph{front-end} y \emph{back-end}. Esta separación entre módulos es una técnica popular en diseño software. El \emph{front-end} es el encargado de
la capa de presentación, sobre este hablaremos más en el próximo capítulo. En este capítulo nos centraremos en explicar el \emph{back-end}. Este es el
encargado de procesar las peticiones provenientes del \emph{front-end} y devolver le a este la información solicitada. 
\par
El \emph{back-end} será implementado en PHP\cite{PHP}. Este es un lenguaje diseñado para desarrollo web y además es una elección muy popular. 
Utilizaremos ZendFramework\cite{ZF}, este es un framework orientado al desarrollo de aplicaciones web. Junto al framework utilizaremos 
Apigility\cite{Apigility}, herramienta que simplifica la creación y mantenimiento de APIs.
\par
Hemos elegido basarnos en la técnica de arquitectura software REST\cite{Rest}, serie de directrices y métodos para crear aplicaciones Web. Los
aspectos básicos de REST a los que más nos ceñiremos son:
\begin{itemize}
	\item	Cliente Servidor. La aplicación debe seguir el modelo Cliente-Servidor. En nuestro caso estas dos identidades son representadas por el
	  	\emph{back-end} y el \emph{front-end}.
	\item	Sin estado. La comunicación entre Cliente y Servidor no mantiene ningún tipo de sesión. La respuesta del Servidor tan solo depende de
	  	la estimulación proporcionada por el Cliente y no es influenciada por los eventos de comunicación anteriores.
	\item	Conjunto bien definido de recursos y operaciones aplicables a esos recursos. En REST por recurso entendemos elementos de información.
	  	Estos recursos serán los solicitados en las peticiones del Cliente. Estos recursos tienen asociada una URL exclusiva que se usa para
		identificarlos.
\end{itemize}
Si nos volvemos a fijar en la figura \ref{fig:herramienta_web_preliminar} podemos ver que en el \emph{back-end} existe la separación entre
\emph{Modelo} y \emph{Controlador}. Esta separación sigue el patrón de diseño MVC\cite{MVCWiki}. Podemos ver que el componente de \emph{Vista} no
existe, en este caso el \emph{front-end} será nuestro componente de \emph{Vista}.
\par
Volviendo a los recursos REST, cada petición recibida por el \emph{back-end} apuntara a un recurso y especificara una operación sobre este. Con el fin
de responder a las peticiones cada recurso tendrá su parte de \emph{Controlador} y \emph{Modelo}. En este capítulo procederemos a explicar los
servicios REST, pero primero explicaremos algunos aspectos técnicos como la base de datos o el ZendFramework.
\section{Protocolo de comunicación}
  	Hemos hablado que entre el \emph{back-end} y \emph{front-end} existe una comunicación. Siendo esta una herramienta web como es de esperar el
	protocolo de comunicación es HTTP\cite{HTTP}. El \emph{back-end} recibirara \emph{HTTP request} y responderá con \emph{HTTP response}. Las
	peticiones especifican su recurso mediante el URL, al que son anexados los parámetros necesitados. El protocolo ofrece una serie de
	métodos, los más utilizados en este trabajo son \emph{GET} y \emph{POST}. 
	\par
	El método \emph{GET} pide una respuesta del recurso especificado. En función de los parámetros anexados al URL la respuesta cambia.
	\par
	El método \emph{POST} no suele tener parámetros, sin embargo tiene un campo especial. En este cambo se encuentran una serie de datos que serán
	procesados por el \emph{back-end}. Estos datos acabaran creando o actualizando entradas en la base de datos. 
	\par
	Haciendo una pequeña abstracción podemos ver los métodos \emph{GET} y \emph{POST} como \emph{leer y escribir} el recurso especificado.


