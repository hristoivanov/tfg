\chapter{Conclusiones y trabajos futuros}
\label{cap_conclusiones}

\section{Conclusiones}
\section{Trabajos futuros. Software de adquisición}
	\subsection{Debian en la BeagleBone Black}
	Para realizar el software de adquisición hemos utilizado Angstrom, la distribución Linux que venía por defecto con la BeagleBone Black.
	Angstrom es una distribución muy ligera y adecuada, pero la comunidad existente es muy pequeña. Nos hemos propuesto como trabajo futuro
	cambiar de la Angstrom a alguna distribución derivada de Debian. Nuestro caso no es algo aislado sino parte de una movimiento mayor, gran
	parte de los usuarios  de BeagleBone Black instalan Debians en sus placas. El fabricante de las placas también anuncio que en un futuro las
	placas llevaran Debian por defecto. Como hemos comentado la razón detrás de esta elección es la comunidad de usuarios. Siendo esta mayor es
	mucho más fácil encontrar información o ayuda. Al hacer este cambio también tendríamos que realizar algunos cambios en nuestro software,
	adaptarlo para el nuevo entorno.
	\subsection{Script de configuración para el software de adquisición}
	Todo el proceso de configuración para implantar el software está reflejado en el archivo \emph{Readme}. Los pasos recogidos en el archivo
	podrían automatizarse, podríamos crear un script que ejecutara todos los comandos necesarios. De esta manera para implantar el software tan
	solo haría falta clonar el repositorio remoto y ejecutar el script. Este script se crearía después de realizar el cambio de distribución a
	Debian.
	\subsection{Watchdog}
	Otro aspecto que podríamos mejorar del software de adquisición es el Watchdog. Tal y como se explicó en el capítulo \ref{cap2} el Watchdog que esta
	implementado actualmente es muy simple. Este tan solo comprueba que el sistema operativo no se quede bloqueado. Si recordamos la función del
	Watchdog es detectar malfuncionamientos y reiniciar el sistema al detectar uno. Que el sistema operativo se quede bloqueado es un
	malfuncionamiento, pero existe otro que también queremos evitar. La función del software es que los datos generados por el resto del sistema
	acaben en una base de datos, que no se generen datos es un malfuncionamiento que queremos evitar. Podremos detectar este problema mirando en
	la base de datos. Ante la presencia de nuevos datos el software tendría que reiniciar el contador del Watchdog y si no están presentes nuevos
	datos el software tendría que no reiniciar el contador. De esta forma si no se generan nuevos datos por el sistema, el software sera
	reiniciado.
\section{Trabajos futuros. Herramienta Web}
	\subsection{Autenticación y autorización}
	Actualmente la herramienta web no implementa ningún sistema para autenticar y autorizar a los usuarios. Los servicios de consulta no necesitan
	ningún tipo de control, estos pueden ser accesibles por cualquiera. Sin embargo los servicios de acción futuros y presentes tendrían que estar
	controlados. Es por esta razón por la que surge la necesidad de implementar un sistema para gestionar el acceso de los usuarios. 
