\chapter{Conclusiones y trabajos futuros}
\label{cap_conclusiones}


\section{Conclusiones}
%TODO

\section{Trabajos futuros}
	A pesar de estar satisfecho con los resultados conseguidos en este proyecto, este podría desarrollarse más.
	\subsection{Software de adquisición}
		A continuación son detalladas una serie de mejoras y cambios que el software de adquisición podría experimentar a fin de mejorar.
		\begin{description}[style=unboxed,leftmargin=0cm,labelwidth=1cm]
			\item[Debian]
				Para la realización del software de adquisición hemos utilizado Angstrom, la distribución que viene por defecto con la
				BeagleBone Black. Esta es una distribución muy ligera y adecuada, pero la comunidad de usuarios es muy pequeña.
				Proponemos como trabajo futuro cambial la Angstrom a alguna distribución derivada de Debian. Actualmente existen
				distribuciones basadas en Debian que son compatibles con las placas, gran parte de los usuarios instalan estas
				distribuciones en sus placas. El fabricante también ha anunciado que en un futuro las placas serán distribuidas con un
				Debian por defecto. El inconveniente que este cambio presenta, razón por la que no ha sido realizado en este trabajo,
				es que el software existente tendría que sufrir algunos cambios a fin de adaptarse al nuevo entorno.
				\par
				Actualmente el núcleo Linux y todo el sistema de archivos son cargados desde la memoria interna de la placa a la que
				nos referimos con el acrónimo eMMC. Es posible albergar la nueva distribución en la tarjeta microSD. El contenido de
				la tarjeta uSD es fácilmente duplicable, podemos exportar este a un archivo de imagen que contendría una copia exacta.
				Seguidamente podemos grabar este archivo de imagen en una nueva uSD. De esta manera todo el proceso descrito en el
				apéndice \ref{app_soft} queda obsoleto, para desplegar el software de adquisición  tan solo tenemos que grabar un
				archivo de imagen un una tarjeta uSD e introducirla en la ranura de la BeagleBone Black. 
			\item[Adaptación para otras estaciones]
				Este software es desarrollado en el entorno de CALMA. El software funciona para esta estación, pero es muy probable
				que en otras estaciones surjan problemas. Proponemos como trabajo estudiar que cambios tendría que sufrir el software
				o el sistema de adquisición entero para adaptarse a estas estaciones. Además de realizar el estudio sería conveniente
				llevar a cabo dichos cambios e incluso implantar el sistema de adquisición en esas estaciones.   
		\end{description}
	\subsection{Herramienta Web}
		A continuación son detalladas una serie de mejoras y cambios que herramienta Web podría experimentar a fin de mejorar.
		\begin{description}[style=unboxed,leftmargin=0cm,labelwidth=1cm]
			\item[Extender funcionalidad]
				Tal y como explicamos en el capítulo de introducción la idea de la herramienta Web ha sido del equipo de CALMA. Este
				concibe una herramienta muy compleja con múltiples funcionalidades que describimos en el capítulo de introducción. En
				este trabajo tan solo nos hemos centrado en una pequeña parte. Como trabajo futuro se propone extender la
				funcionalidad de la herramienta.
				\par
				El avance de la herramienta tendrá que ser acompañado con el avance del sistema de adquisición. La herramienta nos
				permitirá cambiar la configuración de la estación en tiempo real, para esta meta el software de adquisición tendrá que
				ofrecer una interfaz que lo permita.
			\item[Implantación y mantenimiento]
				En el capítulo de introducción también especificamos que en este trabajo tan solo nos centraremos en el proceso de
				implementación. Como trabajo futuro se propone implantar y mantener el software.
			\item[Autenticación y autorización]
				Actualmente la herramienta Web no implementa ningún sistema para autenticar y autorizar a los usuarios. Los servicios
				de consulta no necesitan ningún tipo de control, estos pueden ser accesibles por cualquiera. Sin embargo los servicios
				de acción futuros y presentes tendrían que estar controlados. Es por esta razón por la que surge la necesidad de
				implementar un sistema para gestionar el acceso de los usuarios. 
			\item[Adaptación para otras estaciones]
				Al igual que el software de adquisición la herramienta Web esta desarrollada en el entorno de CALMA. Existen muchas
				variables que actualmente no son manejadas debidamente, están \emph{Hard coded} en nuestro código. Como trabajo futuro
				se propone identificar todas estas variables y manejarlas debidamente.
		\end{description}
