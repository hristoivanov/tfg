\chapter{Manuales}\label{cap6}
\section{Manual de usuario}

A lo largo de este capítulo presentamos todas las posibles acciones que puede
realizar el usuario para acceder a la información de las fichas patrimoniales,
cómo realizar búsquedas detalladas, cómo interactuar con el módulo GIS,
asociación de fichas patrimoniales y cómo crear nuevos patrimonios relacionados
con la Vía de la Plata. 

De esta manera, se intentará ilustrar de la manera más sencilla posible el
manejo de la aplicación y de cada una de sus principales peculiaridades.

\subsection{Estructura de las ventanas}

La aplicación posee una estructura muy similar en cada una de sus secciones
principales. Todas las ventanas de la aplicación web constan de un panel
principal en el que se muestran todos los campos disponibles para ser
visualizados, modificados o creados y en la parte superior, una barra de
herramientas con el fin de facilitar al operador el acceso a la información.

Todas las secciones de acceso a fichas patrimoniales poseen similar estructura.
Para cada una de las opciones se contempla la opción de añadir una nueva ficha
patrimonial del tipo que corresponda o modificar una ya existente. Uno de los
menús desplegados tiene este aspecto:

\subsection{Creación de un elemento.}

Nunc dictum, lacus sodales auctor iaculis, lectus turpis pulvinar dolor, in
gravida nibh enim sit amet elit. Vestibulum sit amet volutpat sapien. Curabitur
nec ultricies massa. Aliquam erat volutpat. Donec maximus volutpat maximus. In
eu elementum ante. Pellentesque neque libero, pellentesque quis sagittis ac,
tempor vel mauris. Praesent et arcu nisi. Maecenas non iaculis mauris.

\subsection{Obtención de un histograma de respuesta.}

Nunc dictum, lacus sodales auctor iaculis, lectus turpis pulvinar dolor, in
gravida nibh enim sit amet elit. Vestibulum sit amet volutpat sapien. Curabitur
nec ultricies massa. Aliquam erat volutpat. Donec maximus volutpat maximus. In
eu elementum ante. Pellentesque neque libero, pellentesque quis sagittis ac,
tempor vel mauris. Praesent et arcu nisi. Maecenas non iaculis mauris.

\section{Manual de Instalación}

Hacer una correcta instalación del sistema es fundamental para un correcto
funcionamiento, por eso se detalla de aquí en adelante la instalación de
\emph{WampServer}, que es un paquete de herramientas que le facilitará al
administrador del sistema la instalación y configuración del servidor en caso
de que tenga que hacer una nueva instalación.

Nunc dictum, lacus sodales auctor iaculis, lectus turpis pulvinar dolor, in
gravida nibh enim sit amet elit. Vestibulum sit amet volutpat sapien. Curabitur
nec ultricies massa. Aliquam erat volutpat. Donec maximus volutpat maximus. In
eu elementum ante. Pellentesque neque libero, pellentesque quis sagittis ac,
tempor vel mauris. Praesent et arcu nisi. Maecenas non iaculis mauris.

