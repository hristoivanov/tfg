\chapter{Análisis de requisitos}
\label{cap3}

\section{Sobre la fase de análisis}

El análisis del problema generalemente describe el ``qué hay que hacer'',
mientras que el diseño dice ``cómo se va a hacer''. De esta forma pueden
existir muchos diseños diferentes solucionar un mismo problema descrito a
través del análisis del mismo.

En el análisis que se llevará a cabo se formalizarán los requisitos generales
del sistema, en los que se habla de los aspectos que éste debe cubrir y de las
distintas funcionalidades que debe poseer para llevar a cabo todas las
operaciones necesarias para cada entidad o usuario del sistema. Lo no
formalizado en estos requisitos se considera fuera del alcance del proyecto.

\section{Requisitos de usuario}

Nunc dictum, lacus sodales auctor iaculis, lectus turpis pulvinar dolor, in
gravida nibh enim sit amet elit. Vestibulum sit amet volutpat sapien. Curabitur
nec ultricies massa. Aliquam erat volutpat. Donec maximus volutpat maximus. In
eu elementum ante. Pellentesque neque libero, pellentesque quis sagittis ac,
tempor vel mauris. Praesent et arcu nisi. Maecenas non iaculis mauris.

\section{Casos de uso}

\section{Actores}

\section{Relaciones}

\section{Casos de uso de la aplicación}

A continuación se detallan los posibles casos de uso de la aplicación:

\subsection{Crear un elemento desde la aplicación}
\begin{description}
\item [Precondiciones:]No existen.
\item [Actores:]Podrá acceder cualquiera de los posibles actores antes
    definidos para la aplicación.
\item [Descripción:] Se encontrará un sistema de autenticación que deberá
    superar para acceder a la aplicación. Tras esto, deberá elegir el tipo de
    elemento que desea añadir (siempre que tenga los privilegios adecuados) y
    entrará a la ventana principal. Tras rellenar un formulario con los datos
    relacionados del elemento en cuestión, deberá guardarlo, de forma que
    quedará almacenado en base de datos.
\end{description}

\subsection{Modificar un elemento desde la aplicación}

Nunc dictum, lacus sodales auctor iaculis, lectus turpis pulvinar dolor, in
gravida nibh enim sit amet elit. Vestibulum sit amet volutpat sapien. Curabitur
nec ultricies massa. Aliquam erat volutpat. Donec maximus volutpat maximus. In
eu elementum ante. Pellentesque neque libero, pellentesque quis sagittis ac,
tempor vel mauris. Praesent et arcu nisi. Maecenas non iaculis mauris.


\section{Requisitos no funcionales}

Éstos hacen referencia a las características que deberá tener el sistema que no
hacen referencia a su funcionalidad, sino a sus características. Es decir, no
hacen referencia a la manera en que se guardan los datos en el sistema ni las
funciones que realiza la aplicación.

Los requisitos no funcionales principales serán:

\begin{enumerate}
    \item Accesibilidad a través de internet. Uno de los requisitos no
        funcionales de este sistema es la posibilidad que se nos brindará de
        acceder desde cualquier sitio a la aplicación sin necesidad de realizar
        instalaciones en un ordenador.
    \item Almacenamiento centralizado. La información se encontrará guardada en
        un servidor, así se tendrá todo bien organizado y a la vista de todo el
        mundo.
    \item Escalabilidad. En un futuro y si así se desea, la aplicación podrá
        ser ampliada para posibles mejoras.
    \item Eficiencia. Se buscará que la aplicación web sea lo más ligera
        posible, evitando así tiempos de espera y de transacciones para el
        usuario.
    \item Precio. El gasto inicial será bastante bajo.
    \item Fácil de mantener. Se buscará sencillez en el mantenimiento de la
        aplicación.
    \item Portabilidad. La aplicación deberá ser portable a otras plataformas
        sin problemas.
\end{enumerate}
