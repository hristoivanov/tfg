\chapter{Diseño del sistema}
\label{cap4}

\section{Arquitectura de Software}

La manera en la que se estructure nuestra aplicación será fundamental a la hora
del diseño, ya que se ha elegido una plataforma Cliente-Servidor, explicada más
adelante. La arquitectura hace referencia a la manera en la que está diseñada
la aplicación. En la arquitectura se pueden especificar detalles físicos
(hardware) y lógicos (software). Lo ideal es que la arquitectura esté
modularizada y las diferentes capas sean independientes, ya que se buscará al
máximo la portabilidad, la escalabilidad y la flexibilidad.

\subsection{Modelo Cliente-Servidor}

Es aquel utilizado para describir una aplicación en la que hay al menos dos
procesos separados trabajando con el fin de completar una misma tarea. El
cliente solicita al servidor la ejecución de una operación particular, llevando
a cabo un proceso cooperativo.

Nunc dictum, lacus sodales auctor iaculis, lectus turpis pulvinar dolor, in
gravida nibh enim sit amet elit. Vestibulum sit amet volutpat sapien. Curabitur
nec ultricies massa. Aliquam erat volutpat. Donec maximus volutpat maximus. In
eu elementum ante. Pellentesque neque libero, pellentesque quis sagittis ac,
tempor vel mauris. Praesent et arcu nisi. Maecenas non iaculis mauris.

\subsection{Patrones de diseño software}

Es la base de la que parte la búsqueda de soluciones para problemas comunes de
desarrollo software u otros ámbitos relacionados con el diseño y desarrollo
software.

Partimos de un problema de diseño. A partir de él, diseñamos un patrón que
posea una serie de características que proporcionen una solución estándar a
raíz de un problema dado.

Para que un diseño pueda ser denominado patrón, tendrá que tener como
característica fundamental la reusabilidad, ya que deberá ser aplicable a
diferentes problemas de diseño en distintas situaciones.

\subsection{Patrón Modelo-Vista-Controlador}

Nunc dictum, lacus sodales auctor iaculis, lectus turpis pulvinar dolor, in
gravida nibh enim sit amet elit. Vestibulum sit amet volutpat sapien. Curabitur
nec ultricies massa. Aliquam erat volutpat. Donec maximus volutpat maximus. In
eu elementum ante. Pellentesque neque libero, pellentesque quis sagittis ac,
tempor vel mauris. Praesent et arcu nisi. Maecenas non iaculis mauris.

\section{Arquitectura general de la aplicación}

Nunc dictum, lacus sodales auctor iaculis, lectus turpis pulvinar dolor, in
gravida nibh enim sit amet elit. Vestibulum sit amet volutpat sapien. Curabitur
nec ultricies massa. Aliquam erat volutpat. Donec maximus volutpat maximus. In
eu elementum ante. Pellentesque neque libero, pellentesque quis sagittis ac,
tempor vel mauris. Praesent et arcu nisi. Maecenas non iaculis mauris.

\section{Diseño del software del cliente}

Nunc dictum, lacus sodales auctor iaculis, lectus turpis pulvinar dolor, in
gravida nibh enim sit amet elit. Vestibulum sit amet volutpat sapien. Curabitur
nec ultricies massa. Aliquam erat volutpat. Donec maximus volutpat maximus. In
eu elementum ante. Pellentesque neque libero, pellentesque quis sagittis ac,
tempor vel mauris. Praesent et arcu nisi. Maecenas non iaculis mauris.

\subsection{Componentes utilizados, jerarquía e interconexión}

Nunc dictum, lacus sodales auctor iaculis, lectus turpis pulvinar dolor, in
gravida nibh enim sit amet elit. Vestibulum sit amet volutpat sapien. Curabitur
nec ultricies massa. Aliquam erat volutpat. Donec maximus volutpat maximus. In
eu elementum ante. Pellentesque neque libero, pellentesque quis sagittis ac,
tempor vel mauris. Praesent et arcu nisi. Maecenas non iaculis mauris.

\section{Diseño del software del servidor}

Nunc dictum, lacus sodales auctor iaculis, lectus turpis pulvinar dolor, in
gravida nibh enim sit amet elit. Vestibulum sit amet volutpat sapien. Curabitur
nec ultricies massa. Aliquam erat volutpat. Donec maximus volutpat maximus. In
eu elementum ante. Pellentesque neque libero, pellentesque quis sagittis ac,
tempor vel mauris. Praesent et arcu nisi. Maecenas non iaculis mauris.

\section{Arquitectura del sistema informático}

Nunc dictum, lacus sodales auctor iaculis, lectus turpis pulvinar dolor, in
gravida nibh enim sit amet elit. Vestibulum sit amet volutpat sapien. Curabitur
nec ultricies massa. Aliquam erat volutpat. Donec maximus volutpat maximus. In
eu elementum ante. Pellentesque neque libero, pellentesque quis sagittis ac,
tempor vel mauris. Praesent et arcu nisi. Maecenas non iaculis mauris.

\subsection{Servidor WEB (Apache)}

Nunc dictum, lacus sodales auctor iaculis, lectus turpis pulvinar dolor, in
gravida nibh enim sit amet elit. Vestibulum sit amet volutpat sapien. Curabitur
nec ultricies massa. Aliquam erat volutpat. Donec maximus volutpat maximus. In
eu elementum ante. Pellentesque neque libero, pellentesque quis sagittis ac,
tempor vel mauris. Praesent et arcu nisi. Maecenas non iaculis mauris.

\subsection{Intérprete de PHP (mod-php)}

Nunc dictum, lacus sodales auctor iaculis, lectus turpis pulvinar dolor, in
gravida nibh enim sit amet elit. Vestibulum sit amet volutpat sapien. Curabitur
nec ultricies massa. Aliquam erat volutpat. Donec maximus volutpat maximus. In
eu elementum ante. Pellentesque neque libero, pellentesque quis sagittis ac,
tempor vel mauris. Praesent et arcu nisi. Maecenas non iaculis mauris.

\begin{lstlisting}
function db_open($server,$user,$pass){
	$link=mysql_connect($server,$user,$pass);
    if(!$link){
    	$error=mysql_error();
        print ("<H1>No esta autorizado para acceder al servidor: $server</H1><BR>");
        print ("<P>Error: $error<P>");
        exit;
    }
    return $link;
}
\end{lstlisting}

