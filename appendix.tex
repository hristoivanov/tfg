\appendix

\chapter{Despliegue Software de adquisición}
	\label{app_soft}
	El propósito de este punto es detallar el proceso que debe seguirse para desplegar el software de adquisición. Siendo este un software para un
	sistema empotrado el estado del hardware tiene una gran influencia, en este punto suponemos que ya está configurado correctamente. También
	suponemos que sobre la \emph{BeagleBone Black} tenemos una distribución \emph{Angstrom} recién instalada.
	\section{Instalación de software} 
		El primer paso es instalar el software y bibliotecas que son necesarias. Antes de nada debemos actualizar la lista de paquetes
		disponibles. 
		\begin{lstlisting}[style=myBash]
$ opkg update
		\end{lstlisting}
		\par
		El primer paquete que debemos instalar es \texttt{ntp}, paquete que implementa el protocolo \emph{NTP}. Este paquete permite
		sincronizar la fecha y hora del sistema.  Una vez instalado este paquete debemos actualizar la fecha y hora del sistema, esto evitará
		posibles problemas.
		\begin{lstlisting}[style=myBash]
$ opkg install ntp
$ ntpdate -b -s -u pool.ntp.org
		\end{lstlisting}
		\par
		Nuestro software es escrito en \emph{Python}, debemos asegurarnos que la versión \texttt{2.7.x} esta instalada. En caso de no estar
		presente debemos instalarla.
		\begin{lstlisting}[style=myBash]
$ python --version
$ opkg install python
		\end{lstlisting}
		Seguidamente debemos instalar \texttt{pip}, herramienta que facilita el proceso de instalación de bibliotecas \emph{Python}.
		Utilizando \texttt{pip} debemos instalar las bibliotecas necesarias. El uso que hemos hecho de estas bibliotecas está descrito en el
		capítulo dedicado al software de adquisición.
		\begin{lstlisting}[style=myBash]
$ opkg install python-pip python-setuptools python-smbus
$ pip install pyserial
$ pip install db-sqlite3
$ pip install MySQL_python
$ pip install Adafruit_BBIO
		\end{lstlisting}
		\par
		Finalmente debemos instalar \emph{Git}, si este no está presente.
		\begin{lstlisting}[style=myBash]
$ opkg install git
		\end{lstlisting}
	\section{Clonar repositorio}
		El software de adquisición ha sido desarrollado utilizando \emph{Git} para llevar un control de versiones. Hemos utilizado
		\emph{GitHub} para mantener un repositorio remoto. La manera más fácil de acceder al software de adquisición es clonar el repositorio
		remoto. Antes de clonar el repositorio tenemos que crear un árbol de carpetas similar a este.
		\begin{lstlisting}[style=myBash]
--/server
   --/server/nmda
   --/server/data
   --/server/logs
		\end{lstlisting}
		A continuación procedemos a clonar el repositorio remoto. Una vez clonado el repositorio podemos utilizar el comando \texttt{git pull}
		para actualizar el software. Para utilizar los dos comandos que a continuación presentamos tenemos que estar en la carpeta
		\texttt{/server/nmda}.
		\begin{lstlisting}[style=myBash]
$ git clone https://github.com/opobla/nmpw.git .
$ git pull
		\end{lstlisting}
		En el capítulo dedicado al software de adquisición se explica como este lee una serie de variables desde un archivo de configuración,
		este archivo debe seguir un formato que también se explica en ese mismo capítulo. Este archivo debe tener el nombre
		\texttt{/server/nmda/.NMDA.conf}. Podemos crear el archivo desde cero o podemos utilizar el archivo de ejemplo que hemos clonado desde
		el repositorio remoto. Este archivo de ejemplo especifica una configuración mínima que permite correr el software. 
		\begin{lstlisting}[style=myBash]
$ cp /server/nmda/NMDA.conf.exmaple /server/nmda/.NMDA.conf
		\end{lstlisting}
	\section{\emph{System Services}}
		\label{appendix:systemctl}
		En el capítulo dedicado al software de adquisición se explica como el sistema de adquisición debe inicializarse automáticamente ante
		la presencia de corriente. La \emph{BeagleBone Black} por defecto está configurada para arrancar automáticamente. Nos queda configurar el
		sistema operativo para arrancar nuestro software de forma automática. Para este propósito vamos a utilizar las \emph{System Services}. 
		\par
		Los servicios de sistema se crean mediante archivos con extensión \texttt{.service} que deben ser guardados en el directorio
		\texttt{/lib/systemd/system}. A continuación listamos los archivos que definen los servicios que necesitamos.
		\begin{lstlisting}[style=myFile]
# /lib/systemd/system/ntpdate.service
[Unit]
Description=Network Time Service (one-shot ntpdate mode)
Before=ntpd.service

[Service]
Type=oneshot
ExecStart=/usr/bin/ntpd -q -g -x
RemainAfterExit=yes

[Install]
WantedBy=multi-user.target
		\end{lstlisting}
		\begin{lstlisting}[style=myFile]
# /lib/systemd/system/ntpd.service
[Unit]
Description=Network Time Service
After=network.target

[Service]
Type=forking
PIDFile=/run/ntpd.pid
ExecStart=/usr/bin/ntpd -p /run/ntpd.pid

[Install]
WantedBy=multi-user.target
		\end{lstlisting}
		\begin{lstlisting}[style=myFile]
# /lib/systemd/system/nmda.service
[Unit]
Description=Neutron Monitor Data Acquisition Service
After=ntpdate.service

[Service]
ExecStart=/usr/bin/python /server/nmda/NMDA.py

[Install]
WantedBy=multi-user.target
		\end{lstlisting}
		\begin{lstlisting}[style=myFile]
# /lib/systemd/system/myWatchDog.service         
[Unit]
Description=WatchDog

[Service]
ExecStart=/usr/bin/python /server/nmda/WatchDog.py

[Install]
WantedBy=multi-user.target
		\end{lstlisting}
		Para habilitar los servicios de sistema que acabamos de declarar tenemos que usar el comando \texttt{systemctl}.
		\begin{lstlisting}[style=myBash]
$ systemctl enable ntpdate.service
$ systemctl enable ntpd.service
$ systemctl enable nmda
$ systemctl enable myWatchDog
		\end{lstlisting}
		Vemos que dos servicios utilizan \texttt{ntpd}. Para asegurar el correcto funcionamiento de este programa tenemos que editar el
		archivo de configuración tal y como exponemos a continuación.
		\begin{lstlisting}[style=myFile]
# /etc/ntp.conf
driftfile /etc/ntp.drift
server pool.ntp.org
restrict default
		\end{lstlisting}
	\section{Tarjeta microSD}
		Tal y como explicamos en el capítulo dedicado al entorno hardware la \emph{BeagleBone Black} dispone de una memoria integrada
		(\emph{eMMC}) de 4GB. En esta memoria tenemos el sistema operativo y todo el software necesario para el sistema de adquisición. Esto
		hace que el porcentaje de ocupación de esta memoria sea muy alto. Es conveniente hacer uso de la ranura \emph{microSD} (\emph{uSD}) para extender
		la capacidad de almacenamiento. A continuación se detallan los pasos a seguir.
		\par
		El primer paso es arrancar la \emph{BeagleBone Black} sin la tarjeta \emph{uSD}. Después debemos hacer uso del comando \texttt{fdisk}, programa que
		permite manipular las particiones de los dispositivos de almacenamiento.
		\begin{lstlisting}[style=myBash]
# Listar las particiones de los dispositivos presentes.
$ fdisk -l
Disk /dev/mmcbk0: 3867 MB, 3867148288 bytes #eMMC

	Device Boot Start     End  Blocks Id System
/dev/mmcblk0p1    *  2048  198655   98304  e W95 FAT16 (LBA)
/dev/mmcblk0p2     198656 7553023 3677184 83 Linux
		\end{lstlisting}
		Seguidamente debemos insertar la tarjeta \emph{uSD} y volver a usar el mismo comando.
		\begin{lstlisting}[style=myBash]
# Listar las particiones de los dispositivos presentes.
$ fdisk -l
Disk /dev/mmcbk0: 3867 MB, 3867148288 bytes #eMMC
	Device Boot Start     End  Blocks Id System
/dev/mmcblk0p1    *  2048  198655   98304  e W95 FAT16 (LBA)
/dev/mmcblk0p2     198656 7553023 3677184 83 Linux

# Puede variar dependiendo de como tengamos particionada la uSD.
# No importa porque vamos a borrar las particiones existentes.
Disk /dev/mmcblk1: 7948 MB, 7948206080 bytes #uSD
	Device Boot Start      End  Blocks Id System
/dev/mmcblk1p1       2048 15523839   98304  c W95 FAT32 (LBA)
		\end{lstlisting}
		Usando \texttt{fdisk} debemos borrar todas las particiones de la \emph{uSD} y después crear dos nuevas. El estado final de la
		\emph{uSD} es presentado a continuación.
		\begin{lstlisting}[style=myBash]
$ fdisk -l
Disk /dev/mmcblk1: 7948 MB, 7948206080 bytes #uSD
        Device Boot  Start      End  Blocks Id  System
/dev/mmcblk1p1        2048   198655   98304  e  W95 FAT16 (LBA)
/dev/mmcblk1p2      198656 15523839 7662592 83  Linux
		\end{lstlisting}
		Una vez creadas las particiones estas debes ser formateadas. A continuación presentamos las instrucciones necesarias utilizando
		\texttt{mkfs}.
		\begin{lstlisting}[style=myBash]
$ mkfs -t vfat /dev/mmcblk1p1
$ mkfs -t ext3 /dev/mmcblk1p2
		\end{lstlisting}
		Antes de seguir es conveniente explicar algunos aspectos del proceso de arranque en la \emph{BeagleBone Black}. Después de una serie
		de pasos, que están fuera del alcance de este trabajo, es cargado desde la \emph{eMMC}(partición \texttt{FAT16}) el programa
		\emph{U-boot}. Este es el encargado de cargar el núcleo Linux y proporcionar información sobre el sistema de archivos Linux. Este
		programa busca el archivo \texttt{uEnv.txt} desde el que lee una serie de parámetros de configuración como la posición del propio
		núcleo Linux que queremos cargar. Algunos de estos parámetros pueden ser pasados al propio núcleo siempre y cuando este los acepte,
		por ejemplo la siguiente línea nos permite habilitar dos de los puertos serie que la \emph{BeagleBone Black} ofrece.
		\begin{lstlisting}[style=myFile]
optargs=quiet drm.debug=7 capemgr.enable_partno=BB-UART2,BB-UART1
		\end{lstlisting}
		Acabamos de explicar este proceso porque ante la presencia de una \emph{uSD} el \emph{U-boot} busca el archivo \texttt{uEnv.txt} en la
		\emph{uSD}. Si el archivo no está presente el programa no puede seguir. Es por esta razón por la que hemos creado la partición
		\texttt{FAT16} en la \emph{uSD}. Para crear el archivo tenemos que seguir los siguientes pasos.
		\begin{lstlisting}[style=myBash]
$ mkdir trash
$ mount /dev/mmcblk1p1 trash
$ vim trash/uEnv.txt
$ umount /dev/mmcblk1p1
$ rm -R trash
		\end{lstlisting}
		El contenido del archivo debe ser el siguiente.
		\begin{lstlisting}[style=myFile]
mmcdev=1
bootpart=1:2
mmcroot=/dev/mmcblk1p2 ro
optargs=quiet 
		\end{lstlisting}
		La línea interesante del archivo es la que asigna valor a \texttt{mmcroot}. Esta variable debe especificar donde está el núcleo Linux
		que querremos arrancar. El valor asignado es \texttt{/dev/mmcblk1p2}, los lectores atentos se habrán dado cuenta que este valor
		corresponde a la segunda partición dentro de la \emph{uSD}, sitio donde no está el núcleo.
		\par 
		En este punto la \emph{eMMC} es el dispositivo \texttt{/dev/mmcbk0} y la \emph{uSD} es el dispositivo \texttt{/dev/mmcbk1}, esto es así porque
		arrancamos sin la \emph{uSD} y añadimos esta posteriormente. Si arrancamos con la \emph{uSD} presente esta es reconocida como el dispositivo
		\texttt{/dev/mmcbk0} y la \emph{eMMC} es \texttt{/dev/mmcbk1}. Teniendo esto en consideración la próxima vez que arranquemos la placa con la
		\emph{uSD} presente la variable \texttt{mmcroot} apuntará correctamente a la partición que contiene el núcleo Linux.
		\par
		En este punto tan sólo nos queda definir un punto de montaje para la segunda partición de la \emph{uSD}, la partición que guardará los datos
		del sistema de adquisición. Para este propósito tenemos que editar el archivo \texttt{/etc/fstab}, tenemos que añadir la siguiente
		línea al final del archivo. La partición será montada en el directorio \texttt{/server/data}.
		\begin{lstlisting}[style=myFile]
/dev/mmcblk0p1   /server/data   auto   defaults     0  0
		\end{lstlisting}
		Finalmente para que se apliquen los cambios debemos reiniciar la placa con la \emph{uSD} insertada.


\chapter{Despliegue aplicación Web.}
\section{\emph{Back-end}}
	\label{app_back}
	El propósito de este punto es detallar el proceso que debe seguirse para desplegar el \emph{back-end} de la aplicación Web. Debido a que en
	este trabajo tan sólo nos centramos en el proceso de desarrollo, el proceso de despliegue presentado es para desplegar una versión de
	desarrollo que permita seguir trabajando en el proyecto.
	\par
	El primer paso es desplegar un servidor \emph{MySql} que contiene los datos que vamos a usar. Es recomendable instalar este servidor en la misma
	máquina desde la que vamos a trabajar, los pasos presentados a continuación asumen que hemos procedido de esta manera. El siguiente comando
	permite instalar un servidor \emph{MySql}. Junto a este es instalado un programa cliente que permite conectar se al servidor desde una consola.
	\begin{lstlisting}[style=myBash]
$ sudo apt-get install mysql-server mysql-client
	\end{lstlisting}
	El siguiente paso es acceder como usuario \texttt{root} al servidor \emph{Mysql}, para este propósito debemos usar el programa cliente que acabamos
	de instalar. Seguidamente debemos crear dos bases de datos, es conveniente que usemos los nombres \texttt{nmdadb} y \texttt{nmdb}. Finalmente
	debemos crear un usuario que el \emph{back-end} utilizará para conectarse al servidor. Debemos dar permisos al usuario sobre las bases de
	datos que acabamos de crear, en este caso damos privilegios completos al usuario.
	\begin{lstlisting}[style=myBash]
$ mysql -u root -p
mysql> CREATE DATABASE nmdadb;
mysql> CREATE DATABASE nmdb;
mysql> CREATE USER 'hristo'@'localhost' IDENTIFIED 'pass';
mysql>  GRANT ALL PRIVILEGES ON *.* TO 'hristo'@'localhost';
mysql> FLUSH PRIVILEGES;
	\end{lstlisting}
	El siguiente paso es introducir datos en las bases de datos. En este trabajo hemos utilizado los datos generados por CaLMa. Los datos se
	almacenan en un servidor \emph{MySql}, para transferir el contenido de un servidor a otro hemos utilizado el comando \texttt{mysqldump} que permite
	verter el contenido de una base en un archivo. A continuación podemos ver un ejemplo de uso.
	\begin{lstlisting}[style=myBash]
$ mysqldump -h remotehost -u hristo -ppass nmdadb > nmdadb.sql
$ mysql -h localhost -u hristo -ppass nmdadb < nmdadb.sql
	\end{lstlisting}
	Antes de continuar debemos asegurarnos de que las tablas dentro de las bases de datos siguen el esquema presentado en el capítulo
	\ref{backend}. También debemos asegurarnos de que los datos se han transferido correctamente.
	\par
	Una vez configuradas las bases de datos podemos seguir con la instalación del software que compone el \emph{back-end}. Este software es
	desarrollado en \emph{PHP}, que podemos instalar usando el comando presentado a continuación. Este comando también instala la extensión
	\texttt{php-mysql} que habilita la conexión al servidor \emph{MySql}. Es importante destacar que necesitamos una versión \emph{PHP} superior a la
	\texttt{5.3.23}, requerida por \emph{ZendFramework} y \emph{Apigility}.
	\begin{lstlisting}[style=myBash]
$ apt-get install php5 php5-mysql
	\end{lstlisting}
	El siguiente paso es clonar el repositorio \emph{Git} que contiene nuestro proyecto. Para este propósito debemos utilizar el comando presentado a
	continuación.
	\begin{lstlisting}[style=myBash]
$ git clone https://github.com/opobla/nmPanel.git directorio_de_instalacion
	\end{lstlisting}
	Para seguir es conveniente navegar hasta el directorio raíz del proyecto que acabamos de clonar. El siguiente paso es instalar todas las
	dependencias de nuestro software. Este proceso puede resultar tedioso, razón por la que hemos utilizado \emph{Composer}, herramienta que permite
	gestionar las dependencias en \emph{PHP}. Es conveniente destacar que \emph{Composer} no es un gestor de paquetes, este gestiona las dependencias a nivel de
	proyecto, instalando estas en el directorio \texttt{ventor} del proyecto. Podemos instalar \emph{Composer} de dos maneras, globalmente como un
	comando y localmente como parte del proyecto. A continuación podemos ver como instalar este de forma local. El segundo comando es el que
	instala las dependencias. 
	\begin{lstlisting}[style=myBash]
$ curl -sS https://getcomposer.org/installer | php
$ php composer.phar install
	\end{lstlisting}
	Para declarar las dependencias del proyecto es usado el archivo \texttt{Composer.json}. A continuación podemos ver un ejemplo de como declarar
	una dependencia básica. 
	\begin{lstlisting}[style=myFile]
{
   "require": {
      "monolog/monolog": "1.2.*"
   }
}
	\end{lstlisting}
	Para trabajar con la herramienta \emph{Apigility}, es necesario habilitar el modo de desarrollo, usando el comando presentado a continuación. 
	\begin{lstlisting}[style=myBash]
$ php public/index.php development enable
	\end{lstlisting}
	Finalmente debemos configurar un servidor Web desde el que acceder a nuestra aplicación. Podemos utilizar un servidor Web completo como
	\emph{Apache}\cite{Apache}, pero en este trabajo por comodidad hemos utilizado el servidor Web interno que \emph{PHP} ofrece. Para arrancar este debemos
	utilizar el siguiente comando. Es importante que  el directorio raíz sea el directorio \texttt{public} de nuestro proyecto.
	\begin{lstlisting}[style=myBash]
$ php -S 0.0.0.0:8080 -t public/ public/index.php
	\end{lstlisting}
	De esta manera todo el contenido presente en el directorio \texttt{public} será visible desde un navegador Web. Un ejemplo es nuestro
	\texttt{front-end}, que desplegaremos en el subdirectorio \texttt{public/nmCpanel}. A continuación presentamos las \emph{URL's} que permiten
	acceder a \emph{Apigility} y a nuestra herramienta siempre y cuando esta esté desplegada.
	\begin{lstlisting}[style=myFile]
http://localhost:8080/
http://localhost:8080/nmCpanel/index.html
	\end{lstlisting}
	Eventualmente para que todo funcione bien debemos verificar que la configuración referente a las bases de datos que creamos anteriormente es
	correcta. Para este propósito debemos examinar los archivos \texttt{config/autoload/global.php} y \texttt{config/autoload/local.php}.

\section{\emph{Front-end}}
	\label{app_front}
	El propósito de este punto es detallar el proceso que permite desplegar el \emph{front-end} de la aplicación Web. Al igual que en la sección
	anterior describimos el proceso que nos permite desplegar una versión de desarrollo que permita seguir trabajando en el proyecto.
	\par
	La aplicación ha sido desarrollada con \emph{Sencha Architect}, entorno de desarrollo que facilita el proceso de desarrollo de aplicaciones con
	\emph{ExtJs}. El primer paso es descargar e instalar la última versión de la herramienta. La herramienta se debe descargar desde la página oficial de
	\emph{Sencha}. La instalación es llevada a cabo mediante un entorno gráfico. Al instalar la herramienta, la biblioteca \emph{ExtJs} será instalada
	automáticamente.
	\par
	El siguiente paso es clonar el repositorio \emph{Git} que contiene nuestro proyecto. Para este propósito debemos utilizar el siguiente comando.
	\begin{lstlisting}[style=myBash]
$ git clone https://github.com/opobla/nmcpanel.git directorio_de_proyecto
	\end{lstlisting}
	Finalmente debemos abrir el proyecto con \emph{Sencha Architect}. Seguidamente podemos empezar a trabajar. Para desplegar una versión de desarrollo
	debemos utilizar el botón de desplegar.

\chapter{Proceso de implementación}
	El propósito de este punto es detallar los aspectos más importantes del proceso de implementación. Nos centraremos en las herramientas y
	técnicas que hemos utilizado.
	\par
	A lo largo de todo el trabajo hemos utilizado el editor de texto \emph{VIM}\cite{vim}, desde la creación del software de control hasta la edición de
	este documento. Este es un editor de texto que ofrece una interfaz accesible mediante una línea de comandos, esta interfaz no se basa en iconos
	y menús, sino en comandos en forma de texto. Este es un editor que puede resultar muy complejo a los usuarios novatos, pero una vez amaestrado
	tiene un gran potencial debido a la gran cantidad de atajos y personalizaciones que podemos realizar.
	\par
	\emph{Secure Shell} o \emph{SSH}\cite{ssh} es un programa que implementa el protocolo con el mismo nombre. Este programa permite el acceso a máquinas
	remotas atreves de  la red mediante un intérprete de comandos, de esta manera podemos tener control total de la máquina remota. Hemos
	utilizado este programa para trabajar con la \emph{BeagleBone Black} y actualmente lo estamos utilizando para realizar todas las labores de
	mantenimiento del software de adquisición.
	\par
	Durante el desarrollo de la herramienta Web hemos utilizado el servidor Web interno que \emph{PHP} ofrece. Este es muy cómodo, fácil y sobre todo
	rápido de utilizar. No es recomendable utilizar este servidor en las faces de producción por lo que debemos configurar un servidor Web
	completo como puede ser \emph{Apache}\cite{Apache}.
	\par
	\emph{Postman}\cite{Postman} es una herramienta Web que permite crear y enviar \emph{HTTP requests}. La funcionalidad es parecida al programa de
	línea de comandos \emph{cURL}. La diferencia radica en que esta herramienta ofrece una interfaz gráfica que facilita el trabajo. La herramienta
	también ofrece un historial de las peticiones realizadas que resulta muy útil. Hemos utilizado esta herramienta durante el desarrollo del
	\emph{back-end} de nuestra herramienta Web.
	\par
	Durante el desarrollo del \emph{front-end} hemos utilizado \emph{Chrome} y las herramientas de desarrollador que este ofrece\cite{ChromeDev}.
	\emph{Chrome} es un navegador Web con una popularidad creciente entre usuarios y desarrolladores. En este trabajo hemos hecho uso
	principalmente del \emph{debugger JavaScript}. También hemos hecho uso de la herramienta \emph{Network} que da información sobre el uso de la red, los
	mensajes que nuestra herramienta intercambia.
	\par
	\emph{Sencha Architect} es un entorno de desarrollo de aplicaciones basadas en \emph{ExtJs} que hemos utilizado para la creación del \emph{front-end} de
	nuestra aplicación. Este entorno facilita el desarrollo ofreciendo un entorno gráfico que permite generar gran parte del código de forma
	automática. También ofrece una vista previa de la aplicación en tiempo real según editamos el código fuente de esta. 
	\par
	Para la elaboración de este documento hemos utilizado \LaTeX\cite{latex}. Este es un conjunto de macros escritos en el lenguaje \TeX\cite{tex}
	que tienen como propósito facilitar el uso de este. \TeX\ un lenguaje de marcado pensado para la creación de documentos. Para los
	usuarios novatos \LaTeX\ puede resultar muy difícil, en este trabajo hemos empezado utilizando una plantilla sobre la que hemos añadido
	contenido. Esto ha permitido centrarnos en el contenido del documento, despreocupándonos de la presentación. 
	\par
	A lo largo de este trabajo hemos realizado un control de versiones, para este propósito hemos utilizado \emph{Git}\cite{git}. Esta es una herramienta
	fácil de aprender y usar que enfatiza en la velocidad y el desarrollo no lineal. Para mantener un repositorio remoto hemos utilizado
	\emph{GitHub}\cite{github}, servicio Web que permite alojar repositorios. A diferencia de \emph{Git}, que es una herramienta de línea de
	comandos, \emph{GitHub}
	ofrece una interfaz gráfica. \emph{GitHub} también ofrece algunas funcionalidades extra como rastreo de \emph{bugs}, gestión de tareas,
	\emph{wikis} y otros. En
	este trabajo hemos mantenido cuatro repositorios que respetivamente contienen el software de adquisición, el \emph{back-end}, el
	\emph{front-end} y finalmente el proyecto \LaTeX\ de este documento. A continuación presentamos las \emph{URL’s} de los cuatro repositorios remotos en
	\emph{GitHub}, podemos acceder a estos mediante un navegador Web e inspeccionarlos. 
	\\
	\\	Software de adquisición:	\hfill	\url{https://github.com/opobla/nmpw}
	\\	\emph{Back-End}:		\hfill	\url{https://github.com/opobla/nmPanel}
	\\	\emph{Front-End}:		\hfill	\url{https://github.com/opobla/nmCpanel}
	\\	Documento \LaTeX:		\hfill	\url{https://github.com/hristoivanov/tfg}
