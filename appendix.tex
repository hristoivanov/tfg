\appendix

\chapter{Despliegue Software de adquisición}
	El propósito de este punto es detallar el proceso que debe seguirse para desplegar el software de adquisición. Siendo este un software para un
	sistema empotrado el estado del hardware tiene una gran influencia, en este punto suponemos que ya esta configurado correctamente. También
	suponemos que la BeagleBone Black esta esta tal y como la distribuye el fabricante.
	\section{Instalación de software} 
		El primer paso es instalar el software y librerías que son necesarias. Antes de nada debemos actualizar la lista de paquetes
		disponibles. 
		\begin{lstlisting}[style=myBash]
$ opkg update
		\end{lstlisting}
		\par
		El primer paquete que debemos instalar es \texttt{ntp}, paquete que implementa el protocolo \emph{NTP}. Este paquete nos permite
		sincronizar la fecha y hora del sistema.  Una vez instalado este paquete debemos actualizar la fecha y hora del sistema, esto evitará
		posibles errores.
		\begin{lstlisting}[style=myBash]
$ opkg install ntp
$ ntpdate -b -s -u pool.ntp.org
		\end{lstlisting}
		\par
		Nuestro software esta escrito en Python, debemos asegurarnos que la versión \texttt{2.7.x} esta instalada. En caso de no estar
		presente debemos instalarla.
		\begin{lstlisting}[style=myBash]
$ python --version
$ opkg install python
		\end{lstlisting}
		Seguidamente debemos instalar \texttt{pip}, herramienta que facilita el proceso de instalación de librerías Python.  Utilizando
		\texttt{pip} debemos instalar las librerías necesarias. El uso que hemos hecho de estas librerías esta descrito en el capitulo
		dedicado al software de adquisición.
		\begin{lstlisting}[style=myBash]
$ opkg install python-pip python-setuptools python-smbus
$ pip install pyserial
$ pip install db-sqlite3
$ pip install MySQL-python
$ pip install Adafruit\_BBIO
		\end{lstlisting}
		\par
		Finalmente debemos instalar Git, si este no esta presente.
		\begin{lstlisting}[style=myBash]
$ opkg install git
		\end{lstlisting}
	\section{Clonar repositorio}
		El software de adquisición ha sido desarrollado utilizando Git para llevar un control de versiones. Hemos utilizado GitHub para
		mantener un repositorio remoto. La manera más fácil de acceder al software de adquisición es clonar el repositorio remoto. Antes de
		clonar el repositorio tenemos que crear un árbol de carpetas similar a este.
		\begin{lstlisting}[style=myBash]
--/server
   --/server/nmda
   --/server/data
   --/server/logs
		\end{lstlisting}
		A continuación procedemos a clonar el repositorio remoto. Una vez clonado el repositorio podemos utilizar el comando Git \texttt{pull}
		para actualizar el software. Para utilizar los dos comandos que a continuación presentamos tenemos que estar en la carpeta
		\texttt{/server/nmda}.
		\begin{lstlisting}[style=myBash]
$ git clone git@github.com:opobla/nmpw.git .
$ git pull
		\end{lstlisting}
		En el capitulo dedicado al software de adquisición se explica como este lee una serie de variables desde un archivo de configuración,
		este archivo debe seguir un formato que también se explica en ese mismo capitulo. Este archivo debe tener el nombre
		\texttt{/server/nmda/.NMDA.conf}. Podemos crear el archivo desde cero o podemos utilizar el archivo de ejemplo que hemos clonado desde
		el repositorio remoto. Este archivo de ejemplo especifica una configuración mínima que nos permite correr el software. Basta con
		copiar el archivo de ejemplo para utilizarlo.
		\begin{lstlisting}[style=myBash]
$ cp /server/nmda/NMDA.conf.exmaple /server/nmda/.NMDA.conf
		\end{lstlisting}
	\section{\emph{System Services}}
		En el capitulo dedicado al software de adquisición se explica como el sistema de adquisición debe inicializarse automáticamente ante
		la presencia de corriente. La BeagleBone Black por defecto está configurada para arrancar automáticamente. Nos queda configurar el
		sistema operativo arrancar nuestro software de forma automática. Para este propósito vamos a utilizar las \emph{System Services}. 
		\par
		Los servicios de sistema se crean mediante archivos con extensión \texttt{.service} que deben ser guardados en la carpeta
		\texttt{/lib/systemd/system}. A continuación listamos los archivos que definen los servicios que necesitamos.
		\begin{lstlisting}[style=myFile]
# /lib/systemd/system/ntpdate.service
[Unit]
Description=Network Time Service (one-shot ntpdate mode)
Before=ntpd.service

[Service]
Type=oneshot
ExecStart=/usr/bin/ntpd -q -g -x
RemainAfterExit=yes

[Install]
WantedBy=multi-user.target
		\end{lstlisting}
		\begin{lstlisting}[style=myFile]
# /lib/systemd/system/ntpd.service
[Unit]
Description=Network Time Service
After=network.target

[Service]
Type=forking
PIDFile=/run/ntpd.pid
ExecStart=/usr/bin/ntpd -p /run/ntpd.pid

[Install]
WantedBy=multi-user.target
		\end{lstlisting}
		\begin{lstlisting}[style=myFile]
# /lib/systemd/system/nmda.service
[Unit]
Description=Neutron Monitor Data Acquisition Service
After=ntpdate.service

[Service]
ExecStart=/usr/bin/python /server/nmda/NMDA.py

[Install]
WantedBy=multi-user.target
		\end{lstlisting}
		\begin{lstlisting}[style=myFile]
# /lib/systemd/system/myWatchDog.service         
[Unit]
Description=WatchDog

[Service]
ExecStart=/usr/bin/python /server/nmda/WatchDog.py

[Install]
WantedBy=multi-user.target
		\end{lstlisting}
		Para habilitar los servicios de sistema que acabamos de declarar tenemos que usar el comando \texttt{systemctl}.
		\begin{lstlisting}[style=myBash]
$ systemctl enable ntpdate.service
$ systemctl enable ntpd.service
$ systemctl enable nmda
$ systemctl enable myWatchDog
		\end{lstlisting}
		Vemos que dos servicios utilizan \texttt{ntpd}. Para asegurar el correcto funcionamiento de este programa tenemos que editar el
		archivo de configuración tal y como exponemos a continuación.
		\begin{lstlisting}[style=myFile]
# /etc/ntp.conf
driftfile /etc/ntp.drift
server pool.ntp.org
restrict default
		\end{lstlisting}
