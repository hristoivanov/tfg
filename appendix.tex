\appendix

\chapter{Despliegue Software de adquisición}
	\label{app_soft}
	El propósito de este punto es detallar el proceso que debe seguirse para desplegar el software de adquisición. Siendo este un software para un
	sistema empotrado el estado del hardware tiene una gran influencia, en este punto suponemos que ya esta configurado correctamente. También
	suponemos que sobre la BeagleBone Black tenemos una distribución Angstrom recién instalada.
	\section{Instalación de software} 
		El primer paso es instalar el software y librerías que son necesarias. Antes de nada debemos actualizar la lista de paquetes
		disponibles. 
		\begin{lstlisting}[style=myBash]
$ opkg update
		\end{lstlisting}
		\par
		El primer paquete que debemos instalar es \texttt{ntp}, paquete que implementa el protocolo \emph{NTP}. Este paquete permite
		sincronizar la fecha y hora del sistema.  Una vez instalado este paquete debemos actualizar la fecha y hora del sistema, esto evitará
		posibles problemas.
		\begin{lstlisting}[style=myBash]
$ opkg install ntp
$ ntpdate -b -s -u pool.ntp.org
		\end{lstlisting}
		\par
		Nuestro software esta escrito en Python, debemos asegurarnos que la versión \texttt{2.7.x} esta instalada. En caso de no estar
		presente debemos instalarla.
		\begin{lstlisting}[style=myBash]
$ python --version
$ opkg install python
		\end{lstlisting}
		Seguidamente debemos instalar \texttt{pip}, herramienta que facilita el proceso de instalación de librerías Python.  Utilizando
		\texttt{pip} debemos instalar las librerías necesarias. El uso que hemos hecho de estas librerías esta descrito en el capitulo
		dedicado al software de adquisición.
		\begin{lstlisting}[style=myBash]
$ opkg install python-pip python-setuptools python-smbus
$ pip install pyserial
$ pip install db-sqlite3
$ pip install MySQL_python
$ pip install Adafruit_BBIO
		\end{lstlisting}
		\par
		Finalmente debemos instalar Git, si este no esta presente.
		\begin{lstlisting}[style=myBash]
$ opkg install git
		\end{lstlisting}
	\section{Clonar repositorio}
		El software de adquisición ha sido desarrollado utilizando Git para llevar un control de versiones. Hemos utilizado GitHub para
		mantener un repositorio remoto. La manera más fácil de acceder al software de adquisición es clonar el repositorio remoto. Antes de
		clonar el repositorio tenemos que crear un árbol de carpetas similar a este.
		\begin{lstlisting}[style=myBash]
--/server
   --/server/nmda
   --/server/data
   --/server/logs
		\end{lstlisting}
		A continuación procedemos a clonar el repositorio remoto. Una vez clonado el repositorio podemos utilizar el comando Git \texttt{pull}
		para actualizar el software. Para utilizar los dos comandos que a continuación presentamos tenemos que estar en la carpeta
		\texttt{/server/nmda}.
		\begin{lstlisting}[style=myBash]
$ git clone https://github.com/opobla/nmpw.git .
$ git pull
		\end{lstlisting}
		En el capitulo dedicado al software de adquisición se explica como este lee una serie de variables desde un archivo de configuración,
		este archivo debe seguir un formato que también se explica en ese mismo capitulo. Este archivo debe tener el nombre
		\texttt{/server/nmda/.NMDA.conf}. Podemos crear el archivo desde cero o podemos utilizar el archivo de ejemplo que hemos clonado desde
		el repositorio remoto. Este archivo de ejemplo especifica una configuración mínima que permite correr el software. 
		\begin{lstlisting}[style=myBash]
$ cp /server/nmda/NMDA.conf.exmaple /server/nmda/.NMDA.conf
		\end{lstlisting}
	\section{\emph{System Services}}
		\label{appendix:systemctl}
		En el capitulo dedicado al software de adquisición se explica como el sistema de adquisición debe inicializarse automáticamente ante
		la presencia de corriente. La BeagleBone Black por defecto está configurada para arrancar automáticamente. Nos queda configurar el
		sistema operativo para arrancar nuestro software de forma automática. Para este propósito vamos a utilizar las \emph{System Services}. 
		\par
		Los servicios de sistema se crean mediante archivos con extensión \texttt{.service} que deben ser guardados en el directorio
		\texttt{/lib/systemd/system}. A continuación listamos los archivos que definen los servicios que necesitamos.
		\begin{lstlisting}[style=myFile]
# /lib/systemd/system/ntpdate.service
[Unit]
Description=Network Time Service (one-shot ntpdate mode)
Before=ntpd.service

[Service]
Type=oneshot
ExecStart=/usr/bin/ntpd -q -g -x
RemainAfterExit=yes

[Install]
WantedBy=multi-user.target
		\end{lstlisting}
		\begin{lstlisting}[style=myFile]
# /lib/systemd/system/ntpd.service
[Unit]
Description=Network Time Service
After=network.target

[Service]
Type=forking
PIDFile=/run/ntpd.pid
ExecStart=/usr/bin/ntpd -p /run/ntpd.pid

[Install]
WantedBy=multi-user.target
		\end{lstlisting}
		\begin{lstlisting}[style=myFile]
# /lib/systemd/system/nmda.service
[Unit]
Description=Neutron Monitor Data Acquisition Service
After=ntpdate.service

[Service]
ExecStart=/usr/bin/python /server/nmda/NMDA.py

[Install]
WantedBy=multi-user.target
		\end{lstlisting}
		\begin{lstlisting}[style=myFile]
# /lib/systemd/system/myWatchDog.service         
[Unit]
Description=WatchDog

[Service]
ExecStart=/usr/bin/python /server/nmda/WatchDog.py

[Install]
WantedBy=multi-user.target
		\end{lstlisting}
		Para habilitar los servicios de sistema que acabamos de declarar tenemos que usar el comando \texttt{systemctl}.
		\begin{lstlisting}[style=myBash]
$ systemctl enable ntpdate.service
$ systemctl enable ntpd.service
$ systemctl enable nmda
$ systemctl enable myWatchDog
		\end{lstlisting}
		Vemos que dos servicios utilizan \texttt{ntpd}. Para asegurar el correcto funcionamiento de este programa tenemos que editar el
		archivo de configuración tal y como exponemos a continuación.
		\begin{lstlisting}[style=myFile]
# /etc/ntp.conf
driftfile /etc/ntp.drift
server pool.ntp.org
restrict default
		\end{lstlisting}
	\section{Tarjeta microSD}
		Tal y como explicamos en el capitulo dedicado al entorno hardware la BeagleBone Black dispone de una memoria integrada (eMMC) de 4GB.
		En esta memoria tenemos el sistema operativo y todo el software necesario para el sistema de adquisición. Esto hace que el porcentaje
		de ocupación de esta memoria sea muy alto. Es conveniente hacer uso de la ranura microSD (uSD) para extender  la capacidad de
		almacenamiento. A continuación se detallan los pasos a seguir.
		\par
		El primer paso es arrancar la BeagleBone Black sin la tarjeta uSD. Después debemos hacer uso del comando \texttt{fdisk}, programa que
		nos permite manipular las particiones de los dispositivos de almacenamiento.
		\begin{lstlisting}[style=myBash]
# Listar las particiones de los dispositivos presentes.
$ fdisk -l
Disk /dev/mmcbk0: 3867 MB, 3867148288 bytes #eMMC

	Device Boot Start     End  Blocks Id System
/dev/mmcblk0p1    *  2048  198655   98304  e W95 FAT16 (LBA)
/dev/mmcblk0p2     198656 7553023 3677184 83 Linux
		\end{lstlisting}
		Seguidamente debemos insertar la tarjeta uSD y volver a usar el mismo comando.
		\begin{lstlisting}[style=myBash]
# Listar las particiones de los dispositivos presentes.
$ fdisk -l
Disk /dev/mmcbk0: 3867 MB, 3867148288 bytes #eMMC
	Device Boot Start     End  Blocks Id System
/dev/mmcblk0p1    *  2048  198655   98304  e W95 FAT16 (LBA)
/dev/mmcblk0p2     198656 7553023 3677184 83 Linux

# Puede variar dependiendo de como tengamos particionada la uSD.
# No importa porque vamos a borrar las particiones existentes.
Disk /dev/mmcblk1: 7948 MB, 7948206080 bytes #uSD
	Device Boot Start      End  Blocks Id System
/dev/mmcblk1p1       2048 15523839   98304  c W95 FAT32 (LBA)
		\end{lstlisting}
		Usando \texttt{fdisk} debemos borrar todas las particiones de la uSD y después crear dos nuevas. El estado final de la uSD esta
		presentado a continuación.
		\begin{lstlisting}[style=myBash]
$ fdisk -l
Disk /dev/mmcblk1: 7948 MB, 7948206080 bytes #uSD
        Device Boot  Start      End  Blocks Id  System
/dev/mmcblk1p1        2048   198655   98304  e  W95 FAT16 (LBA)
/dev/mmcblk1p2      198656 15523839 7662592 83  Linux
		\end{lstlisting}
		Una vez creadas las particiones estas debes ser formateada. A continuación presentamos las instrucciones necesarias utilizando
		\texttt{mkfs}.
		\begin{lstlisting}[style=myBash]
$ mkfs -t vfat /dev/mmcblk1p1
$ mkfs -t ext3 /dev/mmcblk1p2
		\end{lstlisting}
		Antes de seguir es conveniente explicar algunos aspectos del proceso de arranque en la BeagleBone Black. Después de una serie de
		pasos, que están fuera del alcance de este trabajo, es cargado desde la eMMC(partición FAT16) el programa U-boot. Este es el encargado
		de cargar el núcleo Linux y proporcionar información sobre el sistema de archivos Linux. Este programa busca el archivo
		\texttt{uEnv.txt} desde el que lee una serie de parámetros de configuración como la posición del propio núcleo Linux que queremos
		cargar. Algunos de estos parámetros pueden ser pasados al propio núcleo siempre y cuando este los acepte, por ejemplo la siguiente
		línea nos permite habilitar dos de los puertos series que la BeagleBone Black ofrece.
		\begin{lstlisting}[style=myFile]
optargs=quiet drm.debug=7 capemgr.enable_partno=BB-UART2,BB-UART1
		\end{lstlisting}
		Acabamos de explicar este proceso porque ante la presencia de una uSD el U-boot busca el archivo \texttt{uEnv.txt} en la uSD. Si el
		archivo no esta presente el programa no puede seguir. Es por esta razón por la que hemos creado la partición FAT16 en la uSD. Para
		crear el archivo tenemos que seguir los siguientes pasos.
		\begin{lstlisting}[style=myBash]
$ mkdir trash
$ mount /dev/mmcblk1p1 trash
$ cd trash
$ vim uEnv.txt
$ cd ..
$ umount /dev/mmcblk1p1
$ rm -R trash
		\end{lstlisting}
		El contenido del archivo debe ser el siguiente.
		\begin{lstlisting}[style=myFile]
mmcdev=1
bootpart=1:2
mmcroot=/dev/mmcblk1p2 ro
optargs=quiet 
		\end{lstlisting}
		La línea interesante del archivo es la que asigna valor a \texttt{mmcroot}. Esta variable debe especificar donde esta el núcleo Linux
		que querremos arrancar. El valor asignado es \texttt{/dev/mmcblk1p2}, los lectores atentos se habrán dado cuenta que este valor
		corresponde a la segunda partición dentro de la uSD, sitio donde no esta el núcleo.
		\par 
		En este punto la eMMC es el dispositivo \texttt{/dev/mmcbk0} y la uSD es el dispositivo \texttt{/dev/mmcbk1}, esto es así porque
		arrancamos sin la uSD y añadimos esta posteriormente. Si arrancamos con la uSD presente esta es reconocida como el dispositivo
		\texttt{/dev/mmcbk0} y la eMMC es \texttt{/dev/mmcbk1}. Teniendo esto en consideración la próxima vez que arranquemos la placa con la
		uSD presente la variable \texttt{mmcroot} apuntara correctamente a la partición que contiene el núcleo Linux.
		\par
		En este punto tan solo nos queda definir un punto de montaje para la segunda partición de la uSD, la partición que guardara los datos
		del sistema de adquisición. Para este propósito tenemos que editar el archivo \texttt{/etc/fstab}, tenemos que añadir la siguiente
		línea al final del archivo. La partición será montada en el directorio \texttt{/server/data}.
		\begin{lstlisting}[style=myFile]
/dev/mmcblk0p1   /server/data   auto   defaults     0  0
		\end{lstlisting}
		Finalmente para que se apliquen los cambios debemos reiniciar la placa con la uSD insertada.


\chapter{Despliegue aplicación Web.}
\section{\emph{Back-end}}
	\label{app_back}
	El propósito de este punto es detallar el proceso que debe seguirse para desplegar el \emph{back-end} de la aplicación Web. Debido a que en
	este trabajo tan solo nos centramos en el proceso de desarrollo, el proceso de despliegue presentado es para desplegar una versión de
	desarrollo que permita seguir trabajando en el proyecto.
	\par
	El primer paso es desplegar un servidor MySql que contiene los datos que vamos a usar. Es recomendable instalar este servidor en la misma
	máquina desde la que vamos a trabajar, los pasos presentados a continuación asumen que hemos procedido de esta manera. El siguiente comando
	nos permite instalar un servidor MySql. Junto a este es instalado un programa cliente que permite conectar se al servidor desde una consola.
	\begin{lstlisting}[style=myBash]
$ sudo apt-get install mysql-server mysql-client
	\end{lstlisting}
	El siguiente paso es acceder como usuario \texttt{root} al servidor Mysql, para este propósito debemos usar el programa cliente que acabamos
	de instalar. Seguidamente debemos crear dos bases de datos, es conveniente que usemos los nombres \texttt{nmdadb} y \texttt{nmdb}. Finalmente
	debemos crear un usuario que el \emph{back-end} utilizara para conectarse al servidor. Debemos dar permisos al usuario sobre las bases de
	datos que acabamos de crear, en este caso damos privilegios completos al usuario.
	\begin{lstlisting}[style=myBash]
$ mysql -u root -p
mysql> CREATE DATABASE nmdadb;
mysql> CREATE DATABASE nmdb;
mysql> CREATE USER 'hristo'@'localhost' IDENTIFIED 'pass';
mysql>  GRANT ALL PRIVILEGES ON *.* TO 'hristo'@'localhost';
mysql> FLUSH PRIVILEGES;
	\end{lstlisting}
	El siguiente paso es introducir datos en las bases de datos. En este trabajo hemos utilizado los datos generados por CALMA. Los datos son
	almacenados en un servidor MySql, para transferir el contenido de un servidor a otro hemos utilizado el comando \texttt{mysqldump} que permite
	verter el contenido de una base en un archivo. A continuación podemos ver un ejemplo de uso.
	\begin{lstlisting}[style=myBash]
$ mysqldump -h remotehost -u hristo -ppass nmdadb > nmdadb.sql
$ mysql -h localhost -u hristo -ppass nmdadb < nmdadb.sql
	\end{lstlisting}
	Antes de continuar debemos asegurarnos de que las tablas dentro de las bases de datos siguen el esquema presentado en el capítulo
	\ref{backend}. También debemos asegurarnos de que los datos se han transferido correctamente.
	\par
	Una vez configuradas las bases de datos podemos seguir con la instalación del software que compone el \emph{back-end}. Este software es
	desarrollado en PHP, que podemos instalar usando el comando presentado a continuación. Este comando también instala la extensión
	\texttt{php-mysql} que nos permite conectarnos al servidor MySql. Es importante destacar que necesitamos una versión PHP superior a la
	\texttt{5.3.23}, requerida por ZendFramework y Apigility.
	\begin{lstlisting}[style=myBash]
$ apt-get install php5 php5-mysql
	\end{lstlisting}
	El siguiente paso es clonar el repositorio Git que contiene nuestro proyecto. Para este propósito debemos utilizar el comando presentado a
	continuación.
	\begin{lstlisting}[style=myBash]
$ git clone https://github.com/opobla/nmPanel.git directorio_de_instalacion
	\end{lstlisting}
	Para seguir es conveniente navegar hasta el directorio raíz del proyecto que acabamos de clonar. El siguiente paso es instalar todas las
	dependencias de nuestro software. Este proceso puede resultar tedioso, razón por la que hemos utilizado Composer, herramienta que permite
	gestionar las dependencias en PHP. Es conveniente destacar que Composer no es un gestor de paquetes, este gestiona las dependencias a nivel de
	proyecto, instalando estas en el directorio \texttt{ventor} del proyecto. Podemos instalar Composer de dos maneras, globalmente como un
	comando y localmente como parte del proyecto. A continuación podemos ver como instalar este de forma local. El segundo comando es el que
	instala las dependencias. 
	\begin{lstlisting}[style=myBash]
$ curl -sS https://getcomposer.org/installer | php
$ php composer.phar install
	\end{lstlisting}
	Para declarar las dependencias del proyecto es usado el archivo \texttt{Composer.json}. A continuación podemos ver un ejemplo de como declarar
	una dependencia básica. 
	\begin{lstlisting}[style=myFile]
{
   "require": {
      "monolog/monolog": "1.2.*"
   }
}
	\end{lstlisting}
	Para trabajar con la herramienta Apigility, es necesario habilitar el modo de desarrollo, usando el comando presentado a continuación. 
	\begin{lstlisting}[style=myBash]
$ php public/index.php development enable
	\end{lstlisting}
	Finalmente debemos configurar un servidor Web desde el que acceder a nuestra aplicación. Podemos utilizar un servidor Web completo como
	Apache, pero en este trabajo por comodidad hemos utilizado el servidor Web interno que PHP ofrece. Para arrancar este debemos utilizar el
	siguiente comando. Es importante que  el directorio raíz sea el directorio \texttt{public} de nuestro proyecto.
	\begin{lstlisting}[style=myBash]
$ php -S 0.0.0.0:8080 -t public/ public/index.php
	\end{lstlisting}
	De esta manera todo el contenido presente en el directorio \texttt{public} será visible desde un navegador Web. Un ejemplo es nuestro
	\texttt{front-end}, que desplegaremos en el subdirectorio \texttt{public/nmCpanel}. A continuación presentamos las URL que nos permiten
	acceder a Apigility y a nuestra herramienta siempre y cuando esta este desplegada.
	\begin{lstlisting}[style=myFile]
http://localhost:8080/
http://localhost:8080/nmCpanel/index.html
	\end{lstlisting}
	Eventualmente para que todo funcione bien deberemos verificar que la configuración referente a las bases de datos que creamos anteriormente es
	correcta. Para este propósito debemos examinar los archivos \texttt{config/autoload/global.php} y \texttt{config/autoload/local.php}.

\section{\emph{Front-end}}
	\label{app_front}
	El propósito de este punto es detallar el proceso que permite desplegar el \emph{front-end} de la aplicación Web. Al igual que en la sección
	anterior describimos el proceso que nos permite desplegar una versión de desarrollo que permita seguir trabajando en el proyecto.
	\par
	La aplicación ha sido desarrollada con Sencha Architect, entorno de desarrollo que facilita el proceso de desarrollo de aplicaciones con
	ExtJs. El primer paso es descargar e instalar la última versión de la herramienta. La herramienta se debe descargar desde la página oficial de
	Sencha. La instalación es llevada a cabo mediante un entorno gráfico. Al instalar la herramienta la librería ExtJs será instalada
	automáticamente.
	\par
	El siguiente paso es clonar el repositorio Git que contiene nuestro proyecto. Para este propósito debemos utilizar el siguiente comando.
	\begin{lstlisting}[style=myBash]
$ git clone https://github.com/opobla/nmcpanel.git directorio_de_proyecto
	\end{lstlisting}
	Finalmente debemos abrir el proyecto con Sencha Architect.  Seguidamente podemos empezar a trabajar. Para desplegar una versión de desarrollo
	debemos utilizar el botón de desplegar.

